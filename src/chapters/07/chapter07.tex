\documentclass[class=book, crop=false, oneside, 12pt]{standalone}
\usepackage{standalone}
\usepackage{../../style}
\graphicspath{{./assets/images/}}

% arara: pdflatex: { synctex: yes, shell: yes }
% arara: latexmk: { clean: partial }
\begin{document}
\chapter{Analisi sintattica: parsing top-down}

\section{Il parsing}
Il parsing (o analisi sintattica) è quel processo che, data una grammatica \(\G = (V, T, S, \mathcal{P})\) e una parola \(w\), ci permette di dire se \(w \in \mathcal{L(G)}\) e, se questo è vero, fornire il suo albero di derivazione. Solitamente gli approcci al parsing che vengono presi in considerazione nell'ambito dei linguaggi di programmazione sono due: 
\begin{itemize}
    \item \textbf{top-down}: consiste nella costruzione di una derivazione leftmost da uno start symbol della grammatica e quindi procede dalla radice verso le foglie dell'albero di derivazione; a prima vista, si direbbe che sia l'approccio più intuitivo;
    \item \textbf{bottom-up}: consiste invece nella costruzione di una derivazione rightmost (in ordine inverso) della stringa dalle foglie alla radice.
\end{itemize}
A questo punto è necessario aggiungere che il parsing non si limita a questi due approcci, ma esiste anche in forma più generale utilizzando delle tattiche che vengono impiegate nel caso dei linguaggi naturali. Gli approcci descritti non permettono di considerare tutti i possibili linguaggi liberi, ma solamente delle sottoclassi: per questo si ha un'analisi sintattica estremamente efficiente dal punto di vista computazionale.

\subsection{Top-Down Parsing}
\subsubsection{Esempio 1}
Sia \(w\) = \(bd\) e sia \(\G\): 
\begin{align*}
    \G: S &\rightarrow Ad \mid Bd \\
    A &\rightarrow a \\
    B &\rightarrow b
\end{align*}

Per verificare se \(w \in \mathcal{L(G)}\) con un approccio top-down dobbiamo ottenere una derivazione leftmost a partire dallo start symbol. Ovviamente, si noterà subito che non è possibile scegliere \(Ad\) come derivazione iniziale dello start symbol, perché a quel punto l'unica parola ottenibile sarebbe la parola \(ad\); dobbiamo invece optare per la seconda derivazione. La derivazione completa ci porta a
\begin{align*}
    S \Rightarrow Bd \Rightarrow bd
\end{align*}
Visto che abbiamo dimostrato che \(w \in \mathcal{L(G)}\) e che tale derivazione esiste, allora possiamo fornire il suo albero di derivazione che, per questo esempio, risulta davvero molto semplice.

\begin{figure}[H]
    \centering
    \subimport{assets/figures/}{derivation_tree_es1.tex}
    \caption{Albero di derivazione per esercizio 1}
    \label{par-td-es1}
\end{figure}

\subsubsection{Esempio 2}
Sia \(w\) = \(id + id * id\) e sia \(\G\):
\begin{align*}
    \G: E &\rightarrow TE' \\
    E' &\rightarrow +TE' \mid \varepsilon \\
    T &\rightarrow FT' \\
    T' &\rightarrow *FT' \mid \varepsilon \\
    F &\rightarrow (E) \mid id
\end{align*}

Qui non è così intuitivo riuscire a dire se esiste una derivazione leftmost: che cosa mi conviene espandere? Avremo modo di parlare meglio di questo esempio non appena introdurremo il \textbf{predictive top-down parsing}.

\subsubsection{Esempio 3}
Sia \(w\) = \(cad\) e sia \(\G\):
\begin{align*}
    S &\rightarrow cAd \\
    A &\rightarrow ab \mid a
\end{align*}

Nonostante questo esempio sia molto intuitivo, dobbiamo sforzarci di ragionare nei panni dell'algoritmo di parsing: dopo la prima derivazione, infatti, entrambe le opzioni che vengono proposte per poter derivare il non-terminale \(A\) sono apparentemente valide in quanto iniziano entrambe per \(a\). Quale delle due dovrei dunque scegliere? Quanto devo continuare l'esecuzione prima di accorgermi se ho fatto una scelta corretta oppure no? Il nostro algoritmo potrebbe anche trovarsi nel caso di dover fare \emph{backtrack}, perché semplicemente non c'era un modo semplice e generale per capire cosa dover scegliere: ovviamente questo tipo di tecnica funziona, ma vogliamo un approccio efficiente e il backtrack, come sappiamo, in questo non ci aiuta.

\subsection{Predictive Top-Down Parsing}
Nel caso del predictive top-down parsing non è mai necessario applicare la tecnica del backtrack, poiché questo fa riferimento a una classe particolare di grammatiche, le quali vengono definite \textbf{LL(1) grammars}; vengono così chiamate per via della procedura impiegata per analizzarle, in cui:
\begin{itemize}
    \item guardiamo le parole da sinistra a destra;
    \item eseguiamo una produzione leftmost;
    \item guardiamo un solo non-terminale per decidere quale produzione utilizzare.
\end{itemize}
Tali grammatiche prevedono una tipologia di parsing per cui non è necessario backtrack, e per di più è \textbf{completamente deterministica}.

Queste grammatiche vengono classificate a seconda del grado di determinismo che consentono; all'inizio del corso abbiamo visto la differenza tra le grammatiche senza contesto e contestuali, ma adesso le classi che incontreremo da oggi in poi si differenzieranno esclusivamente per il tipo di analizzatore con cui possiamo riconoscere le parole generate da queste grammatiche. 

In questo caso il parsing si basa sul fatto che possiamo costruire una tabella di parsing che ci guida nell'analisi della parola che ci viene data in input; questa strategia ci permette molto efficacemente di dire se la parola appartenga oppure no al linguaggio e, in caso di esito positivo, di costruire la derivazione leftmost richiesta e il conseguente albero di derivazione.

Prendiamo come esempio il caso che abbiamo lasciato in sospeso precedentemente:
\begin{align*}
    \G: E &\rightarrow TE' \\
    E' &\rightarrow +TE' \mid \varepsilon \\
    T &\rightarrow FT' \\
    T' &\rightarrow *FT' \mid \varepsilon \\
    F &\rightarrow (E) \mid id
\end{align*}
\begin{table}[H]
	\centering
	\subimport{assets/tables/}{ptdp-example.tex}
    \caption{Tabella del parsing top-down}
    \label{ptdp-example}
\end{table} 
% \begin{figure}[H]
%     \centering
%     \includegraphics[width=.7\textwidth,keepaspectratio]{parsingTopDownTable.jpg}
%     \caption{parsingTopDownTable}
%     \label{parsingTopDownTable}
% \end{figure}

La tabella è così costruita:
\begin{itemize}
    \item si ha una \textbf{riga} per ogni non-terminale della grammatica;
    \item ed una \textbf{colonna} per ogni terminale della grammatica, a cui si aggiunge il simbolo \$ alla parola fornita in input e utilizzato come terminatore
    \item le entry vuote all'interno della tabella identificano i casi di errore.
\end{itemize}

Avendo a disposizione la tabella di parsing, la cui costruzione verrà trattata successivamente, è possibile utilizzarla per il parsing top-down predittivo.

\subsubsection{Algoritmo per il Predictive Top-Down Parsing}
\begin{itemize}
    \item Input: una stringa \(w\), una tabella \(M\) di parsing top-down per la grammatica \(\G = (V, T, S, \mathcal{P})\);
    \item Output: 
    \begin{itemize}
        \item la derivazione leftmost della stringa \(w \iff w \in \mathcal{L(G)}\);
        \item altrimenti \emph{error()}.
    \end{itemize}
\end{itemize}
Per questo algoritmo utilizzeremo due strutture:
\begin{itemize}
    \item un buffer che conterrà la parola e che terminerà con un terminatore \$;
    \item una pila che andrà a contenere simboli terminali e non; viene inizializzato con lo start symbol \(S\) e il terminatore \$.
\end{itemize}
L'obiettivo del nostro gioco è quello di raggiungere il \$ andando a eseguire dei \texttt{pop()} dalla pila senza mai incappare in degli \texttt{error()}.

\subimport{assets/pseudocode/}{pred-tdparsing.tex}
Nell'algoritmo di parsing si utilizza la variabile \(b\) come primo simbolo delle parola \(w\$\) e si inizializza la variabile \(X\) con la cima dello stack (il caso base corrisponde ad avere \(X = S\)). Finché \(X \neq \$\) (sostanzialmente finché non ho svuotato completamente la pila), sono dati i seguenti casi:

\begin{enumerate}
    \item se \(X = b\), allora tolgo l'elemento dalla cima della pila e imposto \(b\) al simbolo successivo nell'input buffer. Questo corrisponde al caso in cui nella derivazione parziale ho ottenuto un match con un terminale nella parola;
    \item se invece \(X\) è comunque un terminale, allora deve essere che \(X \neq b\), ma ciò produce un errore;
    \item se invece \(X\) è un non-terminale, allora è necessario verificare la tabella di top-down parsing in posizione \(M[X, b]\) e possono valere le seguenti:
    \begin{itemize}
        \item \(M[X, b]\) = \texttt{error} e allora viene restituito \texttt{error()};
        \item \(M[X, b] = X \rightarrow Y_1...Y_k\) e quindi è una produzione che verrà utilizzata per continuare la derivazione: questa viene stampata in output, viene rimosso l'elemento \(X\) dalla testa della pila e infine viene inserito il body della produzione in ordine inverso (cioè in modo che \(Y_1\) sia l'elemento in cima allo stack).
    \end{itemize}
\end{enumerate}

Infine, come ultima operazione \(X\) viene assegnato all'elemento in cima alla pila e si ripete. In Tab.\ref{tab:ptdp-ex1} possiamo vedere un esempio che fa uso dell'algoritmo appena descritto.
\begin{table}[H]
	\centering
	\subimport{assets/tables/}{ptdp-ex1.tex}
    \caption{Tabella delle strutture a ogni passo}
    \label{tab:ptdp-ex1}
\end{table} 
Questo algoritmo ha complessità lineare.
% \begin{figure}[H]
%     \centering
%     \includegraphics[width=.7\textwidth,keepaspectratio]{TopDownExercise.png}
%     \caption{TopDownExercise}
%     \label{TopDownExercise}
% \end{figure}

\subsection{Tabella di parsing}
Qual è quindi la procedura che dobbiamo seguire per costruire codesta tabella di parsing? In quale posizione dobbiamo mettere le produzioni della grammatica per fare sì che l'algoritmo funzioni?

Ricordiamo che la cella \(M[A, b]\) della parsing table viene consultata per espandere il non-terminale \(A\) sapendo che il prossimo terminale nell'input buffer è \(b\). Andiamo dunque a valorizzare la cella \(M[A, b]\) = \(A \rightarrow \alpha\) se:

\begin{itemize}
    \item il body della nostra produzione con driver \(A\), è tale per cui esiste una derivazione del tipo \(\alpha \Rightarrow^* b \beta\), ossia partendo da \(\alpha\) si riesce, con zero o più passi, a ottenere una stringa che inizia per \(b\);
    \item oppure \(\alpha \Rightarrow^* \varepsilon\) ed è possibile avere \(S \Rightarrow^* w A \gamma\) (ottenuta da una derivazione di tipo leftmost) con \(\gamma \Rightarrow^* b \beta\)
\end{itemize}

Le celle per cui non è possibile inserire una produzione (cioè quelle vuote) verranno valorizzate a \texttt{error()}.

\subsubsection{Esercizio 1}

Sia data la seguente grammatica:
\begin{align*}
    \G: S &\rightarrow aA \mid bB \\
    A &\rightarrow c \\
    B &\rightarrow c
\end{align*}

Andiamo a vedere come costruire la tabella, aiutandoci dal fatto che è molto semplice ricavare a vista il linguaggio denotato dalla seguente grammatica, che include semplicemente le due parole \(ac\) e \(bc\). Partiamo dalle produzioni di \(S\):
\begin{itemize}
    \item se applichiamo la definizione precedente abbiamo che \(M[S, a] = S \rightarrow aA\), in quanto solamente utilizzando quella produzione riusciremmo ad avere una stringa che comincia per \(a\) (\(aA \Rightarrow ac\));
    \item lo stesso ragionamento può essere applicato per \(M[S, b] = S \rightarrow bB\);
    \item dal momento che non è possibile creare delle stringhe che cominciano per il terminale \(c\), allora avremo che \(M[S, c] = error()\);
\end{itemize}
A questo punto è possibile passare agli altri due non-terminali della grammatica: 
\begin{itemize}
    \item nel caso di \(A\) è possibile notare che esiste una sola produzione che permette di ottenere soltanto \(c\), quindi avremo che \(M[A, c] = A \rightarrow c\);
    \item nel secondo caso è possibile applicare la stessa logica, per cui avremo che \(M[B, c] = B \to c\).
\end{itemize} 
La tabella finale sarà quindi costruita in questo modo:
\begin{table}[H]
	\centering
	\subimport{assets/tables/}{filling-parsing-table1.tex}
    \caption{Parsing table per esercizio 1}
    \label{filling-parsing-table1}
\end{table} 
Ricordiamoci che non è possibile che più produzioni possano essere inserite all'interno della stessa cella, in quanto ci stiamo occupando delle grammatiche \(LL(1)\), per cui è possibile applicare un parsing di tipo deterministico; se incrociamo una cella che contiene più di una entry, allora non è possibile fare un parsing deterministico.

Quando si arriva ad una error entry vuol dire che sulla testa della pila c'è un non-terminale che, indipendentemente da come decida di espanderlo, non mi porterà mai alla stringa che sto cercando di ottenere. 

\subsubsection{Esercizio 2}
Sia data la seguente grammatica:
\begin{align*}
    \G: S &\rightarrow aAb \\
    A &\rightarrow \varepsilon
\end{align*}

In questo esempio, invece, il linguaggio denotato dalla nostra grammatica è composto solamente dalla parola \(ab\). Per la definizione di \(\alpha\), l'unica cella che ha come non-terminale (nonché start symbol) \(S\) è la cella \(M[S, a] = S \rightarrow aAb\), perché l'unica stringa che è possibile ottenere inizia per \(a\). Da qui è possibile intuire anche che, per ottenere la parola \(ab\), è necessario che \(M[A, b] = S \rightarrow \varepsilon\); in tutti gli altri casi verrà ritornato un errore, in quanto tale parola non apparterrebbe al linguaggio denotato dalla grammatica. 

\begin{table}[H]
	\centering
	\subimport{assets/tables/}{filling-parsing-table2.tex}
    \caption{Parsing table per esercizio 2}
    \label{filling-parsing-table2}
\end{table}

\section{First(\(\alpha\))}
\subsection{Definizione}
\begin{definition}
    Chiamiamo \(first(\alpha)\) quell'insieme di terminali che sono posti all'inizio delle stringhe derivate da \(\alpha\).
    
    Inoltre, se \(\alpha \Rightarrow^* \varepsilon\), allora \(\varepsilon \in \textrm{first}(\alpha)\): ciò vuol dire che \(\alpha\) è un non-terminale annullabile (nullable) e quindi, dopo una serie di passi, sarà pari a \(\varepsilon\).
\end{definition}

Il concetto di \(first\) può essere definito ricorsivamente come segue:

\begin{labeling}{step}
    \item[Base] casi base:
    \begin{itemize}
        \item \(first(\varepsilon) = \{\varepsilon\}\)
        \item \(first(a) = \{a\}\)
    \end{itemize}
    \item[Step] \(first(A) = \bigcup_{A \rightarrow \alpha} first(\alpha)\)
\end{labeling}

    Nell'ultimo caso, quello ricorsivo, si ha che \(first(A)\) è dato dall'unione di tutti i \(first(\alpha)\), dove \(\alpha\) è il body di tutte quelle produzioni della grammatica che hanno \(A\) come driver.

\subsection{Algoritmo di calcolo di first(\(\alpha\))}

Mostriamo ora un algoritmo che ci permette di calcolare \(first(Y_1 \ldots Y_n)\) (con \(Y_i \in V\)), cioè l'insieme dei first per una certa parola considerata.

\subimport{assets/pseudocode/}{first.tex}

Andiamo a vedere più da vicino quale idea stiamo seguendo in questo algoritmo. Dopo aver inizializzato l'insieme dei \(first(Y_1 \ldots Y_n)\), iteriamo su tutti gli elementi \(Y_j \in V\) della parola: 
\begin{itemize}
    \item finché non abbiamo esaminato tutti gli \(Y_j : 1 \leq j \leq n\), si aggiunge \(first(Y_j) \setminus\{\varepsilon\}\) ai \(first(Y_1 \ldots Y_n)\) e poi si controlla se \(\varepsilon \in first(Y_j)\);
    \item se il controllo restituisce \texttt{true}, allora vuol dire che è necessario continuare la ricerca dei first, perché è possibile che in almeno un caso \(Y_j = \varepsilon\) e quindi è possibile che nessuno dei \(first(Y_j) \in first(Y_1, \ldots Y_n)\);
    \item se il controllo restituisce invece \texttt{false}, allora possiamo fermarci, in quanto abbiamo trovato tutti i \(first(Y_1 \ldots Y_n)\).
\end{itemize} 
L'ultimo controllo serve per verificare se, \(\forall\; Y_j \in (Y_1 \ldots Y_n)\), non sono in realtà tutti annullabili; in tal caso, sarebbe necessario aggiungere anche \(\varepsilon\) ai \(first(Y_1 \ldots Y_n)\), in quanto \((Y_1 \ldots Y_n)\) sarebbe annullabile.

\subsection{Esempi di calcolo}
Sia data la seguente grammatica:
\begin{align*}
    \G: E &\rightarrow TE' \\
    E' &\rightarrow +TE' \mid \varepsilon \\
    T &\rightarrow FT' \\
    T' &\rightarrow *FT' \mid \varepsilon\\
    F &\rightarrow (E) \mid id
\end{align*}

Osservando la grammatica è possibile, per la definizione espressa precedentemente, affermare con abbastanza semplicità che: 
\begin{align*}
    \{\varepsilon\} &\in first(E')\textrm{,} &\textrm{da}\; E' &\to \varepsilon \\
    \{\varepsilon\} &\in first(T')\textrm{,} &\textrm{da}\; T' &\to \varepsilon \\
    \{id\} &\in first(F)\textrm{,} &\textrm{da } F\; &\to id
\end{align*}

Insomma, l'idea generale è quella di partire da quelle produzioni che necessitano di meno passaggi possibili, cioè quelle che hanno un terminale (o \(\varepsilon\)) come body, e il cui driver può essere utilizzato per risolvere i \(first\) di un'altra produzione.

Per calcolare i \(first\) della grammatica proposta ci conviene partire da \(first(F)\): per il passo induttivo dobbiamo calcolare \(\bigcup_{F \to \alpha} first(\alpha)\), cioè \(first((E)) \cup first(id)\). Separiamo dunque il calcolo nelle due componenti:
\begin{itemize}
    \item nel primo caso analizziamo \(first((E))\), per cui dobbiamo procedere al calcolo di una parola composta da tre elementi (\(Y_1 = '(', Y_2 = E, Y_3 = ')'\), il lettore perdoni l'abuso di notazione nell'utilizzo degli apici) e dunque per prima cosa dobbiamo considerare \(first(Y_1)\); tuttavia, \(Y_1 = '('\) è un terminale, per cui avremo che \(first(Y_1) = \{Y_1\}\): questo significa che \(\varepsilon \notin \{Y_1\}\) e che possiamo aggiungere \(\{Y_1\}\) a \(first(F)\) e interrompere subito l'analisi dei \(first((E))\);
    \item nel secondo caso, invece, ci troviamo di fronte al caso base \(first(id) = \{id\}\) e dunque possiamo subito aggiungerlo a \(first(F)\).
\end{itemize}
In questo caso il risultato finale è dunque \(first(F) = \{(\} \cup \{id\} = \{(, id\}\); in modo analogo è possibile arrivare alla soluzione per \(first(T')\) e \(first(E')\).

\subparagraph{}
A questo punto, il miglior modo per procedere con l'algoritmo è, in mancanza di casi base, quello di trovare una produzione del tipo \(A \rightarrow F\alpha\); in quel modo potremmo sfruttare le conoscenze appena ottenute relativamente a \(first(F)\). Applicando dunque l'algoritmo per \(first(T)\) ci si accorge che è necessario analizzare la produzione \(T \rightarrow FT' = Y_1Y_2\): iniziamo da \(first(Y_1)\). Dal momento che che abbiamo già calcolato \(first(F)\) e, poiché \(\varepsilon \notin first(F)\), possiamo arrestare l'algoritmo e affermare che \(first(T) = first(FT') = first(F) = \{(, id\}\). Svolgendo dunque l'esercizio nella sua interezza è possibile ottenere il seguente risultato:
\begin{table}[H]
	\centering
	\subimport{assets/tables/}{computing-first.tex}
    \caption{Esercizio sui first}
    \label{computing-first}
\end{table}
Cosa succede se invece, al posto della produzione \(T \to FT'\), diventa \(T \to T’F\)? Lasciando il resto dell'esercizio invariato, proviamo a calcolare \(first(T)\). In questo caso avremmo dovuto analizzare prima i \(first(T’) = \{\varepsilon, \ast\}\) e aggiungere \(first(T’) \setminus \{\varepsilon\} = \{\ast\}\) a \(first(T)\); tuttavia, dal momento che \(\varepsilon \in first(T')\), l’algoritmo non può arrestarsi (perché potrebbe darsi che \(T'\) sia nullo), ma è necessario considerare anche i \(first(F)\). Come svolto precedenetemente aggiungiamo \(first(F) \setminus \{\varepsilon\} = \{id, (\}\) a \(first(T)\); essendo che \(\varepsilon \notin first(F)\) posso interrompere l'algoritmo e segnare che \(first(T) = \{\ast, id, (\}\)

\section{Follow(\(A\))}
\subsection{Definizione}
A differenza dei \emph{first}, che sono definiti per stringhe generiche di terminali e non-terminali, i \emph{follow} sono calcolati solamente per i non-terminali: scriveremo dunque \(follow(A)\).
\begin{definition}
    Con \(follow(A)\) indichiamo l'insieme dei terminali che possono seguire A in qualche derivazione.    
\end{definition}
I \emph{first} evidenziano quali sono i terminali per cui iniziano le stringhe derivabili da certi elementi (stringhe o non-terminali), i \emph{follow} invece indicano quali sono i terminali che possono seguire.

\subsection{Algoritmo per il calcolo dei follow(\(A\))}
\subimport{assets/pseudocode/}{follow.tex}
L'algoritmo comincia inizializzando \(follow(S) = \$\): dal momento che dallo start symbol si possono generare tutte le possibili parole appartenenti al linguaggio della grammatica, allora possiamo aspettarci di trovare il terminatore di stringa \$ dopo una qualsiasi parola. Negli altri casi, invece, \(\forall A\) tale che \(A\) è un non-terminale, inizializziamo \(follow(A) = \emptyset\).

A questo punto, è necessario sottolineare che, visto che siamo interessati a quei terminali che seguono un particolare non terminale, ci interessano solamente le produzioni della grammatica per cui il generico non terminale \(A\) non compare nel driver della produzione, bensì nel body. Per ogni \(B \rightarrow \alpha A \beta\), si eseguono le seguenti operazioni: 

\begin{itemize}
    \item se \(\beta \neq \varepsilon\), allora aggiungiamo \(first(\beta) \setminus \{\varepsilon\}\) a \(follow(A)\);
    \item se \(\beta = \varepsilon\) or \(\varepsilon \in first(\beta)\), allora aggiungiamo \(follow(B)\) a \(follow(A)\); questo perché ciò che segue \(B\) potrà seguire anche \(A\) (si tenga presente che se \(\beta = \varepsilon\), allora \(A\) è la radice dell'ultimo sottoalbero generato da \(B\).
\end{itemize}   
% se si immagina un albero di derivazione qusto è formato da un sottoalbero per alpha, poi abbiamo un sottoalbero di A e poi quello di beta che contiene tutto quello che derivad a beta: tutti i terminali che sono diversi da epsilon e derivano da beta allora li mettiamo in A. Se invece beta è uguale a epsilon oppure epsilon appartiene ai first di beta allora dobbiamo aggiungere i follow di B ai follow di A perché ciò che segue B potrà anche seguire A perché se noi andiamo a considerare i sottoalberi che vengono generati andiamo a scoprire che A è la radice dell'utlimo sottoalbero della B e quindi quello che può seguire questa particolare occorrenza di B ... Si procede aggiungendo degli elementi ogni volta che andiamo a considerare delle partciolari istanze nella nostra grammatica e ci fermiamo fino a quando non possiamo inserire altro. 
\subsection{Esercizi su first/follow}
\subsubsection{Esercizio first/follow 1}
\label{first-follow-ex-1}
Utilizziamo sempre la grammatica dell'esempio precedente, che qui riportiamo per comodità del lettore, e andiamone a definire i follow.
\begin{align*}
    \G: E &\rightarrow TE' \\
    E' &\rightarrow +TE' \mid \varepsilon \\
    T &\rightarrow FT' \\
    T' &\rightarrow *FT' \mid \varepsilon\\
    F &\rightarrow (E) \mid id
\end{align*}
\begin{enumerate}
    \item Dall'algoritmo sappiamo di dover inizializzare \(follow(E) = \$\).
    \item Iniziamo guardando la prima produzione, cioè \(E \rightarrow TE'\): se poniamo \(A = E'\), allora \(\beta = \varepsilon\) e quindi dobbiamo aggiungere i \(follow(E)\) (quelli del driver) a \(follow(E')\) (quelli del body).
    \item Sempre soffermandoci sulla stessa produzione poniamo \(A = T\): questo vuol dire che \(\beta = E' \neq \varepsilon\) e quindi, ricadendo nel primo caso, aggiungiamo \(first(E') \setminus \{\varepsilon\} = \{+\}\) a \(follow(T)\); tuttavia, essendo che \(\varepsilon \in first(E')\), allora ricadiamo anche nel secondo caso, per cui aggiungiamo \(follow(E)\) a \(follow(T)\).
    \item Passiamo ora alla produzione \(E' \rightarrow +TE'\): poniamo quindi \(A = E'\) e \(\beta = \varepsilon\); ricadendo nel secondo caso dovremmo aggiungere \(follow(E')\) a \(follow(E')\) ma, dato che questo non fornisce informazioni aggiuntive, possiamo scartare questa informazione.
    \item Facendo sempre riferimento ad \(E' \rightarrow +TE'\), esaminiamo il caso per cui \(A = T\): visto che \(\beta = E'\), tale caso è identico a quello esaminato al punto 3 e porterà all'aggiunta dei medesimi terminali.
    \item L'ultimo elemento che non abbiamo preso in considerazione per la produzione corrente è terminale \(+\) ma, in quanto terminale, non possiamo calcolarne i follow.
    \item Continuiamo ad applicare l'algoritmo come visto nei punti precedenti per calcolare i follow delle produzioni rimanenti.
\end{enumerate}

Un caso che potrebbe risultare interessante riguarda la produzione \(F \rightarrow (E)\). Poniamo come fatto precedentemente \(A = E\): essendo che \(\beta = \textrm{ } ) \neq \varepsilon\), è necessario aggiungere \(first(\beta) \setminus \{\varepsilon\} = \{)\}\) a \(follow(E)\); visto che la seconda condizione non è vera, possiamo fermarci ed affermare che \(follow(E) = \{\$, )\}\).

Inoltre, nei punti in cui si devono aggiungere ai follow di un non-terminale i follow di un altro si lasciano delle annotazioni che si andranno a risolvere una volta terminata l'analisi di tutte le produzioni. Il risultato finale verrà rappresentato come segue:
\begin{table}[H]
	\centering
	\subimport{assets/tables/}{computing-follow.tex}
    \caption{Esercizio sui follow, step intermedio}
    \label{computing-follow}
\end{table}
A questo punto è possibile eliminare mano a mano le dipendenze e le varie ripetizioni presenti all'interno delle computazioni dei \emph{follow}, ottenendo il seguente risultato: 
\begin{table}[H]
	\centering
	\subimport{assets/tables/}{follow.tex}
    \caption{Esercizio sui follow, risultato finale}
    \label{follow}
\end{table}
Come è possibile osservare dalla tabella conclusiva, alcuni passaggi potevano essere tranquillamente evitati (ad esempio l'aggiunta più volte degli stessi terminali): la scelta di evitare tali passaggi oppure di eseguirli per una maggiore sicurezza è a discrezione del lettore.

\subsubsection{Esercizio first/follow 2}
\label{first-follow-ex-2}
Prendiamo ora in analisi la seguente grammatica:
\begin{align*}
       \G: S &\to aABb \\
       A &\to Ac \mid d \\
       B &\to CD \\
       C &\to e \mid \varepsilon \\
       D &\to f \mid \varepsilon
\end{align*}
Questa volta ripassiamo anche il calcolo dei first.
\begin{enumerate}
    \item Partiamo da \(S\): ricordiamoci che per calcolare i \emph{first} di un non-terminale dobbiamo vedere i \emph{first} di tutte le sue produzioni; in particolare, so che qualunque stringa derivata da \(S\) inizierà con \(a\) (da \(S \to aABb\)), e siccome \(a\) è un terminale ed è diverso da \(\varepsilon\), non proseguo oltre nella ricerca di \emph{first} per \(S\);
    \item per calcolare i \emph{first} di \(A\), invece, ho due produzioni da vagliare: la prima (che derivo da \(A \to Ac\)) mi dice che i \emph{first} di \(A\) contengono anche i \emph{first} dello stesso \(A\) (non aggiunge informazioni), la seconda mi dice che \(d\) può essere un \emph{first} di \(A\), quindi scrivo \(\{d\}\);
    \item per calcolare i \emph{first} di \(B\) devo conoscere quelli di \(C\) e \(D\), e dunque parto da \(C\): in questo caso ottengo semplicemente che i \emph{first} sono \(\{e\}\) e possibilmente \(\{\varepsilon\}\), quindi per \(C\) scrivo \(\{e, \varepsilon\}\);
    \item in modo simile a \(C\) posso facilmente ricavare che i \emph{first} di \(D\) sono \(\{f, \varepsilon\}\);
    \item tornando ora ad analizzare \(B\) posso dire che i suoi \emph{first} sono \(first(C) \setminus \{\varepsilon\}\), ma siccome \(C\) contiene \(\varepsilon\) tra i suoi \emph{first}, allora devo aggiungere \(first(D)\) a \(first(B)\);
    \item infine, notando che anche \(first(D)\) contiene \(\varepsilon\), posso annoverare in \(first(B)\) anche \(\varepsilon\) stesso.    
\end{enumerate}
Ora che abbiamo terminato la nostra ricerca dei \emph{first} possiamo ricavare la tabella risolutiva dell'esercizio, che si presenta come in Tab. \ref{first-follow-ex-2_step-1}.
\begin{table}[H]
	\centering
	\subimport{assets/tables/}{first-follow-ex-2_step-1.tex}
    \caption{Esercizio \ref{first-follow-ex-2} su first/follow, step 1}
    \label{first-follow-ex-2_step-1}
\end{table}
Passiamo ora gaiamente al prossimo step, ovvero il calcolo dei vari \emph{follow} seguendo l'algoritmo indicato in Alg.\ref{alg:follow}.

Per comodità notazionale, dato che nelle produzioni dell'esercizio compare il non-terminale \(A\), scriveremo \(X\) per indicare la \(A\) dell'algoritmo per il calcolo dei \emph{follow} (assumiamo quindi che ogni \(A\) nell'algoritmo venga sostituita da \(X\); quindi, ad esempio, \(\alpha A \beta\) diventa \(\alpha X \beta\)).
\begin{enumerate}
    \item la fase di inizializzazione del calcolo dei \emph{follow} vuole che lo start symbol abbia come \emph{follow} il terminatore di stringa , per le ragioni che abbiamo già visto, quindi poniamo \(follow(S) = \$\);
    \item cominciamo l'analisi da \(S \to aABb\):
    \begin{enumerate}
        \item osservando l'algoritmo, cerchiamo un match possibile per \(\alpha X \beta\); partiamo con il considerare \(X\) = \(A\) e di conseguenza \(\beta\) = \(Bb\);
        \item in questo caso abbiamo che \(\beta \neq \varepsilon\), quindi dobbiamo aggiungere i \(first(\beta)\) ai \(follow(A)\);
        \item i \emph{first} di \(\beta\) in questo caso sono i \emph{first} di \(Bb\), quindi \(\{e, f, b\}\) (notare che non sono \(first(B)\) ma \(first(Bb)\)); come indicato nella procedura, li aggiungo a \(follow(A)\);
        \item \(\beta\) è diverso da \(\varepsilon\) ed il suo first non contiene \(\varepsilon\), quindi il secondo \texttt{if} non si applica e terminiamo questo branch;
        \item ora dobbiamo analizzare il caso in cui, studiando la produzione di \(S\), assumiamo \(X = B\) e quindi \(\beta = b\);
        \item in questo caso ho che devo aggiungere a \(follow(B)\) i \(first(b)\), ovvero \(\{b\}\);
        \item \(\beta\) è diverso da \(\varepsilon\) ed inoltre \(\varepsilon \notin first(\beta)\), quindi termino anche questo branch.
    \end{enumerate}
    \item Ho concluso quindi l’analisi delle produzioni di \(S\) e passo quindi ad analizzare la produzione di \(A\):
    \begin{enumerate}
        \item parto da \(A \to Ac\): in questo caso non ho altra scelta che considerare \(X = A\), quindi \(\beta = c\): ricadendo nel primo \texttt{if} dell'algoritmo, aggiungo \(first(c) \setminus \{\varepsilon\}\) (ovvero \(\{c\}\)) ai \(follow(A)\);
        \item chiudo quindi il branch dato che il secondo \texttt{if} non si applica;
        \item considero ora la seconda produzione di \(A\), ovvero \(A \to d\); tuttavia, questa non è nella forma \(\alpha X \beta\) e quindi non ci dà informazioni, per cui posso passare oltre; ho terminato le produzioni di \(A\).
    \end{enumerate}
    \item Passiamo all'analisi della produzione \(B \to CD\):
    \begin{enumerate}
        \item considero \(X = C\) e ottengo quindi \(\beta = D\); dato che \(\beta \neq \varepsilon\) aggiungo ai \(follow(C)\) i \(first(D) \setminus \{\varepsilon\} = \{f\}\);
        \item però, dal momento che \(\varepsilon \in first(\beta)\), entro anche nel secondo \texttt{if} e aggiungo i \(follow(B)\) ai \(follow(C)\); visto che non posso eseguire questa operazione fino a quando non avrò esaminato tutte le altre produzioni, lascio un placeholder nella tabella. Avendo raggiunto la fine del ciclo \texttt{while} chiudo qui il branch;
        \item ora considero la seconda opzione, ovvero \(X = D\), quindi \(\beta = \varepsilon\);
        \item in questo caso andiamo direttamente nel secondo \texttt{if} che ci dice di aggiungere i \(follow(B)\) ai \(follow(D)\); nella tabella, quindi, indico un altro placeholder;
        \item abbiamo terminato le possibili interpretazioni delle produzioni per la derivazione \(B \to CD\).
    \end{enumerate}
    \item Le produzioni di \(C\), come quelle di \(D\), non sono del tipo \(\alpha X \beta\), dato che non contengono caratteri non-terminali; queste produzioni non ci danno informazioni sui \emph{follow} e l’algoritmo ci dice pertanto di ignorarle;
    \item a questo punto abbiamo terminato le produzioni da analizzare, non ci resta altro che risolvere i placeholder che abbiamo lasciato nella tabella durante i passi precedenti.
\end{enumerate}
Osserviamo in Tab.\ref{first-follow-ex-2_step-2} come risulta la tabella prima della sostituzione dei placeholder
\begin{table}[H]
	\centering
	\subimport{assets/tables/}{first-follow-ex-2_step-2.tex}
    \caption{Esercizio \ref{first-follow-ex-2} su first/follow con i placeholder}
    \label{first-follow-ex-2_step-2}
\end{table}
Mentre in Tab.\ref{first-follow-ex-2_step-3} si può osservare il risultato finale, con sostituzione dei placeholder.
\begin{table}[H]
	\centering
	\subimport{assets/tables/}{first-follow-ex-2_step-3.tex}
    \caption{Esercizio \ref{first-follow-ex-2} su first/follow una volta sostituiti i placeholder}
    \label{first-follow-ex-2_step-3}
\end{table}

\subsubsection{Esercizio first/follow 3}
\label{first-follow-ex-3}
Passiamo ora ad un'altro esercizio dello stesso tipo, sviluppato a partire dalla seguente grammatica:
\begin{align*}
    \G: S &\to aA \mid bBc \\
    A &\to Bd \mid Cc \\
    B &\to e \mid \varepsilon \\
    C &\to f \mid \varepsilon
\end{align*}
Questa volta saremo più spigliati con la risoluzione dei \emph{first}, ma li scriveremo comunque, dato che dobbiamo tener ben presente questa massima:
\begin{displayquote}
    "I \emph{first} ed i \emph{follow} li dovete sapere bene perché noi, con questi, ci faremo gli spaghetti."
    ~Paola Quaglia
\end{displayquote}

\noindent Andiamo quindi, senza ulteriori indugi, a risolvere i \emph{first} dei vari non-terminali.

\begin{enumerate}
    \item Per \(S\) inseriamo \(\{a, b\}\) e terminiamo subito, dato che entrambi sono terminali;
    \item per \(A\) dobbiamo conoscere i \emph{first} di \(B\) e \(C\);
    \item per \(B\) abbiamo che i \emph{first} sono \(\{e, \varepsilon\}\);
    \item per \(C\) abbiamo che i \emph{first} sono \(\{f, \varepsilon\}\);
    \item infine torniamo a risolvere \(A\):
    \begin{itemize}
        \item analizziamo \(A \to Bd\): inseriamo \(first(B) \setminus \{\varepsilon\}\), ovvero inseriamo \(\{e\}\);
        \item siccome \(\varepsilon \in first(B)\), aggiungiamo anche i \emph{first} di \(d\), che sono proprio \(\{d\}\);
        \item analizziam poi \(A \to Cc\);
        \item esattamente come per \(A \to Bd\), abbiamo che i \emph{first} sono \(\{f, c\}\);
    \end{itemize}
\end{enumerate}
Una volta inseriti i first nella tabella ci troveremo nella situazione rappresentata in Tab.\ref{first-follow-ex-3_step-1}.
\begin{table}[H]
	\centering
	\subimport{assets/tables/}{first-follow-ex-3_step-1.tex}
    \caption{Esercizio \ref{first-follow-ex-3} su first/follow, step 1}
    \label{first-follow-ex-3_step-1}
\end{table}
Ora è il momento di tuffarci a capofitto nel calcolo dei \emph{follow}; anche in questo caso tenteremo di matentere la spiegazione concisa ma completa e, per comodità notazionale, dato che nelle produzioni dell'esercizio compare il non-terminale \(A\), anche in quest'esercizio scriveremo \(X\) per indicare la \(A\) dell'algoritmo per il calcolo del \emph{follow} (come già fatto per il caso precedente). 

Prima di affogare nei calcoli, è interessante osservare più da vicino qual è l’idea sottesa all'inserimento di \(\$\) in follow(\(S\)) come primo passo.

Questa azione risulta intuitiva se si pensa che \(follow(Z)\) rappresenta quello che io mi aspetto di poter trovare dopo quello che derivo da un certo simbolo non-terminale \(Z\), con un certo numero di passi; quindi si capisce che, essendo \(S\) l’origine di tutte le parole generate da una certa grammatica, mi aspetto che, una volta analizzato tutto ciò che è generato da \(S\), troverò il terminatore di stringa, ovvero proprio \(\$\).

\noindent Bene, ora siamo pronti a fare a pugni coi calcoli.

\begin{enumerate}
    \item Partiamo con l’analizzare le produzioni di \(S\), iniziando con \(S \to aA\);
    \begin{enumerate}
        \item possiamo considerare solo \(X = A\), quindi per forza di cose avremo che \(\beta = \varepsilon\);
        \item passiamo subito al secondo \texttt{if}, che ci dice di aggiungere i \(follow(S)\) ai \(follow(A)\); lasciamo quindi un placeholder nella tabella e proseguiamo.
    \end{enumerate}
    \item Analizziamo ora \(S \to bBc\);
    \begin{enumerate}
        \item in questo caso siamo obbligati a scegliere \(X = B\) e quindi avremo che \(\beta = c\);
        \item passiamo dentro al primo \texttt{if} e aggiungiamo \(first(c) = \{c\}\) a \(follow(B)\);
        \item il secondo \texttt{if} non si applica, quindi terminiamo.
    \end{enumerate}
    \item Analizziamo ora \(A \to Bd\);
    \begin{enumerate}
        \item dobbiamo scegliere \(X = B\), quindi abbiamo che \(\beta = d\);
        \item anche questa volta soddisfiamo la condizione del primo \texttt{if} ed aggiungiamo \(first(d)= \{d\}\) a \(follow(B)\);
        \item come nel caso precedente, il secondo \texttt{if} non si applica: terminiamo il branch.
    \end{enumerate}
    \item Analizziamo \(A \to Cc\);
    \begin{enumerate}
        \item in questo caso \(X = C\), quindi troviamo che \(\beta = c\);
        \item come per il caso appena visto, aggiungiamo \(first(c)=\{c\}\) a \(follow(C)\) e chiudiamo il branch.
    \end{enumerate}
    \item Infine, tutte le rimanenti produzioni non sono nella forma \(\alpha X \beta\), quindi possiamo ragionevolmente dire di aver terminato l’analisi;
    \item l'ultimo passo che rimane da svolgere è la sostituzione dei placeholder.
\end{enumerate}
Osserviamo quindi in Tab.\ref{first-follow-ex-3_step-2}, come risulta la tabella dell' Es.\ref{first-follow-ex-3} prima della sostituzione dei placeholder.
\begin{table}[H]
	\centering
	\subimport{assets/tables/}{first-follow-ex-3_step-2.tex}
    \caption{Esercizio \ref{first-follow-ex-3} su first/follow con i placeholder}
    \label{first-follow-ex-3_step-2}
\end{table}
Quindi, in Tab.\ref{first-follow-ex-3_step-3}, ammiriamo la soluzione finale dell' Es.\ref{first-follow-ex-3}, con sostituzione dei placeholder.
\begin{table}[H]
	\centering
	\subimport{assets/tables/}{first-follow-ex-3_step-3.tex}
    \caption{Esercizio \ref{first-follow-ex-3} su first/follow una volta sostituiti i placeholder}
    \label{first-follow-ex-3_step-3}
\end{table}

\section{Costruzione una tabella di parsing top-down}
Ora che ci siamo esercitati con i meccanismi necessari, possiamo tornare ad occuparci della nostra preoccupazione principale, ovvero come costruire una tabella di parsing.

\subsection{Algoritmo per la costruzione di una parsing table}
Rappresentato in Alg. \ref{alg:parsing-table-comp} si può osservare l'algoritmo che, data in input una grammatica \(\G\), ritorna la tabella di aprsing top down propria di quella grammatica. 

\subimport{assets/pseudocode/}{parsing-table-comp.tex}
Ricordiamo brevemente che le tabelle di parsing top-down ci servono per verificare se, data una certa parola, questa può essere derivata tramite derivazione leftmost da una certa grammatica.

L'algoritmo prevede di scorrere tutte le produzioni \(A \to \alpha\) presenti nella grammatica e, per ognuna di queste:
\begin{enumerate}
    \item aggiungere \(A \to \alpha\) in \(M[A, b]\) per ogni \(b \in (first(\alpha)\setminus \{\varepsilon\})\);
    \item se  \(\varepsilon \in first(\alpha)\), aggiungere \(A \to \alpha\) a \(M[A, x]\) per tutti gli \(x \in follow(A)\). 
\end{enumerate}
Una volta terminato questo ciclo, si va a settare il valore \(error()\) in tutte le entry della tabella ancora vuote. Alcune note interessanti:
\begin{itemize}
    \item si osservi che \(follow(A)\) può contenere il simbolo \(\$\), ed è questo il motivo per cui nel punto due usiamo \(x\) invece che \(b\); infatti, mentre \(b\) indica terminali, \(x\) serve proprio a far notare che potrebbe esserci \(\$\), che non è un terminale della grammatica, ma il carattere segnalatore della terminazione di una parola;
    \item a livello grafico caselle di errore saranno lasciate vuote, mentre nell'implementazione vera e propria queste vengono fatte puntare tutte a una routine di errore;
    \item è possibile finire con l'avere con due entry in una stessa cella; tuttavia, le grammatiche che compongono tabelle con questa forma non sono deterministiche e non appartengono alla classe \(LL(1)\), di cui ci stiamo attualmente occupando;
    \item le caselle con entry multiple nella tabella di parsing si dicono \emph{entry multiple-defined}.
\end{itemize}

\subsection{Applicazioni}
Viene proposta ora come esempio al lettore questa grammatica.
\begin{align}
    \label{non-ll1_grammar}
    \G: E &\to E+T \mid T \\
    T &\to T*F \mid F \nonumber \\ \notag
    F &\to (E) \mid id \nonumber  \notag
\end{align}
Questa grammatica \emph{non} appartiene alla classe \(LL(1)\); ciò si può dimostrare creandone la parsing table per tale grammatica.
% Saprebbe dire il lettore se la grammatica qui presentata appartiene alla classe LL(1)?

% Mentre il lettore riflette su questo quesito può leggere un'altra importante citazione del nostro punto di riferimento, Paola Quaglia, la quale riflette sul senso di incomunicabilità che attraversa i tempi moderni, quel conflitto generazionale di cui tutti, in certo momento della nostra vita e in un certo schieramento, siamo stati attori.
% \begin{displayquote}
%     "NWY5YTliMmVjMmRhNiwyOS8xMC8yMDIwIDExOjUxLGthbHQxMA==  kaltura ci parla cosi' :-("
% \end{displayquote}
% Torniamo a noi: la risposta alla domanda posta in precedenza è che la grammatica in \ref{non-ll1_grammar} non appartiene alla classe \(LL(1)\); ciò si può dimostrare creandone la parsing table per tale grammatica.

Si può notare di fatto come \(first(E) = first(T) = first(F) = \{(, id\}\).
Di conseguenza, se seguiamo i passi dell'Alg.\ref{alg:parsing-table-comp} per la creazione di una parsing table, otteniamo una situazione in cui la casella \(M[E, id]\) contiene sia \(E \to E+T\) che \(E \to T\): ci troviamo nel caso di una \emph{entry multiply-defined}. Ma su questa grammatica abbiamo anche altre cose da dire.

\section{Grammatiche con ricorsione sinistra}
La grammatica vista sopra (Eq.\ref{non-ll1_grammar}) presenta anche una proprietà (o difetto) molto interessante, chiamata ricorsione sinistra:
\begin{definition}
    Una grammatica presenta ricorsione sinistra (left recursive grammar) se, per qualche \(A\) e \(\alpha\), \(A \Rightarrow^* A\alpha\)
\end{definition}

Ad esempio, consideriamo la seguente grammatica:
\begin{align*}
    S &\to B \mid a \\
    B &\to Sa \mid b
\end{align*}
Questa grammatica è ricorsiva a sinistra, perché possiamo ottenere una derivazione del tipo: \(S \Rightarrow B \Rightarrow Sa\).

\subsection{Grammatiche con ricorsione sinistra immediata}
Se osserviamo la grammatica \ref{non-ll1_grammar}, ci rendiamo conto immediatamente che presenta ricorsione sinistra, dato che possiamo espandere un numero indefinito di volte la produzione \(E \to E+T\) ed ottenere una sequenza di derivazioni come la seguente:
\begin{equation*}
    E \Rightarrow E+T \Rightarrow E+T+T \Rightarrow E+T+T+T \Rightarrow \dots
\end{equation*}
Ma non finisce qui: infatti, questa grammatica presenta, per la precisione, la caratteristica di ricorsione immediata sinistra (\emph{immediately left recursive grammar}), ovvero presenta una produzione in forma \(A \to A\alpha\).

Presentiamo ora un lemma sulle grammatiche con ricorsione sinistra.
\begin{lemma}\label{ll1-leftrec}
    Se una grammatica \(\G\) presenta ricorsione sinistra, immediata o no che sia, allora la tal grammatica \(\G\) \underline{non} appartiene alla classe LL(1).
\end{lemma}
Una volta venuti a conoscenza di questo lemma, è lecito chiedersi se tali grammatiche non siano riducibili a grammatiche di tipo \(LL(1)\) per fare in modo da poterle analizzare in maniera deterministica e di conseguenza più efficiente.

\subsection{Eliminazione della ricorsione sinistra immediata}
In molti casi è effettivamente possibile eliminare la ricorsione sinistra. 
Proviamo a capire l'intuizione da seguire per ottenere una grammatica \(LL(1)\) da una che presenta ricorsione sinistra; per farlo, aiutiamoci tentando di risolvere tale problema per la seguente grammatica d'esempio:
\begin{equation}
    \label{left-recursive_grammar}
    A \to A \alpha \mid \beta \textrm{  con  } \alpha \neq \varepsilon \land \beta \neq A \gamma
\end{equation}
Di fatto, se valesse \(\alpha = \varepsilon\), allora la grammatica sarebbe solo scritta male e non sarebbe quindi left recursive. Il nostro obiettivo, ricordiamolo, è quello di ottenere una grammatica che ci consente di avere lo stesso linguaggio della precedente, ma che tuttavia non presenta ricorsione sinistra. 

\subsubsection{Intuizione}
Per aiutarci a trovare un'idea, possiamo tracciare un albero piuttosto grezzo che rappresenti, più o meno, quello che sta succedendo se cerchiamo di derivare lo schema Eq.\ref{left-recursive_grammar}:
\begin{figure}[H]
    \centering
    \subimport{assets/figures/}{LR_elimination_problem.tex}
    \caption{Rappresentazione ad albero del problema}
    \label{lrremove-intuition_1}
\end{figure}
Questa figura non è un vero e proprio albero di derivazione, ma ci permette di capire una cosa: quello che noi vogliamo fare è modificare la struttura dell'albero, quindi le produzioni della grammatica, e al contempo mantenere inalterata la frontiera dell'albero. Potremmo ottenere questo risultato aggiungendo un nuovo non-terminale alla grammatica, in modo da eliminare l'elemento ricorsivo della produzione; idealmente, vorremmo ottenere un risultato di questo tipo:
\begin{figure}[H]
    \centering
    \subimport{assets/figures/}{LR_elimination_intuition.tex}
    \caption{Rappresentazione della nostra intuizione}
    \label{lrremove-intuition_2}
\end{figure}

\subsubsection{Soluzione formale}
A questo punto possiamo scrivere formalmente la nostra strategia per l'eliminazione della ricorsione sinistra immediata: data una produzione del tipo:
\begin{equation*}
    A \to A \alpha \mid \beta
\end{equation*}
dove \(\alpha \ne \varepsilon\) e \(\beta \ne A \gamma\), possiamo riscriverla equivalentemente come segue:
\begin{align*}
    A &\to \beta A' \\
    A' &\to \alpha A' \mid \varepsilon
\end{align*}
dove \(A'\) è un non-terminale \emph{fresh} per la grammatica di riferimento, questo perché altrimenti potrebbe interferire con la derivazione di alcune parole nel linguaggio generato dalla stessa grammatica.

\subsubsection{Formulazione generale}
\label{subsec:left-recursive-to-ll1}
Riformuliamo la precedente strategia, in modo da astrarla al caso più generale possibile rispetto ai body di una certa produzione.

Per eliminare la ricorsione sinistra immediata di una produzione e mantenerne inalterate le eventuali derivazioni, devo considerare la sua forma:
\begin{equation}
    A \to A \alpha_1 \mid \dots \mid A \alpha_n \mid \beta_1 \mid \dots \mid \beta_k
\end{equation} 
dove \(\alpha_j \ne \varepsilon \;\; \forall j: 1 \le j \le n\) e anche \(\beta_i \ne A \gamma_i \;\; \forall i: 1 \le i \le k\), con la seguente forma:

\begin{align}
    A &\to \beta_1 A' \mid \dots \mid \beta_k A' \\
    A' &\to \alpha_1 A' \mid \dots \mid \alpha_n A' \mid \varepsilon \notag
\end{align}
dove \(A' \notin \A \setminus T\), cioè è un non-terminale \emph{fresh}.

Andiamo adesso a vedere come (e se) è possibile eliminare la ricorsione sinistra anche quando non è immediata.

\subsection{Eliminazione di una ricorsione sinistra qualsiasi}
Qui il problema si fa più spinoso: possiamo già anticipare che non sarà un'operazione possibile in tutte le grammatiche e per cui c'è addirittura il rischio che, l'eliminazione della ricorsione sinistra per un certo non-terminale, causi la sua comparsa per un altro non-terminale.

In ogni caso, l'idea è di ridurre gli steps della derivazione \(A \Rightarrow^* A \alpha\), in modo da ottenere una produzione che presenta ricorsione sinistra immediata ed applicare la tecnica vista in \ref{subsec:left-recursive-to-ll1}.

Consideriamo la seguente grammatica:
\begin{align*}
    A &\to Ba \mid b \\
    B &\to Bc \mid Ad \mid b
\end{align*}
Notiamo subito che abbiamo una ricorsione sinistra immediata su \(B\),  ma abbiamo anche una ricorsione sinistra su \(A\), attraverso \(A \Rightarrow Ba \Rightarrow Ada\).

Ci eravamo detti di tentare di ridurre gli steps di derivazione; per cui, quello che facciamo adesso è sostituire i non-terminali coi body delle loro produzioni, ad esempio:
\begin{equation*}
    B \to Ad \textrm{ diventerà } B \to Bad \mid bd
\end{equation*}
Mantenendo inalterate le altre produzioni, ottengo la seguente grammatica:
\begin{align*}
    A &\to Ba \mid b \\
    B &\to Bc \mid Bad \mid bd \mid b
\end{align*}
A questo punto diventa molto semplice eliminare la ricorsione immediata di \(B\) con lo stesso metodo visto in precedenza:
\begin{align*}
    A &\to Ba \mid b \\
    B &\to bdB' \mid bB' \\
    B' &\to cB' \mid adB' \mid \varepsilon
\end{align*}

\subsection{Eliminazione di ricorsione sinistra: efficacia}
Fermiamoci un attimo e facciamo il punto della situazione. Prima di questa ampia digressione sulle grammatiche con ricorsione sinistra stavamo parlando di top-down parsing, giusto? Bramavamo una tabella di parsing senza entries con definizioni multiple, perché questo avrebbe reso il nostro parsing deterministico. Sapevamo che, fintantoché andavamo a considerare delle grammatiche \(LL(1)\), avremmo potuto dormire sogni tranquilli; ma il lemma \ref{ll1-leftrec} ci ha detto che le grammatiche con ricorsioni sinistra (sia immediata, sia non immediata) non sono \(LL(1)\), per cui abbiamo iniziato a chiederci se fosse possibile eliminare in qualche modo questa proprietà. Ebbene, dopo tanto tempo speso nell'impresa, ci chiediamo: dopo aver eliminato tutte le ricorsioni, abbiamo certezza che la grammatica ottenuta sia \(LL(1)\)?

Consideriamo la grammatica delle espressioni aritmetiche, visto che siamo partiti da quella; proviamo ad applicare le nostre regole di eliminazione e vediamo cosa succede.
\begin{align*}
    \G: E &\to E+T \mid T \\
    T &\to T*F \mid F \nonumber \\
    F &\to (E) \mid id \nonumber 
\end{align*}
% TODO mostrare gli steps
E il risultato finale è proprio lei, la grammatica che abbiamo visto a \ref{first-follow-ex-1} nel calcolo dei \emph{first/follow}!% Contenti? Fossi in voi, me la tatuerei da qualche parte per evitare che, dopo averla trovata all'esame senza saper rispondere, infesti i vostri sogni per il resto della vostra vita.
\begin{align*}
    \mathcal{G'}: E &\rightarrow TE' \\
    E' &\rightarrow +TE' \mid \varepsilon \\
    T &\rightarrow FT' \\
    T' &\rightarrow *FT' \mid \varepsilon \\
    F &\rightarrow (E) \mid id
\end{align*}
Per di più sappiamo questa grammatica è \(LL(1)\), possiamo dunque ritenerci soddisfatti? L'algoritmo di eliminazione della ricorsione sinistra garantisce effettivamente di ottenere grammatiche \(LL(1)\)?

Per evitare di cadere in ragionamenti induttivi fallaci del tipo 
\begin{equation*}
    \textrm{uh, in questo caso funziona} \implies \textrm{funzionerà in qualsiasi caso, no?}
\end{equation*}
andiamo a vedere un altro caso, considerando una grammatica che genera comunque il linguaggio delle espressioni aritmetiche ma, a differenza della precedente, è ambigua:
\begin{equation*}
    \G: E \to E + E \mid E * E \mid (E) \mid id
\end{equation*}
Eseguendo la nostra procedura otteniamo la seguente grammatica:
\begin{align*}
    \mathcal{G'}: E &\to (E)E' \mid idE' \\
    E' &\to +EE' \mid \ast EE' \mid \varepsilon
\end{align*}
Questa grammatica è \(LL(1)\)? Sfortunatamente, se costruiamo la tabella di parsing (Tab.\ref{tab:ll-rem-efficiency-table}) per questa grammatica, troviamo che le celle [\(E', +\)] e [\(E', \ast\)] contengono due produzioni:
\begin{itemize}[noitemsep]
    \item \(E' \to +EE'\) (o \(E' \to *EE'\));
    \item \(E' \to \varepsilon\).
\end{itemize}
\begin{table}
    \centering 
    \subimport{assets/tables/}{ll-rem-efficiency-table.tex}
    \caption{Tabella di parsing LL(1) per \(\G'\)}
    \label{tab:ll-rem-efficiency-table}
\end{table}
Ne abbiamo un ulteriore conferma se andiamo a calcolare i first/follow per \(E\) e \(E'\):
\begin{table}[H]
    \centering
    \subimport{assets/tables/}{ll-rem-efficiency-ff.tex}
    \caption{Tabella con i first/follow per \(E\) e \(E'\)}
    \label{ll-rem-efficiency-ff}
\end{table}
% \begin{figure}[H]
%     \centering
%     \includegraphics[width=.4\textwidth,keepaspectratio]{fxness-lrremove-fftable.png}
%     \caption{Tabella con i first/follow per \(E\) e \(E'\)}
%     \label{fxness-lrremove- fftable}
% \end{figure}
\paragraph{Conclusione}
Nota bene: non dobbiamo cadere nell'errore di supporre che, magari, la nostra procedura funzioni a meno che non sia lanciata su grammatiche ambigue, perché non ne abbiamo fornito alcuna prova. Da questo  esempio possiamo solamente concludere, a malincuore, che la procedura di eliminazione della ricorsione sinistra \emph{non garantisce} di ottenere delle grammatiche \(LL(1)\).

\subsection{Eliminare la ricorsione sinistra elimina l'ambiguità?}
Possiamo però chiederci: la grammatica \(\mathcal{G'}\) che abbiamo ottenuto è ancora ambigua, oppure applicando la nostra procedura ne abbiamo eliminato l'ambiguità? Ricordiamo che l'ambiguità della grammatica di partenza risiedeva nell'impossibilità di esprimere l'associatività degli operatori e la precedenza dell'uno sull'altro \footnote{Si riveda a \ref{sec:ambiguity} in caso servisse un ripasso.}. 

Sfortunatamente, riusciamo a trovare senza troppi problemi due derivazioni rightmost che ci portano a ottenere la medesima stringa \(id + id * id\):
\begin{align*}
    E &\Rightarrow idE' \Rightarrow id + EE' \Rightarrow id + E * EE' \Rightarrow id + E * E \\
        &\Rightarrow id + E * idE' \Rightarrow id + E * id \Rightarrow id + idE' * id \\
        &\Rightarrow id + id * id 
    \\ \\
    E &\Rightarrow idE' \Rightarrow id + EE' \Rightarrow id + E \Rightarrow id + idE' \\
        &\Rightarrow id + id * EE' \Rightarrow id + id * E \Rightarrow id + id * idE' \\
        &\Rightarrow id + id * id  
\end{align*}

Possiamo convincerci di questo risultato anche seguendo un altro procedimento, ossia tracciando un albero di derivazione parziale per la nostra grammatica:
\begin{figure}[H]
    \centering
    \subimport{assets/figures/}{LR_elimination_partial_derivation.tex}
    \caption{Albero di derivazione parziale per \(\mathcal{G'}\)}
    \label{fxness-lrremove- ambiguity_1}
\end{figure}
A questo punto notiamo subito che, per ottenere la parola \(id + id * id\), possiamo completare questo albero parziale sostituendo i nodi marcati con i seguenti sottoalberi:
\begin{figure}[H]
    \begin{minipage}[b]{0.4\textwidth}
        \centering
        \subimport{assets/figures/}{LR_elimination_partial_derivation_v1_part2.tex}
        \subcaption{}
        \label{fxness-lrremove- ambiguity_2_2}
    \end{minipage}
    \hfill
    \begin{minipage}[b]{0.4\textwidth}
        \centering
        \subimport{assets/figures/}{LR_elimination_partial_derivation_v2_part1.tex}
        \subcaption{}
        \label{fxness-lrremove- ambiguity_2_1}
    \end{minipage}
    \caption{Completamento di \ref{fxness-lrremove- ambiguity_1}, versione 1}
    \label{fxness-lrremove- ambiguity_2}
\end{figure}
Ma, allo stesso identico modo, potremmo completare lo stesso albero parziale e ottenere la stessa identica parola sostituendo gli stessi nodi con questi due diversi sottoalberi:
\begin{figure}[H]
    \begin{minipage}[b]{0.4\textwidth}
        \centering
        \subimport{assets/figures/}{LR_elimination_partial_derivation_v1_part1.tex}
        \subcaption{}
        \label{fxness-lrremove- ambiguity_3_1}
    \end{minipage}
    \hfill
    \begin{minipage}[b]{0.4\textwidth}
        \centering
        \subimport{assets/figures/}{LR_elimination_partial_derivation_v2_part2.tex}
        \subcaption{}
        \label{fxness-lrremove- ambiguity_3_2}
    \end{minipage}
    \caption{Completamento di \ref{fxness-lrremove- ambiguity_1}, versione 2}
    \label{fxness-lrremove- ambiguity_3}
\end{figure}

\paragraph{Conclusione}
Anche qui, non possiamo fare altro che concludere che applicare la procedura di eliminazione della ricorsione sinistra su una grammatica ambigua \emph{non} ne elimina l'ambiguità.

\section{Fattorizzabilità Sinistra}
Consideriamo ora la grammatica \(\G: S \rightarrow aSb \mid ab\), che ormai ben conosciamo: questa grammatica ci permette di denotare, ad esempio, un linguaggio composto da parentesi annidate. Quello che non sapevamo è che non è una grammatica \(LL(1)\): ce ne possiamo accorgere calcolandone i \(first(S)\). Dal momento che l'unico elemento appartenente a questo insieme, nel caso delle due produzioni, risulta infatti essere il non-terminale \(a\), questo vuol dire che, applicando l'algoritmo per la generazione della tabella di parsing predittivo, ci troveremo con due produzioni nella posizione \(M[S, a]\).

Il punto è che \(\G\) può essere \textbf{fattorizzata a sinistra}; in generale, una grammatica è fattorizzabile a sinistra quando:
\begin{itemize}
    \item ci sono almeno due produzioni che hanno lo stesso non-terminale come driver;
    \item possiedono un prefisso comune nel body.
\end{itemize}
Nel caso precedente, ad esempio, vi sono due produzioni che hanno \(S\) come driver e \(a\) come prefisso. Rispetto a tutto ciò, analogamente alle grammatiche con ricorsione a sinistra, abbiamo un lemma.

\begin{lemma}
    Se la grammatica \(\G\) può essere fattorizzata a sinistra, allora sicuramente \(\G\) non è \(LL(1)\).    
\end{lemma}

\subsection{Strategia}
Non possiamo quindi fare a meno di chiederci, di nuovo, se possiamo modificare le produzioni della grammatica in modo da non avere più produzioni per un singolo prefisso, ed evitare quindi di trovarci una tabella di parsing con entries multiple defined.

L'idea che sta alla base della strategia che si utilizza per la fattorizzazione sinistra è quella di rimandare il più possibile la scelta delle produzioni con lo stesso prefisso. Quindi, data una grammatica \(\G\) che ha due produzioni con lo stesso driver e con un prefisso comune, quello che si fa è rimpiazzare la produzione iniziale:
\begin{equation*}
    A \rightarrow \alpha \beta_1 \mid \alpha \beta_2  
\end{equation*}
andando a sostituirla con due produzioni di questo tipo:
\begin{align*}
    A &\rightarrow \alpha A' \\
    A' &\rightarrow \beta_1 \mid \beta_2        
\end{align*}
dove \(A' \notin \A \setminus T\), cioè è un non-terminale \emph{fresh}.

Nel caso della prima grammatica non vi può essere determinismo nell'espansione data da \(A\), perché sceglieremo l'espansione indipendentemente dalla lettera successiva a quella considerata correntemente; ad esempio, se potessi sapere che il prossimo simbolo sarà una \(b\), allora non avrei dubbi su cosa scegliere.

\subsection{Algoritmo di fattorizzazione a sinistra}
Il tipo di trasformazione che possiamo fare ad una grammatica generica in modo da rimuovere questi prefissi comuni mantenendo il linguaggio è espresso dal seguente algoritmo: 

\subimport{assets/pseudocode/}{left-fact.tex}

L'idea è che, data in input una grammatica che può essere fattorizzata a sinistra, per ogni non-terminale della grammatica
\begin{enumerate}
	\item si trova il più lungo prefisso (\(\alpha\)) comune a due o più produzioni aventi lo stesso driver
	\item se tale prefisso \(\alpha\) esiste, allora
	\begin{enumerate}
		\item viene scelto un non-terminale \(A' : A' \notin \A \setminus T\)
		\item e si sostituiscono le produzioni di \(A\) nella forma
		\begin{equation*}
    			A \rightarrow \alpha \beta_1 \mid ... \mid \alpha \beta_n \mid \gamma_1 \mid \ldots \mid \gamma_n
		\end{equation*}
		con le seguenti produzioni: 
		\begin{align*}
    			A &\rightarrow \alpha A' \mid \gamma_1 \mid \ldots \mid \gamma_n \\
    			A' &\rightarrow \beta_1 \mid \ldots \mid \beta_n
		\end{align*}
	\end{enumerate} 
	\item si ripete da 1 fino a quando non è più possibile trovare produzioni con un prefisso comune (\(\alpha\))
\end{enumerate}

L'ultimo punto ci fa intuire che non basta una semplice iterazione di tipo \texttt{foreach} per eliminare i prefissi comuni, ma è necessario esaminare tutte le produzioni, anche dopo che sono stati rimossi dei prefissi; questo perché, naturalmente, è possibile che si generino dei nuovi prefissi a seguito della rimozione dei primi, per cui è opportuno ripetere la procedura finché nessun nuovo prefisso viene trovato.

\paragraph{Considerazioni sull'efficacia}
Vediamo se tale algoritmo risolve il problema che avevamo precedentemente sollevato, ossia se questo impedisca la generazione di entries a definizione multipla nella tabella di parsing. Scopriamolo eseguendo la fattorizzazione sulla nostra fida \(\G: S \to aSb \mid ab\), da cui otteniamo:
\begin{align*}
    \G': S &\rightarrow aS' \\
    S' &\rightarrow Sb \mid b
\end{align*}

Ma quindi, questa \(\G'\) che abbiamo appena ottenuto è una grammatica \(LL(1)\)? Se andiamo a calcolarne i \emph{first/follow}, dovremmo trovare una qualcosa di simile a questo:
\begin{itemize}
    \item \(first(S) = \{a\}\) e \(follow(S) = \{\$, b\}\);
    \item \(first(S') = \{a, b\}\) e \(follow(S') = \{\$, b\}\).
\end{itemize}
Per cui la tabella di parsing viene costruita come segue:
\begin{table}[H]
    \centering
    \subimport{assets/tables/}{topDownParsingAfterFactorization.tex}
    \caption{Top Down Parsing Table - Dopo fattorizzazione sinistra}
    \label{topDownParsingAfterFactorization}
\end{table}

Essendo che non abbiamo entries multiply-defined non possiamo fare altro che affermare che la grammatica così ottenuta è \(LL(1)\); come sempre però un esempio non è sufficiente a dimostrare che questo valga in tutti i casi.

\subsubsection{Dangling Else}

Passiamo ora ad esaminare un caso di fattorizzazione sinistra molto famoso nei linguaggi di programmazione e che abbiamo già introdotto, ossia la grammatica ambigua degli \texttt{if} e degli \texttt{else}:
\begin{equation*}
    S \rightarrow \; \textrm{\texttt{if}} \; b \; \textrm{\texttt{then}} \; S \mid \textrm{\texttt{if}} \; b \; \textrm{\texttt{then}} \; S \; \textrm{\texttt{else}} \; S \mid c  
\end{equation*}
Questa grammatica, una volta fatta passare nell'algoritmo di fattorizzazione sinistra, diventa:
\begin{align*}
    S &\rightarrow \textrm{\texttt{if}} \; b \; \textrm{\texttt{then}} \; SS' \mid c \\
    S' &\rightarrow \textrm{\texttt{else}} \; S \mid \varepsilon
\end{align*}
Per il risultato ottenuto in precedenza tale grammatica dovrebbe ora essere \(LL(1)\), giusto? Calcoliamone i \emph{first/follow}: 
\begin{itemize}
    \item \(first(S) = \{\textrm{\texttt{if} } b \textrm{ \texttt{then}, } c\}\) e \(follow(S) = \{\$, \textrm{ \texttt{else}}\}\)
    \item \(first(S') = \{\textrm{\texttt{else}, } \varepsilon\}\) e \(follow(S') = \{\$, \textrm{ \texttt{else}}\}\)
\end{itemize}  
In questo caso la grammatica risultante non è \(LL(1)\), perché in \(M[S', \textrm{\texttt{else}}]\) ci sono due produzioni: si può infatti osservare che 
\begin{align*}
    M[S', \textrm{\texttt{else}}] &= S' \rightarrow \textrm{\texttt{else}} \; S, \; \textrm{in quanto} \; \textrm{\texttt{else}} \in \textrm{first}(\textrm{\texttt{else}} \; S') \\
    M[S', \textrm{\texttt{else}}] &= S' \to \varepsilon, \; \textrm{in quanto} \; \textrm{\texttt{else}} \in \textrm{follow}(S')
\end{align*}
Abbiamo anche un altro modo per constatare che la nostra grammatica rimane ambigua, ossia andando a vedere il seguente albero di derivazione:
\begin{figure}[H]
    \centering
    \subimport{assets/figures/}{left_factorization_ambiguity.tex}
    \caption{Albero di derivazione per una parola sulla grammatica degli \texttt{if-else}}
    \label{leftFactorizationAmbiguity}
\end{figure}

Si può infatti notare che

\begin{itemize}
	\item in un caso è possibile scegliere di sostituire \({S'}_1\) con \(\varepsilon\) e \({S'}_2\) con \texttt{else} \(c\);
	\item nell'altro invece è possibile sostituire \({S'}_1\) con \texttt{else} \(c\) e \({S'}_2\) con \(\varepsilon\);
\end{itemize}

In entrambi i casi si arrivare di fatto alla stessa stringa da due alberi di derivazione differenti (ottenuti impiegando due derivazioni leftmost).

Questo problema viene definito \emph{dangling else} e corrisponde a non sapere identificare in maniera certa a quale \texttt{then} appartenga un determinato \texttt{else}. Per poter risolvere tale problematica vi sono due differenti soluzioni:
\begin{itemize}
    \item proibire il costrutto \texttt{if-then} e sostituirlo con un \texttt{if-then-else}, tecnica che viene impiegata nel caso dei linguaggi funzionali: l'idea è che bisogna avere un ramo in cui il booleano \(b\) sia valutato a true e un altro nel caso in cui sia valutato a false; nel caso in cui l'\texttt{else} dovesse rivelarsi completamente inutile, allora si crea un ramo fittizio;
    \item imporre l'\textbf{innermost binding}, dove si fa corrispondere l'\texttt{else} al \texttt{then} più vicino e non ancora matchato: ovviamente tale soluzione ha senso solo nel caso in cui non si vogliono utilizzare parentesizzazioni.
\end{itemize}
L'innermost binding può anche essere implementato attraverso la definizioni di policy per definire particolari direttive al parser (approfondiremo il discorso quando parleremo di Bison, niente paura); oppure specializzando la grammatica, permettendo solo a coppie di \texttt{then-else} che matchano tra le occorrenze di \texttt{then} e \texttt{else}: in particolare, il secondo risultato può essere ottenuto utilizzando la grammatica seguente:
\begin{align*}
    S &\rightarrow M \mid U \\
    M &\rightarrow \texttt{if} \; b \; \texttt{then} \; M \; \texttt{else} \; M \mid c \\
    U &\rightarrow \texttt{if} \; b \; \texttt{then} \; S \mid \texttt{if} \; b \; \texttt{then} \; M \; \texttt{else} \; U
\end{align*}

\section{Riepilogo sulle grammatiche LL(1)}
A questo punto dovrebbe essere chiaro che, se una grammatica è
\begin{itemize}
    \item ricorsiva a sinistra \(\lor\)
    \item fattorizzabile a sinistra \(\lor\)
    \item ambigua,
\end{itemize}
allora non è possibile che sia una grammatica \(LL(1)\). Abbiamo quindi un lemma, che fa più o meno così:
\begin{lemma}
    \(\G\) è una grammatica \(LL(1)\) se e solo se, nel caso in cui \(\G\) avesse delle produzioni del tipo \(A \rightarrow \alpha \mid \beta\), allora:
    \begin{itemize}
        \item first(\(\alpha\)) \(\cap\) first(\(\beta\)) = \(\emptyset\);
        \item se \(\varepsilon \in\) first(\(\alpha\)), allora first(\(\beta\)) \(\cap\) follow(\(A\)) = \(\emptyset\) e, viceversa, se \(\varepsilon \in\) first(\(\beta\)), allora first(\(\alpha\)) \(\cap\) follow(\(A\)) = \(\emptyset\)
    \end{itemize}
\end{lemma}
Questo termina la discussione sulle grammatiche \(LL(1)\). Va però detto che tali proprietà sono estensibili alle grammatiche \(LL(K)\); ricordiamo infatti che l'acronimo \(LL\) si riferisce al fatto che la stringa viene letta da sinistra verso destra e che viene applicata un tipo di derivazione leftmost, mentre il numero 1 messo tra parentesi ci dice che il nostro algoritmo di parsing analizzerà un elemento di input alla volta. L'algoritmo si può dunque estendere se scegliamo di controllare un numero arbitrario \(K\) di elementi alla volta: nel complesso, la strategia è analoga; tuttavia, la tabella di parsing prevedrà delle entries in più. 

Ad esempio, nel caso di una grammatica \(LL(2)\) vanno segnalate sulle colonne coppie di simboli terminali; se prendiamo come riferimento la grammatica \(S \rightarrow aSb \mid ab\), avremo sulle colonne le coppie di terminali \(aa\), \(ab\), \(bb\), \(ba\), \(b\$\), \(a\$\), \(\$\$\). In questo caso, avendo la possibilità di vedere due simboli alla volta riuscirei a fare un parsing predittivo deterministico? Sì, perché so per certo che se leggessi \(ab\) dovrei utilizzare la produzione \(S \rightarrow ab\), mentre in tutti gli altri casi \(S \rightarrow aSb\).

\section{Esercizi riassuntivi sulle grammatiche LL(1)}
\subsection*{Esercizio 1}
Sia data la grammatica \(S \rightarrow aSb \mid \varepsilon\) che denota il linguaggio \(\{a^n b^n \mid n \geq 0\}\); questa grammatica è \(LL(1)\)?

Possiamo facilmente calcolare che first(\(S\)) = \(\{a, \varepsilon\}\), così come follow(\(S\)) = \(\{\$, b\}\). 

Per il lemma precedente visto che first(\(aSb\)) \(\cap\) first(\(\varepsilon\)) = \(\emptyset\) e che, nonostante \(\varepsilon \in\) first(\(\varepsilon\)), abbiamo che first(\(aSb\)) \(\cap\) follow(\(S\)) = \(\emptyset\); ciò può essere osservato anche dalla tabella di parsing predittivo:
\begin{table}[H]
    \centering
    \subimport{assets/tables/}{trainingLL1_1.tex}
    \caption{Es 1 - Training LL(1)}
    \label{trainingLL1_1}
\end{table}
Sorprendentemente dunque la grammatica è \(LL(1)\).

\subsection*{Esercizio 2}
Consideriamo adesso la seguente grammatica:
\begin{align*}
    S &\rightarrow AbB \mid B \\
    A &\rightarrow cB \mid a \\
    B &\rightarrow A
\end{align*}
La grammatica sopra citata è \(LL(1)\)? Il sospetto è di no, perché a prima impressione i first(\(A\)) sono uguali ai first(\(B\)) e, essendo che nelle produzioni di \(S\) compaiono in prima posizione entrambi i non-terminali discussi, è possibile che si vadano a collocare nella stessa cella. Procediamo per gradi calcolando i first e, se necessario, i follow:
\begin{table}[H]
    \centering
    \subimport{assets/tables/}{trainingLL1_first_2.tex}
    \caption{Es 2: Calcolo First - Training LL(1)}
    \label{trainingLL1_first_2}
\end{table}
Questa grammatica non è dunque \(LL(1)\), perché nella tabella di parsing avrò le produzioni \(S \rightarrow AbB\) e \(S \rightarrow B\) sia nella cella \(M[S, a]\) che in \(M[S, c]\).
\begin{table}[H]
    \centering
    \subimport{assets/tables/}{trainingLL1_2.tex}
    \caption{Es 2: Training LL(1)}
    \label{trainingLL1_2}
\end{table}

\subsection*{Esercizio 3}
Esaminiamo infine questa grammatica:
\begin{align*}
    S &\rightarrow Aa \\
    A &\rightarrow bB \mid c \\
    B &\rightarrow aA \mid \varepsilon
\end{align*}
\begin{table}[H]
    \centering
    \subimport{assets/tables/}{trainingLL1_first_3.tex}
    \caption{Es 3: Calcolo First \& follow - Training LL(1)}
    \label{trainingLL1_first_3}
\end{table}
Si può notare che mi ritroverò con produzioni in \(M[B, a]\), e ciò è osservabile anche per via del lemma enunciato precedentemente: infatti, si ha che \(\varepsilon \in\) first(\(\alpha\)) e first(\(aA\)) \(\cap\) follow(\(B\)) = \(\{a\} \neq \emptyset\).

la tabella risultante è la seguente, e possiamo dedurne che la grammatica non è \(LL(1)\).
\begin{table}[H]
    \centering
	\subimport{assets/tables/}{trainingLL1_3.tex}
    \caption{Es 3: Training LL(1)}
    \label{trainingLL1_3}
\end{table}

\end{document}
