\documentclass[class=book, crop=false, oneside, 12pt]{standalone}
\usepackage{standalone}
\usepackage{../../style}
\usepackage[normalem]{ulem}
\graphicspath{{./assets/images/}}

% arara: pdflatex: { synctex: yes, shell: yes }
% arara: latexmk: { clean: partial }
\begin{document}
\chapter[Parsing bottom-up: LR(1) e LALR(1)]{Analisi sintattica: parsing bottom-up LR(1) e LALR(1)}

\section{L'automa caratteristico LR(1)}
Riprendiamo il filo del discorso: nel precedente capitolo siamo andati a discutere tutti i passaggi che, partendo da una grammatica, ci conducono ad eseguire il parsing bottom-up. Consigliamo di tornare a dare un occhio allo schema presentato in \autoref{subsec:schema}. Ci eravamo lasciati con una questione alquanto spinosa: avevamo constatato che è possibile ritrovarsi con due reducing items in uno stesso stato anche se la grammatica di partenza non è ambigua, e questa situazione potrebbe essere risolta conservando, per ogni stato, delle informazioni rispetto a quali simboli potrebbero seguirgli in lettura. Ci eravamo resi conto di non aver modo di conservare e impiegare efficacemente queste informazioni con la tecnica di costruzione dell'automa SLR(1), il che lascia pensare solo a una cosa: è il momento di abbandonare questa tecnica e guardare a soluzioni più complesse, certo, ma senza dubbio più efficaci. È il momento di affrontare la costruzione di un automa caratteristico LR(1).
\subsection{Stati}
\paragraph{Gli LR(1)-items}
Per prima cosa dobbiamo ricordarci che la principale differenza sta proprio in cosa popola gli stati dei diversi automi: per gli automi SLR(1) utilizzavamo gli LR(0)-items, gli automi LR(1) invece utilizzano gli LR(1)-items.
\paragraph{Forma}
Gli LR(1)-items hanno questa forma:
\begin{equation}
    \label{lr0}
    [A \to \alpha \cdot B \beta, \Delta]
\end{equation}
Possiamo vedere subito che un LR(1)-item è una tupla di due elementi:
\begin{enumerate}
    \item il primo è un LR(0)-item \((A \to \alpha \cdot B \beta)\);
    \item il secondo è un insieme di caratteri \((\Delta)\) detto \textbf{lookahead set}.
\end{enumerate}

\subsection{Chiusura di un insieme di LR(1)-item}
Quando andiamo a calcolare la chiusura di un item di tipo LR(1) utilizziamo la \(closure_1(item)\) e non la \(closure_0(item)\), che funziona esattamente come la \(closure_0\) ma in più ci permette di aggiornare l'insieme \(\Delta\), il lookahead set. Questo insieme ci torna poi utile quando dobbiamo inserire mosse di riduzione in qualche stato dell'automa caratteristico LR(1): il lookahead set specifica quali caratteri ci dobbiamo aspettare di vedere in lettura quando stiamo per utilizzare una mossa di reduce.

\paragraph{Funzionamento}
Ma specifichiamo ora meglio come è strutturata l'operazione di \(closure_1(item)\):
\begin{itemize}
    \item la prima parte della chiusura è in tutto e per tutto uguale alla \(closure_0(item)\), quindi quando si calcola una \(closure_1(item)\) la prima cosa che si fa è appunto calcolare la \(closure_0(item)\), utilizzando la parte LR(0) dell'item LR(1) che stiamo chiudendo;
    \item la seconda parte prevede di aggiornare l'insieme di lookahead, che viene aggiunto alla \(closure_0(item)\) appena calcolata.  
\end{itemize}
Di fatto con questo metodo di calcolo l'informazione contenuta nell'insieme \(\Delta\) di un LR(1)-item viene tramandata agli item che ne derivano grazie alla seconda parte della procedura \(closure_1\), così se si segue una derivazione lungo tutto l'albero delle produzioni guardando a \(\Delta\) si ha sempre sotto controllo quali simboli ci si aspetta di leggere in un certo stato quando si deve effettuare una mossa di riduzione.

\paragraph{Calcolo}
Come facciamo a calcolare la chiusura degli insiemi di LR(1)-items? Il metodo di calcolo che ci viene presentato è un classico calcolo risolvibile grazie al teorema del punto fisso:
\begin{definition}
    \label{def:lr1-closure}
    Sia \(P\) un insieme di LR(1)-item, la \(closure_1(P)\) identifica il più piccolo insieme di item, con il più piccolo lookahead-set, che soddisfa la seguente equazione:
    \begin{align*}
        closure_1(P) = P \; \cup \; &\{[B \rightarrow \cdot \gamma, \Gamma] : [A \rightarrow \alpha \cdot B \beta, \Delta] \in closure_1(P) \; \land \\ 
        & B \rightarrow \gamma \in \P' \; \land \\
        & first(\beta \Delta) \subseteq \Gamma\}
    \end{align*}  
    dove \(first(\beta \Delta) = \cup_{d \in \Delta} first(\beta d)\) e ricordiamo che \(\P'\) è ricavato dall'insieme delle produzioni \(\P\) aggiungendo la produzione \(S' \to S\).
\end{definition}
\paragraph{Algoritmo}
Dal momento che la Def.\ref{def:lr1-closure} non risponde molto bene al richiamo degli attributi "semplice e intuitiva", potrebbe essere una buona idea dare un'occhiata alla sua formulazione algoritmica in Alg.\ref{alg:lr1-closure}. \\
% \begin{figure}[H]
%     \centering
%     \includegraphics[width=\textwidth,keepaspectratio]{alg_closure-lr1.png}
%     \caption{Algoritmo per il calcolo della chiusura di uno stato LR(1)}
%     \label{alg:closure-lr1}
% \end{figure}
\subimport{assets/pseudocode/}{lr1-closure.tex}
In questo algoritmo si deve fare conto che \(B \to \cdot \gamma \notin prj(P)\) significa che non è ancora presente in \(closure_1(P)\) una produzione \(B \to \cdot \gamma\).
Di fatto quello che succede all'interno del \texttt{foreach} centrale è che se l'elemento è nuovo vado ad aggiungerlo con il suo \(\Delta\), se l'elemento è già presente vado a vedere se è il caso di aggiungere al \(\Delta\) di questa qualche elemento. Il procedimento potrebbe risultare un po' ostico, ma sarà tutto più chiaro dopo qualche esercizio esplicativo, come al solito.

\subsubsection{Esercizio di calcolo di \(closure_1(P)\)}
Facendo riferimento alla grammatica dell'esercizio precedente (riportata qui sotto), calcolare tutti gli stati dell'automa caratteristico LR(1).
\begin{align}
    \label{eq:ex3-slr1-grammarr}
    \G: S &\to aAd \mid bBd \mid aBe \mid bAe \\
    A &\to c \nonumber \\ \notag
    B &\to c \notag
\end{align}
\paragraph{Descrizione della procedura per lo stato 0}
Partiamo con l'inizializzazione dello stato \(0\), che conterrà la \(closure_1(\{[S' \to \cdot S, \{\$\}]\})\); questa inizializzazione sottolinea il fatto che, che una volta che avremo raggiunto il particolare stato in cui avremo l'accepting item \(S' \to S\cdot \), vorremo vedere il \$ che viene utilizzato come terminatore per definire una parola appartenente al linguaggio. Quindi, che si fa ora?

Secondo la Def.\ref{def:lr1-closure}, la \(closure_1(P)\) va inizializzata con \(P\) stesso, ovvero \([S' \to \cdot S, \{\$\}]\).
In seguito dobbiamo andare a prendere, all'interno della \(closure_1(P)\) (che è parziale, non ancora completa), tutti gli elementi in forma \([A \rightarrow \alpha \cdot B \beta, \Delta]\) e dovremo aggiungere alla stessa \(closure_1(P)\) tutti quegli item che soddisfano la forma \([B \rightarrow \cdot \gamma, \Gamma]\), dove \(\Gamma\) è un insieme calcolato come descritto dall'algoritmo; niente paura, in seguito sarà tutto più chiaro.

Nel nostro caso l'unico item in \(closure_1(P)\) è \([S' \to \cdot S, \{\$\}]\), che corrisponde alla forma desiderata (\([A \to a \cdot B\beta, \Delta]\)), quindi possiamo assumere \(\Delta = \{\$\} \textrm{ e } \beta = \varepsilon\); di conseguenza, \(first(\varepsilon \$) = first(\$) = \{\$\} \subseteq \Gamma\). Gli item che otteniamo da \(closure_1(\{[S' \to \cdot S, \{\$\}]\})\) saranno dunque:
\begin{align*}
    [&S' \to \cdot S, \{\$\}] \\
    [&S \to \cdot aAd, \{\$\}] \\
	[&S \to \cdot bBd, \{\$\}] \\
	[&S \to \cdot aBe, \{\$\}] \\
	[&S \to \cdot bAe, \{\$\}]
\end{align*}
Quanto abbiamo appena ottenuto è la chiusura dell'insieme che costituisce lo stato 0 del nostro automa caratteristico di tipo LR(1).
Come già detto prima, la prima parte di questi item è esattamente corrispondente all'item LR(0) che avremmo trovato utilizzando SLR parsing, la seconda parte (ovvero il \(\Delta\)) è l'unica novità introdotta dalla procedura LR(1).

\paragraph{Iterazione della procedura sulle transizioni}
È da notare anche che la componente del lookahead-set è usata solo in caso di riduzioni, e non incide sulle transizioni, per cui il procedimento rimarrà sotto quel punto di vista inalterato rispetto al metodo SLR. Di conseguenza, possiamo affermare la presenza, di tre transizioni che portano a tre stati differenti: \(\tau(0,S)=1 \textrm{, } \tau(0,a)=2 \textrm{ e } \tau(0,b)=3\). A questo punto è possibile procedere come abbiamo sempre fatto, prestando attenzione al calcolo della \(closure_1(S)\):
\begin{enumerate}
    \item Proseguiamo osservando lo stato \(\tau(0, S)=\)1. Il suo kernel è:
    \begin{equation*}
        [S' \to S \cdot, \{\$\}]
    \end{equation*}
    dobbiamo dunque calcolare \(closure_1(\{[S' \to S \cdot, \{\$\}]\})\), per cui sappiamo che \(B = \varepsilon\) e dunque non possiamo fare altro se non appuntarci che lo stato 1 contiene l'Accepting Item.
    \item Analizziamo dunque lo stato \(\tau(0, a)=\)2, il cui kernel è popolato da questi item:
    \begin{align*}
        [S &\to a \cdot Ad, \{\$\}] \\
        [S &\to a \cdot Be, \{\$\}]
    \end{align*}
    entrambi soddisfano la forma \([A \rightarrow \alpha \cdot B \beta, \Delta]\).
    Nel primo caso abbiamo le seguenti corrispondenze: \(A=S\), \(\alpha = a\), \(B = A\), \(\beta = d\) e \(\Delta=\$\); andiamo a cercare nella nostra grammatica se sono presenti derivazioni in forma \(B \to \gamma\). Troviamo \(A \to c\), quindi grazie alla formula del calcolo di \(closure_1(P)\) sappiamo che dobbiamo aggiungere agli item dello stato 2 un item formato così: \([A \to \cdot c, {\Gamma}]\), dove \(\Gamma = first(\beta\Delta) = first(d\$) = \{d\}\); in definitiva, aggiungiamo l'item \(A \to \cdot c, \{d\}\).

    Abbiamo detto che anche la seconda produzione (\([S \to a \cdot Be, \{\$\}]\)) soddisfa la forma \([A \rightarrow \alpha \cdot B \beta, \Delta]\), quindi con lo stesso procedimento appena adottato possiamo trovare la chiusura di \([S \to a \cdot Be, \{\$\}]\): in questo caso le corrispondenze sono: \(A=S\), \(\alpha=a\), \(B=B\), \(\beta=e\) e \(\Delta=\$\), la produzione da analizzare è \(B \to c\) e l'item che ne ricaviamo è \([B \to \cdot c, {\Gamma}]\), dove \(\Gamma = first(\beta\Delta) = first(e\$) = \{e\}\).
    
    In sostanza, lo stato 2 contiene questi LR(1) items:
    \begin{align*}
        [S &\to a \cdot Ad, \{\$\}] \\
        [S &\to a \cdot Be, \{\$\}] \\
        [A &\to \cdot c, \{d\}] \\
        [B &\to \cdot c, \{e\}]
    \end{align*}
    Come succedeva nell'esempio con il parsing LR(0), nello stato 2 possiamo osservare la presenza di tre transizioni e quindi di tre possibili nuovi stati che sono \(\tau(2,A)=4 \textrm{, } \tau(2,B)=5 \textrm{ e } \tau(2,c)=6\).
    \item[...]
    \item[6.] Ovviamente la parte interessante dell'esercizio è verificare se siamo riusciti a risolvere il conflitto che occorreva nel parsing SLR(1). In modo pressoché analogo a quanto accadeva, abbiamo che il kernel per lo stato 6 è :
    \begin{align*}
        [A &\to c \cdot, \{d\}] \\
        [B &\to c \cdot, \{e\}]
    \end{align*}
    Anche questa volta nello stato sono presenti due item di riduzione, ma non troviamo più il conflitto che avevamo con il parsing LR(0), perché in questo caso la riduzione \(A \to c\) si applica solo se nell'input buffer si sta leggendo \(d\), mentre la riduzione \(B \to c\) si applica solo nel caso in cui nell'input \emph{buffer} si sta leggendo \(e\). Questo implica che nello stato 6, grazie al parsing LR(1), è stata eliminata l'ambiguità del conflitto r/r.
\end{enumerate}

\subsubsection{Un secondo esempio}
\label{ex:closure-lr1}
Sia data la seguente grammatica:
\begin{align}
    \G: S &\to L=R \mid R \\
    L &\to *R \mid id \nonumber \\ \notag
    R &\to L \notag
\end{align}
Inizializziamo lo stato 0 ponendo come suo kernel:
\begin{equation*}
    [S' \to \cdot S, \{\$\}]
\end{equation*}
Calcoliamo \(closure_1(0)\). Questo item soddisfa la forma \([A \rightarrow \alpha \cdot B \beta, \Delta]\), con le seguenti corrispondenze: \(B = S \texttt{, } \beta = \varepsilon \texttt{, } \Delta = \{\$\}\) (e quindi \(\Gamma = first(\beta \Delta) = first(\varepsilon \$) = \{\$\}\)).
Le due produzioni di \(B\) (ovvero di \(S\)) in questo caso sono \(S \to L=R \mid R\), quindi aggiungo i due item LR(1) seguenti:
\begin{align}
    \label{prod:production-L}
    [S &\to \cdot L = R, \{\$\}] \\
    [S &\to \cdot R, \{\$\}] \notag
\end{align}
A questo punto la chiusura non è completa in quando abbiamo aggiunto due elementi che possono essere presi in considerazione nella definizione ricorsiva della chiusura, ovvero soddisfano anch'essi la famosa forma \([A \rightarrow \alpha \cdot B \beta, \Delta]\).
Analizziamo dunque \([S \to \cdot L = R, \{\$\}]\), le sue corrispondenze sono: \(B = L \texttt{, } \beta = \; \; =R \texttt{, } \Delta = \{\$\}\) (e quindi \(\Gamma = first(\beta \Delta) = first(=R\$) = \{=\}\)); le produzioni di \(L\) sono \(L \to *R \mid id\), quindi arriviamo a generare i seguenti LR(1)-items:
\begin{align*}
    [L &\to \cdot *R, \{=\}] \\
    [L &\to \cdot id, \{=\}]
\end{align*}
che non presentano altre possibili espansioni (grazie al cielo).

Ci rimane ora da analizzare \([S \to \cdot R, \{\$\}]\), che soddisfa la nostra formosa forma e presenta le seguenti corrispondenze: \(B = R \texttt{, } \beta = \{\varepsilon\} \texttt{, } \Delta = \{\$\}\) (e quindi \(\Gamma = first(\beta \Delta) = first(\varepsilon \$) = \{\$\}\)), inoltre le produzioni di \(R\) sono \(R \to L\), quindi otteniamo il seguente LR(1)-items:
\begin{align*}
    [R &\to \cdot L, \{\$\}]
\end{align*}
questo item presenta il marker davanti a un non-terminale; può generare nuovi item per la chiusura che stiamo calcolando?
Nonostante la produzione con driver \(L\) sia già stata analizzata (vedi Eq.\ref{prod:production-L}), è necessario eseguire nuovamente la sua chiusura in quanto ha un lookahed-set differente (nel caso precedente \(\beta= \; \;  =R\), mentre ora \(\beta=\varepsilon\)), il che ci porta a ripetere la chiusura per l'item \([R \to \cdot L, \{\$\}]\), ricavando i seguenti LR(1)-items:
\begin{align*}
    [L &\to \cdot *R, \{\$\}] \\
    [L &\to \cdot id, \{\$\}]
\end{align*}
Visto che però avevamo già inserito due LR(1)-item con lo stesso LR(0)-item (la parte \(L \to \cdot qualcosa\)) possiamo semplicemente accrescere i lookahead-set corrispondenti. Il risultato finale per la chiusura degli item dello stato iniziale è:
\begin{align*}
    [&S'\to \cdot S, \{\$\}] \\
    [&S \to \cdot L = R, \{\$\}] \\
    [&S \to \cdot R, \{\$\}] \\
    [&L \to \cdot *R, \{\$, =\}] \\
    [&L \to \cdot id, \{\$, =\}] \\
    [&R \to \cdot L, \{\$\}]
\end{align*}
Bene, ora abbiamo fatto un po' di allenamento con il calcolo degli stati, ma dobbiamo ricordare che non è questo il nostro obiettivo finale, noi vogliamo ricavare l'automa caratteristico!

\subsection{Costruzione di un automa caratteristico LR(1)}
Finalmente possiamo spiegare come si costruisce un automa caratteristico LR(1). 

\paragraph{Idea}
L'idea di base è semplice:
\begin{enumerate}
    \item costruiamo lo stato di partenza con l'item \([S' \to \cdot S, {\$}]\);
    \item ricorsivamente aggiungiamo gli stati che si possono raggiungere dagli stati presenti nell'automa;
    \item aggiungamo le transizioni che ci permettono di passare da uno stato all'altro.
\end{enumerate}
E come facciamo a capire quali stati sono raggiungibili a partire da un certo stato \(P\)? Se uno stato \(P\) contiene un item nella forma \([A \to \alpha \cdot Y \beta, \Delta]\) allora esiste una transizione da \(P\) ad uno stato \(Q\) che contiene l'item \([A \to \alpha Y \cdot \beta , \Delta]\); inoltre, siccome \(Q\) contiene \([A \to \alpha Y \cdot \beta , \Delta]\), esso contiene anche tutti gli item in \(closure_1([A \to \alpha Y \cdot \beta , \Delta])\).

\paragraph{Algoritmo}
Proviamo ora a chiarire tale procedura tramite la sua scrittura in forma algoritmica, si veda \ref{alg:lr1-automata}.

% \begin{figure}
%     \centering
%     \includegraphics[width=\textwidth]{alg-construct-lr1-automata.png}
%     \caption{Algoritmo per la costruzione di un automa LR(1)}
%     \label{alg:construct-lr1-automata}
% \end{figure}
\subimport{assets/pseudocode/}{lr1-automata.tex}
\subsubsection{Esempio di costruzione}
Indovinate un po' con quale grammatica stiamo per andare a osservare cosa si ottiene con questa procedura? Oh sì! proprio lei:
\begin{align}
    \label{eq:ex3-slr1-grammarrrrrr-again-and-again}
    \G: S &\to aAd \mid bBd \mid aBe \mid bAe \\
    A &\to c \nonumber \\ \notag
    B &\to c \notag
\end{align}
Ora chiederemo al lettore di fare uno sforzo mnemonico poderoso e di ricordare che in passato, quando abbiamo provato a costruire l'automa caratteristico SLR(1) per questa grammatica ci siamo trovati ad avere la situazione rappresentata in figura \ref{fig:lr0-automata_conflict_2}, che riprtiamo qui sotto per comodità.
\begin{figure}[H]
    \centering
    \subimport{assets/figures/}{automa_conflict_LR.tex}
    \caption{Automa di tipo LR(0), si noti il conflitto sullo stato 6}
    \label{fig:lr0-automata_conflict_2}
\end{figure}
Abbiamo già calcolato quali sono gli stati LR(1) per la grammatica in questione in un esercizio precedente (vedi \ref{ex:closure-lr1}), ed applicando l'algoritmo per la costruzione degli automi LR(1) arriviamo ad ottenere il seguente automa (parziale):
\begin{figure}[H]
    \centering
    \subimport{assets/figures/}{automa_conflict_solution_LR.tex}
    \caption{Automa di tipo LR(1), si noti che il conflitto sullo stato 6 è stato eliminato}
    \label{fig:lr1-automata_no-conflict}
\end{figure}
Ma siamo sicuri di aver effettivamente risolto il problema? per verificarlo dobbiamo costruire la parsing table!

\paragraph{}
Nonostante qui la prenderemo come magia nera per non spostare il focus dalla costruzione dell'automa in sé, possiamo anticipare che la costruzione della tabella di parsing LR(1) è semplice da ottenere, perché si crea esattamente come la sua corrispondente SLR(1), con queste uniche due differenze:
\begin{itemize}
    \item l'automa caratteristico è di tipo LR(1);
    \item la lookahead function utilizzata è la seguente: per ogni \([A \to \beta \cdot , \Delta] \in P\), \(\mathcal{LA}(P, A \to \beta ) = \Delta\).
\end{itemize}
Naturalmente una grammatica \(\mathcal{G}\) è di tipo LR(1) se la sua tabella di parsing LR(1) non presenta conflitti. Sbirciamo quindi cosa sarebbe successo costruendo la parsing table LR(1) per la nostra grammatica preferita.

\paragraph{}
Innanzitutto, l'automa caratteristico LR(1) si presenta come in Fig.\ref{fig:lr1_automata-complete}
\begin{figure}[H]
    \centering
    \subimport{assets/figures/}{automa_LR_complete_solution.tex}
    \caption{Automa di tipo LR(1) completo}
    \label{fig:lr1_automata-complete}
\end{figure}
Negli stati 6 e 9, che ci avrebbero causato problemi con la costruzione di tipo LR(0), ora troviamo i seguenti item:
\begin{itemize}
    \item Stato 6
    \begin{align*}
        A &\to c \cdot , \{d\} \\
        B &\to c \cdot , \{e\}    
    \end{align*}
    \item Stato 9
    \begin{align*}
        A &\to c \cdot , {e} \\
        B &\to c \cdot , {d}    
    \end{align*}
\end{itemize}
 Quindi, seguendo le regole di compilazione della tabella di parsing LR(1) avremmo:
 \begin{itemize}
     \item nella casella M[6,d] la riduzione \(A \to c \cdot , {d}\);
     \item nella casella M[6,e] la riduzione \(B \to c \cdot , {e}\);
     \item nella casella M[9,e] la riduzione \(A \to c \cdot , {e}\);
     \item nella casella M[9,d] la riduzione \(B \to c \cdot , {d}\);
 \end{itemize}
Festa! Festa! non ci sono più conflitti!
È però notevole il fatto che il numero di stati sia aumentato parecchio: questo è quello che succede quando si va ad utilizzare tecniche di parsing più precise e raffinate.

\section{Costruzione di una tabella di parsing LR(1)}
Chi avesse seguito con sufficiente costanza e attenzione a questo punto dovrebbe sapere che non ci sarà alcuna nuova informazione in questa sezione, perché la procedura di costruzione della tabella di parsing è la medesima per tutte le tipologie di parsing bottom-up. Di conseguenza, se qualcuno avesse bisogno di una rinfrescata, suggeriamo di tornare al capitolo precedente prima di proseguire; per tutti gli altri, rimbocchiamoci le maniche e andiamo a mettere le mani in pasta su queste tabelle di parsing.  

\subsection{La grammatica dei puntatori}
La grammatica che scegliamo come esempio è questa:
\begin{align}
    \label{gr:pointers-grammar}
    S &\to L = R \mid R \\
    L &\to *R \mid id \nonumber \\ \notag
    R &\to L \notag
\end{align}
Questa è una grammatica che ci parla di puntatori, ma capiremo in seguito cosa significa ciò, per ora concentriamoci sulla sua risoluzione. 

\subsubsection{Definizione stati e costruzione dell'automa}
Il primo passo, naturalmente, è la costruzione dell'automa caratteristico. Per fare questo è necessario prima definire come sono fatti gli stati del nostro automa, ossia quali LR(1)-items comprenderanno; per fare questo, sappiamo che abbiamo la nostra fida \(closure_1\). Aggiungiamo la produzione canonica \(S' \to S\) e cominciamo.

\paragraph{Stato 0}
Il kernel dello stato è \(S' \to \cdot S\), a cui andiamo subito ad aggiungere il risultato del calcolo di \(closure_1(S)\). Ricordiamo che il lookahead set \(\Gamma\) si calcola come first(\(\beta\Delta\)); nel caso di \(closure_1(S)\) abbiamo \(\beta = \varepsilon\) e \(\Delta = \{\$\}\).
\begin{align*}
    &&Kernel: &&S' &\to \cdot S, \{\$\}& \\
    &&closure_1(S): &&S &\to \cdot L = R, \{\$\} &\\
    && &&S &\to \cdot R, \{\$\} &
\end{align*}
Abbiamo calcolato la chiusura per \(S\), ma adesso  dobbiamo calcolarla anche per \(L\) e \(R\), dal momento figurano in un qualche item con il marker \(\cdot\) davanti. Per quanto riguarda il calcolo di \(\Gamma\):
\begin{itemize}
    \item in \(closure_1(L)\) abbiamo che \(\beta = \{=R\}\) e \(\Delta = \{\$\}\);
    \item in \(closure_1(R)\) abbiamo che \(\beta = \varepsilon\) e \(\Delta = \{\$\}\);
\end{itemize}
\begin{align*}
    &&closure_1(L): &&L &\to \cdot \ast R, \{=\} &\\
    && &&L &\to \cdot id, \{=\} &\\
    &&closure_1(R): &&R &\to \cdot L, \{\$\}&
\end{align*}
Non è ancora finita purtroppo, perché trovato un altro item con il marker \(\cdot\) davanti alla \(L\); che facciamo, dobbiamo ricalcolarla? Sì, dobbiamo farlo assolutamente! Nel caso precedente la produzione \(S \to \cdot L = R\) aveva \(\beta = \{=R\}\), mentre in questo caso la produzione è \(R \to L\) e ha \(\beta = \varepsilon\), per cui il lookahead set delle produzioni di \(L\) potrebbe essere diverso in questo secondo caso, e difatti è proprio questo che succede, perché il calcolo della chiusura ci ritorna questo:
\begin{align*}
    L &\to \cdot \ast R, \{\$\} \\
    L &\to \cdot id, \{\$\}
\end{align*}
Quindi lo stato 0 contiene i seguenti items, e presenta queste transizioni:
\begin{align*}
    &\textrm{items:} & &\textrm{transizioni:} \\
    &S' \to \cdot S, \{\$\}  & &\tau(0, S) = 1 \\
    &S \to \cdot L = R, \{\$\}  & &\tau(0, L) = 2 \\
    &S \to \cdot R, \{\$\}  & &\tau(0, R) = 3 \\
    &L \to \cdot \ast R, \{=, \$\}  & &\tau(0, \ast) = 4 \\
    &L \to \cdot id, \{=, \$\}  & &\tau(0, id) = 5 \\
    &R \to \cdot L, \{\$\}  & &
\end{align*}

\paragraph{Stato 1}
Andiamo adesso ad analizzare lo stato 1, che ha kernel 
\begin{equation*}
    S' \to S\cdot, \{\$\}    
\end{equation*}
È l'accepting item, non c'è nulla da dire oltre a questo (e soprattutto nulla da calcolare).

\paragraph{Stato 2}
Lo stato 2 ha questo kernel:
\begin{align*}
    S &\to L \cdot = R, \{\$\} \\
    R &\to L \cdot, \{\$\}
\end{align*}
Questo stato è già chiuso, perché nel primo item il marker \(\cdot\) si trova davanti al terminale \(=\), mentre invece il secondo item ci propone la riduzione \([R \to L, \{\$\}]\); inoltre, il primo item ci propone la transizione \(\tau(2, =)\). Notiamo anche che in questo stato abbiamo sia uno shift che una riduzione, teniamolo a mente per dopo.

\paragraph{Stato 3}
Passiamo allo stato 3, e il kernel ci porta delle belle notizie:
\begin{equation*}
    S \to R\cdot, \{\$\}    
\end{equation*}
Questo stato è già chiuso e l'unica cosa che ci regala è la mossa di reduce \([S \to R\cdot, \{\$\}]\).

\paragraph{Stato 4}
Il kernel di questo stato, purtroppo, non è altrettanto lungimirante:
\begin{equation*}
    L \to \ast \cdot R, \{=, \$\}
\end{equation*}
Siamo infatti costretti a calcolare la chiusura \(closure_1(R)\), che ci porta a inserire l'item \(R \to \cdot L, \{=, \$\}\) allo stato; quest'ultimo, dal canto suo, ci impone di inserire allo stato la chiusura \(closure_1(L)\), dal momento che abbiamo il marker \(\cdot\) davanti al non-terminale \(L\). A fine del processo lo stato appare così:
\begin{align*}
    L &\to \ast \cdot R, \{=, \$\} \\
    R &\to \cdot L, \{=, \$\} \\
    L &\to \cdot \ast R, \{=, \$\} \\
    L &\to \cdot id, \{=, \$\} 
\end{align*}
Le mosse di shift che troviamo sono invece:
\begin{gather*}
    \tau(4, R) \\
    \tau(4, L) \\
    \tau(4, \ast) \\
    \tau(4, id)
\end{gather*}

\paragraph{Stato 5}
Abbiamo nuovamente uno stato tranquillo. Il kernel è:
\begin{equation*}
    L \to id \cdot, \{=, \$\}
\end{equation*}
Questo ci offre la rispettiva riduzione (\([L \to id \cdot, \{=, \$\}]\)), senza alcuna mossa di shift.

\paragraph{Stato 6}
Questo stato viene calcolato seguendo la transizione \(\tau(2, =)\). Dal momento che abbiamo il marker \(\cdot\) davanti al non-terminale \(R\), dobbiamo calcolare \(closure_1(R)\), che in omaggio ci porta anche il calcolo di \(closure_1(L)\).
\begin{align*}
    &&Kernel: &&S &\to L = \cdot R, \{\$\}& \\
    &&closure_1(R): &&R &\to \cdot L, \{\$\} &\\
    &&closure_1(L): &&L &\to \cdot \ast R, \{\$\}&\\
    && &&L &\to \cdot id,  \{\$\}&
\end{align*}
Troviamo anche le seguenti mosse di shift:
\begin{gather*}
    \tau(6, R) \\
    \tau(6, L) \\
    \tau(6, \ast) \\
    \tau(6, id)
\end{gather*}

\paragraph{Stato 7}
Calcoliamo questo stato grazie allo shift \(\tau(4, R)\). Il kernel:
\begin{equation*}
    L \to \ast R \cdot, \{=, \$\}
\end{equation*}
Un altro stato tranquillo insomma, che ci offre solamente una riduzione (\([L \to \ast R \cdot, \{=, \$\}]\)) e nulla di più. 

\paragraph{Stato 8}
Il copione è esattamente lo stesso di quanto visto sopra: calcoliamo questo stato grazie allo shift \(\tau(4, L)\). Il kernel:
\begin{equation*}
    R \to L \cdot, \{=, \$\}
\end{equation*}
Andiamo avanti lisci, abbiamo la nostra riduzione \([R \to L \cdot, \{=, \$\}]\) e nulla di più.

\paragraph{Stato 9}
Anche in questo stato abbiammo del lavoro da fare. Ci arriviamo tramite \(\tau(4, \ast)\), e il kernel si presenta così:
\begin{equation*}
    L \to \ast \cdot R, \{=, \$\}
\end{equation*}
Ma fermi... è esattamente il kernel dello stato 4! Questo vuol dire che troviamo subito una transizione \(\tau(4, \ast) = 4\), il che significa che lo stato 4 ha un self loop per \(\ast\).

\subparagraph*{}
Proviamo allora a ricavare uno stato 9 dalla transizione \(\tau(4, id)\). Il kernel è questo:
\begin{equation*}
    L \to id \cdot, \{=, \$\}
\end{equation*}
E niente, è di nuovo un altro kernel che abbiamo già visto nello stato 5, per cui concludiamo che \(\tau(4, id) = 5\).

\subparagraph*{}
Facciamo un altro tentativo, questa volta con \(\tau(6, R)\). Il kernel è:
\begin{equation*}
    S \to L = R \cdot, \{\$\}
\end{equation*}
Fatto, finalmente siamo riusciti a creare un nuovo stato con questa transizione \(\tau(6, R) = 9\). Non abbiamo altre chiusure da calcolare, e torniamo a casa con la riduzione \([S \to L = R \cdot, \{\$\}]\).

\paragraph{Stato 10}
Adesso seguiamo la transizione \(\tau(6, L)\). Il kernel è:
\begin{equation*}
    R \to L \cdot, \{\$\}
\end{equation*}
È vero, abbiamo già uno stato con questo kernel (lo stato 8), ma in quel caso il lookahead set è diverso, per cui creiamo ugualmente un nuovo stato. Abbiamo una mossa di shift \(\tau(6, L) = 10\) e la riduzione data dal kernel \([R \to L \cdot, \{\$\}]\).

\paragraph{Stato 11}
La situazione è molto simile a quella descritta sopra. Seguendo la transizione \(\tau(6, \ast)\) troviamo:
\begin{equation*}
    L \to \ast \cdot R, \{\$\}
\end{equation*}
Questo è lo stesso kernel dello stato 9, ma per via del differente lookahead set creiamo comunque un nuovo stato (\(\tau(6, \ast) = 11\)). Dobbiamo anche calcolare \(closure_1(R)\) che, anche quessta volta, forza il calcolo di \(closure_1(L)\); lo stato sarà infine composto di questi items:
\begin{align*}
    &&Kernel: &&L &\to \ast \cdot R, \{\$\}& \\
    &&closure_1(R): &&R &\to \cdot L, \{\$\} &\\
    &&closure_1(L): &&L &\to \cdot \ast R, \{\$\}&\\
    && &&L &\to \cdot id,  \{\$\}&
\end{align*}
Notiamo che, con l'eccezione dell'ultimo, tutti gli item sono stati già visitati, e già sappiamo dove portano (\(R \to \cdot L, \{\$\}\) a \(\tau(11, L) = 10\) e \(L \to \cdot \ast R, \{\$\}\) a \(\tau(11, \ast) = 11\), self loop), per cui ci risparmiamo di analizzare in futuro quelle mosse di shift. Ci sentiamo anche di anticipare che pure l'ultima transizione, \(\tau(11, id)\), non porterà a un nuovo stato, poiché risulterà essere un collegamento allo stato 12 (\(\tau(11, id) = 12\)).

\paragraph{Stato 12}
Uh, per fortuna questo stato è tranquillo. Stiamo infatti analizzando la transizione \(\tau 11, R\) e il kernel è:
\begin{equation*}
    L \to id \cdot, \{\$\}
\end{equation*}
E, come ormai sappiamo, ci propone solamente la rispettiva mossa di riduzione.

\paragraph{Stato 13}
Dulcis in fundo, analizziamo la transizione \(\tau(11, R)\). Il kernel è:
\begin{equation*}
    L \to \ast R \cdot , \{\$\}
\end{equation*}
Lo stato 7 ha lo stesso kernel LR(0), ma ancora, il lookahead set è diverso, per cui creiamo comunque un nuovo stato \(\tau(11, R) = 13\). Nessuna nuova transizione, solamente la riduzione proposta dal kernel.

Rimarrebbe da analizzare \(\tau(11, id)\) ma, come già anticipato, conduce allo stato 12, per cui non c'è bisogno di aggiungere un quattordicesimo stato al nostro automa.

% \begin{figure}[H]
%     \centering
%     \includegraphics[width=.8\textwidth]{send_help_1.png}
%     \caption{help}
% \end{figure}
% \begin{figure}[H]
%     \centering
%     \includegraphics[width=.8\textwidth]{send_help_2.png}
%     \caption{help}
% \end{figure}
% Per tutti gli stati 8,9 e 10 non ci sono chiusure ma si presentano riduzioni. Nello stato 11 il kernel permette di cercare una chiusura per R, la chiusura per R ci porta a R -> .L, {\$} che ci permette un'ulteriore chiusura, con conseguente inserimento di L->.*R,{\$} e L->.id,{\$}. Passiamo allo stato 12 ha un kernel che potrebbe sembrare uguale allo stato 5, ma il lookahead set è diverso… quindi lo stato 12 è diverso dal 5 Aggiungiamo anche lo stato 13 per rappresentare la transizione \(\tau(11,R)\) ecc. ecc. Nota che la Quaglia non ha riportato tutti gli stati ed i relativi items. Questo è l'automa che ricaviamo:
\paragraph{Costruzione dell'automa}
Ora che abbiamo gli stati non c'è più nulla di diverso rispetto a quello che facevamo prima, per cui possiamo terminare con la costruzione dell'automa caratteristico. Quest'ultimo (Fig.\ref{fig:pointers-automaton}) è alquanto complicato e pare una strana mappa del tesoro, ma ce l'aspettavamo: abbiamo già sottolineato come e perché gli automi LR(1) sono più complessi di un corrispondente SLR(1).
\begin{figure}[H]
    \centering
	\subimport{assets/figures/}{figures_9-7.tex}
    \caption{Automa caratteristico LR(1) per Eq.\ref{gr:pointers-grammar}}
    \label{fig:pointers-automaton}
\end{figure}
A questo punto potremmo chiederci: la grammatica Eq.\ref{gr:pointers-grammar} è effettivamente LR(1)? Non è che abbiamo fatto fatica per niente? Per rispondere, la prima cosa che ci viene in mente è costruire la tabella di parsing a partire dal nostro automa, ma possiamo essere più scaltri di così.
% \begin{figure}[H]
%     \centering
%     \includegraphics[width=.8\textwidth]{send_help_4.png}
%     \caption{help}
% \end{figure}
\paragraph{La grammatica è LR(1)?}
Infatti, possiamo fin da subito escludere la presenza di conflitti r/r (ossia reduce/reduce), perché nessuno stato del nostro automa presenza due reducing items.

Anche per verificare la presenza di conflitti s/r (shift/reduce) abbiamo una furberia simile, anche se un po' meno veloce: dobbiamo accertarci che nessuno degli stati che contengono reducing items abbia contemporaneamente delle mosse di shift; detto in termini semplici, i cerchietti col doppio cerchiettino non devono avere una freccettina uscente. Una volta che abbiamo attivato la nostra vista di falco possiamo vedere che gli stati 3, 5, 7, 8, 9, 10, 12, 13 hanno tutti il doppio cerchiettino (reducing items) ma non hanno nessuna freccia uscente, il che ci farebbe festeggiare, se non fosse per quel dannato stato 2, colpevole di avere e il doppio cerchiettino, e la freccia uscente. È finita, quindi? Abbiamo davvero un conflitto? 

No, fortunatamente: la riduzione, infatti, si applica solo quando il prossimo simbolo in lettura è \$, mentre invece lo shift viene applicato quando in lettura c'è =; non essendoci ulteriori possibilità di conflitto, possiamo felicemente asserire che la nostra grammatica è LR(1).

\paragraph{La grammatica è SLR(1)?}
Per rispondere a questa domanda è comunque conveniente costruirne il relativo automa caratteristico; per gli smemorati, ricordiamo che l'automa caratteristico SLR(1) si costruisce con stati di LR(0)-items e utilizzando come lookahead function l'insieme follow(\(P\)). Il risultato finale dovrebbe essere più o meno come Fig.\ref{fig:pointers-automaton-slr1}.
\begin{figure}[H]
    \centering
	\subimport{assets/figures/}{figures_9-9.tex}
    \caption{Automa caratteristico SLR(1) per Eq.\ref{gr:pointers-grammar}}
    \label{fig:pointers-automaton-slr1}
\end{figure}
E purtroppo vediamo subito il problema: dal momento che la nostra lookahead function è follow(B), lo stato 2 presenta sia una mossa di shift verso =, sia una mossa di reduce. Pertanto, nella tabella avremo un conflitto nella cella [2, =], e tanto basta per permetterci di dire che la grammatica Eq.\ref{gr:pointers-grammar} non è SLR(1).

\paragraph{La grammatica dei puntatori}
Approfondiamo il discorso su questa grammatica.
% Questo paragrafo al momento è vuoto perché a lezione il discorso è stato lasciato cadere magistralmente, ma forse poi ci torniamo sopra perché pare interessante.
% Abbiamo già detto in passato come questa sia una grammatica che ci parla di puntatori, andiamo ora ad indagare più a fondo tale affermazione:

% L sta per left
% R sta per right

% Notiamo che il non-terminale S si presenta una sola volta per derivazione, ed essendo l'unico che genera = allora l'= può essere presente all'interno di una derivazione al massimo una volta.. Cosa sappiamo del resto? Che linguaggio possiamo ricavare da tale grammatica? % {( (*)^n id ) = ( (*)^k id ), n,k>=0} U {(*)^n id, n>=0} forse Continueremo questo discorso? Mah

\begin{figure}[H]
    \centering
	\subimport{assets/figures/}{figures_9-10.tex}
    \caption{Automa caratteristico SLR(1) per Eq.\ref{gr:pointers-grammar}, con evidenziati conflitti e grammatica di partenza}
    \label{fig:pointers-automaton-slr1-conflict}
\end{figure}
Cerchiamo di capire come mai arrivare nello stato 2 ci porta a un conflitto, se utilizziamo un automa SLR(1). La prima cosa che osserviamo è che 2 è raggiungibile solo dallo stato 0 e solo quando in lettura troviamo L; poiché L è un non-terminale, non arriveremo mai a 2 con una mossa di shift; l'unico modo per  raggiungere 2 è una mossa di goto da 0, e perché questo succeda vuol dire che devo aver prima operato una riduzione tale che sono ritornato in 0 e mi trovo con L nella pila dei simboli. Una riduzione che potrebbe asservire lo scopo è la riduzione dello stato 5; capiamolo meglio eseguendo il parsing della parola \(w = "id=id"\).

\paragraph{Parsing di \(w = id\!=\!id\)}
Quindi, penna alla mano, lanciamo l'algoritmo shift/reduce. Al lancio della procedura abbiamo queste strutture:
\begin{align*}
    w &= id\!=\!id\$ \\
    stSt &= 0 \\
    symSt &= 
\end{align*}
Da 0 leggo \(id\) ed eseguo la relativa transizione verso lo stato 5. Situazione:
\begin{align*}
    w &= \underline{id}\!=\!id\$ \\
    stSt &= 0\;5 \\
    symSt &= id
\end{align*}
In 5 troviamo la riduzione \(L \to id\), per cui aggiorniamo le nostre strutture secondo le regole che ben conosciamo; ci dovremmo quindi ritrovare di nuovo nello stato 0 e, leggendo \(=\) e compiendo la transizione a 2, ci troviamo finalmente nello stato del conflitto:
\begin{align*}
    w &= \underline{id}\!=id\$ \\
    stSt &= 0\;\xout{5}\;2\\
    symSt &= \xout{id}\;L 
\end{align*}
Come risolviamo quindi questo conflitto s/r? Cosa applico, la reduce \(R \to L\), oppure uno shift verso lo stato 6? Distogliamo un attimo gli occhi dal ragionamento e guardiamo la grammatica da cui siamo partiti (convenientemente riportata in Fig.\ref{fig:pointers-automaton-slr1-conflict}): non ha nessun senso applicare la riduzione, perché durante la procedura non abbiamo mai eseguito produzioni \(R \to L\), per cui in questo caso la mossa giusta è lo shift.

\paragraph{Parsing di \(w = ***id\!=\!id\)}
Proviamo adesso con un'altra parola, questa volta con tre terminali \(\ast\). L'inizio è il medesimo della parola precedente:
\begin{align*}
    w &= ***id\!=\!id\$ \\
    stSt &= 0 \\
    symSt &=  
\end{align*}
Le prime tre mosse sono piuttosto lisce: leggiamo tre \(\ast\), per cui inizialmente percorreremo la \(\ast\)-transizione da 0 a 4, e successivamente percorreremo due volte il self-loop dello stato 4; infine, leggiamo \(id\) e operiamo la transizione verso lo stato 5. A questo punto le nostre strutture sono così:
\begin{align*}
    w &= \underline{***id}\!=id\$ \\
    stSt &= 0\;4\;4\;4\;5 \\
    symSt &= ***id 
\end{align*}
E niente, c'è la riduzione \(L \to id\) da fare: rimuovo 5 da \(stSt\) e sostituisco il body \(id\) con il driver \(L\) in \(symSt\). Notiamo che, a differenza di prima, la riduzione in questo caso ci fa tornare allo stato 4; da questo, quindi, leggiamo \(L\) e transizioniamo verso lo stato 8.
\begin{align*}
    w &= \underline{***id}\!=id\$ \\
    stSt &= 0\;4\;4\;4\;8 \\
    symSt &= ***L 
\end{align*}
Non è ancora finita, perché in 8 troviamo un'altra riduzione, in particolare \(R \to L\). Aggiorniamo le strutture e torniamo indietro allo stato 4; leggendo quindi \(R\) ci ritroviamo nello stato 7, che ci porta a operare una nuova riduzione.
\begin{align*}
    w &= \underline{***id}\!=id\$ \\
    stSt &= 0\;4\;4\;4\;7 \\
    symSt &= ***L 
\end{align*}
La riduzione da operare è \(L \rightarrow\; ^{*}R\), per cui cancelliamo gli ultimi due stati da \(stSt\) e modifichiamo contestualmente la testa di \(symSt\), per ritrovarci infine in questa situazione:
\begin{align*}
    w &= \underline{***id}\!=id\$ \\
    stSt &= 0\;4\;4\;8 \\
    symSt &= **L 
\end{align*}
Siamo di nuovo nello stato 4 con L in lettura: la medesima situazione in cui ci eravamo ritrovati poco sopra! Questo vuol dire che andremo a ripetere di nuovo le stesse operazioni: transizione verso 8, riduzione \(R \to L\), transizione verso 7, riduzione \(L \to\; ^{*}R\); questo ciclo andrà iterato finché tutti gli \(\ast\) non saranno spariti, e descrive molto bene quello che succede a livello macchina quando andiamo a risolvere un puntatore. Infatti, quando avremo risolto tutti gli \(\ast\), saremo di nuovo nello stato 0 e avremo un goto \(L\), punto di boa a cui eravamo arrivati anche nella precedente parola analizzata. Adesso completiamo il ragionamento e andiamo fino in fondo.
\subparagraph*{}
Itaque seguiamo il goto \(L\) e dirigiamoci verso le acque burrascose del tormentato stato 2. 
\begin{align*}
    w &= \underline{id}\!=id\$ \\
    stSt &= 0\;2\\
    symSt &= L 
\end{align*}
Che fare? Non possiamo più affidarci a ragionamenti pigri come quello che abbiamo fatto prima, ma magari ci va bene uguale; scegliamo di fare uno shift e andiamo avanti a testa alta e petto in fuori, diretti verso la gloria o verso il baratro. Arriviamo quindi a 6.
\begin{align*}
    w &= \underline{id\!=}id\$ \\
    stSt &= 0\;2\;6 \\
    symSt &= L\!=
\end{align*}
In lettura abbiamo \(id\), per cui facciamo uno shift verso 5; bruciamo la tappa ed eseguiamo subito la riduzione \(L \to id\) che ci fa rimbalzare indietro dallo stato 6 e infine allo stato 8.
\begin{align*}
    w &= \underline{id\!=id}\$ \\
    stSt &= 0\;2\;6\;\xout{5}\;8 \\
    symSt &= L\!=\!\xout{id}\!L
\end{align*}
Di nuovo una riduzione: questa \(R \to L\) ci fa saltapicchiare nuovamente allo stato 6 e questa volta finiamo nello stato 9. Situazione pile:
\begin{align*}
    w &= \underline{id\!=id}\$ \\
    stSt &= 0\;2\;6\;\xout{8}\;9 \\
    symSt &= L\!=\!\xout{L}\!R
\end{align*}
Ci siamo: in 9 abbiamo la riduzione \(S \to L = R\), per cui torniamo indietro allo stato 0, leggiamo \(S\) e corriamo veloci ad abbracciare l'accepting item nello stato 1.
\subparagraph*{}
Nonostante la scelta arbitraria, ci è decisamente andata bene; ma se invece dello shift avessimo scelto la riduzione? Torniamo a quel bivio e proviamo a percorrere anche quest'altra strada. Operando la riduzione \(R \to L\) saremmo passati allo stato 3:
\begin{align*}
    w &= \underline{id}\!=id\$ \\
    stSt &= 0\;3 \\
    symSt &= R
\end{align*}
Da qui leggiamo \(=\), ma non abbiamo assolutamente nessuna mossa da fare: cadiamo nel baratro delle caselle \texttt{error}, e il nostro parsing si ferma qui. La scelta giusta, di nuovo, era lo shift.

% LALR
% Le grammatiche LALR sono tali per cui la loro complessità in spazio (la dimensione della tabella) è
% esattamente uguale a SLR ma la funzione di lookahead è ben più raffinata.
% Mentre in SLR si va ad usare il follow(B), mentre in SLR(1) si va a raffianre questo insieme, non sono
% più i follow del driver, ma un sottinsieme appunto raffinato.
% LALR si posiziona a metà tra questi due casi, si esegue il raffinamento solo in alcuni casi.
% Vediamo subito cosa si intende
% A proposito della grammatica che abbiamo appena finito di trattare ci sono, negli stati dell'automa
% lr1, degli stati che hanno esattamente gli stessi item-lr0 dell'automa lr0.
% Gli stati in questione sono:
% \begin{figure}[H]
%     \centering
%     \includegraphics[width=.8\textwidth]{send_help_7.png}
%     \caption{help}
% \end{figure}
% SI nota di fatto che tra i due automi lr0 e lr1 esiste una sottostruttura che è replicata in entrambi,
% questa similarità si presenta esattamente in questi stati.
% Il processo verso l'automa lalr prevede di passare per l'automa intermedio LRm(1), ovvero LR(1)
% merged
% Tutti gli stati di lr1 ma si mergiano quegli stati che appunto hanno lr0 item uguali tra di loro
% Ad esempio lo stato 4 e lo stato 11 hanno esattamente gli stessi item-LR(0), quindi si possono
% mergiare!
% Gli stati dell'automa LRm1 sono:
% Le transizioni invece:
% \begin{figure}[H]
%     \centering
%     \includegraphics[width=.8\textwidth]{send_help_8.png}
%     \caption{help}
% \end{figure}
% Ora che ho l'automa LRm(1) costruisco la tabella di parsing LALR in questo modo:

% - Utilizzo l'automa LRm(1) per costruire la tabella
% - Le riduzioni le decido utilizzando come lookahead function l'unione dei \(\Delta\) degli item-LR(1)
% degli stati (se uno stato è merged vado a vedere tutti i singoli Delta degli stati che ho mergiato) - Ora che ho l'automa LRm(1) costruisco la tabella di parsing LALR in questo modo:

\subsection{La differenza pratica e concettuale tra SLR(1) e LR(1)}
E così abbiamo concluso che Eq.\ref{gr:pointers-grammar} non è una SLR(1), ma LR(1) invece sì. Vi va di indagare un attimino di più su quali siano le cause di ciò?
\paragraph*{}
Lo stato che genera un conflitto all'interno del automa caratteristico SLR(1) è quello identificato dal numero 2: poiché in quest'ultimo gli item di riduzione vengono trasformati in riduzioni effettive in presenza dei follow del driver delle produzioni non riusciamo ad avere un'espressività tale da discriminare le operazioni di shift e riduzione, per cui si crea un conflitto s/r. Utilizzando un parsing LR(1) nel medesimo stato avremo invece un'operazione di \texttt{shift 6} solo nel caso in cui leggessimo \{=\} dall'input buffer mentre invece \texttt{reduce \(R \to L\)} nel caso in cui il simbolo letto in input sia \{\$\}.
\paragraph*{}
Supponiamo a questo proposito di fare un'analisi di una stringa del tipo \(w_1 = w_2\) utilizzando l'automa caratteristico LR(1): visto che = è un simbolo generato solo se si sceglie una produzione del tipo \(S \to L = R\), una volta che abbiamo letto tutto ciò che fa a capo \(w_1\) dobbiamo accertarci di eseguire una riduzione ad \(L\). A questo punto, però, osservando la grammatica dovrebbe essere abbastanza chiaro che le uniche parole derivabili da \(L\) appartengono a \(\{*^n id \mid n \geq 0\}\) e che le uniche derivazioni di parole in \(\{*^n id \mid n > 0\}\) coinvolgono la \(R\) (i.e. se vogliamo una stringa in cui compaia almeno una volta il simbolo \(\ast\) è necessario utilizzare la produzione \(L \rightarrow *R\)).
\subparagraph*{}
Se la stringa che si sta analizzando è del tipo \(id = w_2\), allora ci spostiamo dallo stato 5 per poi effettuare riduzione \(L \to id\) e spostarci allo stato 2. Se invece ci stiamo concentrando su una stringa del tipo \(*w_1' = w_2\), allora ci sposteremo invece allo stato 4, dove continueremo a eseguire un ciclo fino a che non esauriremo gli \(*\); nel momento in cui leggeremo invece \(id\) ci sposteremo allo stato 5, eseguiremo riduzione \(L \to id\) e finiremo con lo spostarci allo stato 8 e \textbf{non} al 2. 
\paragraph*{}
Il parsing di tipo SLR(1) non riconosce dunque la differenza fra lo stato 2 e lo stato 8 a differenza del parsing LR(1), che ha un costo maggiore. Quindi, il problema del parsing SLR(1) (in questo caso) è che andiamo a piazzare un'operazione di riduzione per tutti i follow dei driver mentre in quello di tipo LR(1) lo si fa per sottoinsiemi dei follow dei driver.

\section{Il parsing LALR(1)}
Nonostante il parsing di tipo LR(1) sia il più completo è possibile che l'automa caratteristico risultante contenga delle ridondanze: nell'esempio precedente vi sono infatti delle sottostrutture che sono isomorfe e ciò deriva dal fatto che, visto che le transizioni dipendono sempre dalla prima componente (i.e. item LR(0)), è possibile notare che le ridondanze si verificano nel caso in cui gli stati abbiano la stessa proiezione LR(0).
\subparagraph*{}
Per mantenere la precisione e l'affidabilità del parsing LR(1) ma senza rinunciare all'efficienza di quello SLR(1) possiamo utilizzare una terza via, il parsing LALR(1): quest'ultimo utilizza gli stati in maniera simile al parsing SLR(1), ma utilizza una funzione di lookahead un filo più raffinata. Andiamo a conoscerla più da vicino.

\subsection{L'Automa caratteristico LRm(1)}

Un passaggio intermedio obbligatorio è l'automa LRm(1) \(\mathcal{AM}\), costruito a partire dall'automa LR(1) \(\mathcal{A}\) tramite raffinazione unendo più stati diversi in un singolo stato \emph{merged} (la lettera \(\mathcal{M}\) sta proprio per \emph{merged}).

Gli \textbf{Stati} del nuovo automa caratteristico sono ottenuti unendo all'interno di un singolo stato di \(\mathcal{AM}\) tutti gli item negli stati \(<P_1, ..., P_n>\) di \(\mathcal{A}\) che hanno le stesse LR(0)-proiezioni (i.e. con i medesimi item LR(0)).

\textbf{Transizioni}: Se lo stato \(P_1\) di \(\mathcal{A}\) presenta una \(Y\)-transizione a \(Q_1\) (stato di \(\mathcal{A}\)) e se \(P_1\) è stato inserito in uno stato merged \(<P_1, ..., P_n>\) di \(\mathcal{AM}\) e \(Q_1\) in \(<Q_1, ..., Q_m>\), allora c'è una \(Y\)-transizione in \(\mathcal{AM}\) da \(<P_1, ..., P_n>\) a \(<Q_1, ..., Q_m>\).

Se prendiamo in considerazione l'automa LR(1) definito precedentemente, possiamo osservare che gli stati 4 e 11 hanno la medesima prima proiezione e quindi possiamo unirli all'interno di un nuovo stato, che chiamiamo 4\&11 e corrisponde ad uno stato merged.

Cosa facciamo per le transizioni? Nell'automa LR(1) abbiamo un'unica transizione etichettata \(id\) che va dallo stato 4 allo stato 5 (e allo stesso modo un'altra etichettata \(id\) dallo stato 11 al 12). Visto che gli stati 4 e 11 sono stati uniti in un unico stato (4\&11) e lo stesso vale per 5 e 12 (5\&12), per la definizione precedente possiamo inserire in \(\mathcal{AM}\) una transizione da 4\&11 a 5\&12 etichettata \(id\).

Eseguendo l'operazione descritta la dimensione (i.e. il numero di stati) dell'automa LRm(1) così generato è sicuramente uguale a quella dell'automa LR(0): andando a combinare stati con con gli stessi item LR(0) ci ritroveremo con lo stesso numero di stati dell'automa LR(0) e con le stesse transizioni. 

\subsection{Costruzione di una tabella di parsing LALR(1)}
\label{subsec:lalr1-parsing-table}

Il prossimo passo per la costruzione dell'automa LALR(1) è ottenerne la tabella di parsing, per questo compito necessitiamo di:

\begin{itemize}
    \item l'automa caratteristico LRm(1)
    \item la lookahead function \(\mathcal{LA}(P, A \to \beta) = \cup_{[A \to \beta \cdot, \Delta_j]} \Delta_j\)
\end{itemize}

Nel caso in cui non vi sia più di uno stato con la stessa proiezione LR(0) allora tale stato non subisce l'operazione di unione e viene inserito direttamente nell'automa LRm(1) con il medesimo lookahead-set.

La grammatica \(\G\) è LALR(1) se e solo se la sua tabella di parsing LALR(1) non ha conflitti. 

Il vantaggio della grammatica LALR(1) è di essere più potente (i.e. espressiva) della grammatica SLR(1) pur presentando una tabella di dimensioni contenute rispetto alla rispettiva controparte LR(1). 

\subsection*{Esercizio parsing LALR(1)}
\begin{align*}
    S &\to AaB \mid b \\
    A &\to BcBaA \mid \varepsilon \\
    B &\to \varepsilon
\end{align*}

Il nostro obiettivo è quello di costruire la parsing table LALR(1) per la grammatica citata: per poter procedere dobbiamo però prima costruire l'automa caratteristico.
\paragraph{Ricaviamo l'automa}
\begin{enumerate}
    \item Inizializziamo lo stato 0; il suo kernel è 
    \begin{equation*}
        S' \to \cdot S, \{\$\}
    \end{equation*}
    Di questo kernel devo calcolare la chiusura:
    \begin{align*}
        S &\to \cdot AaB, \{\$\} \\
        S &\to \cdot b, \{\$\} \\
        A &\to \cdot BcBaA, \{a\} \\
        A &\to \cdot, \{a\} \\
        B &\to \cdot, \{c\}
    \end{align*}
    Essendo questi gli item per lo stato 0, possiamo identificare quattro possibili transizioni (e quindi quattro possibili nuovi stati): \(\tau(0,S)=1 \textrm{, } \tau(0,A)=2 \textrm{, } \tau(0,b)=3 \textrm{ e } \tau(0,B)=4\).
    
    Interessante notare che le produzioni del tipo \(A \to \varepsilon\) vengono convertite in reducing item \(A \to \cdot\).
    \item Analizziamo ora lo stato 1; il suo kernel è 
    \begin{equation*}
        S' \to S \cdot, \{\$\}    
    \end{equation*}
    Non serve calcolare la chiusura di questo item in quanto è formata solamente da sé stesso, possiamo però evidenziare il fatto che questo è lo stato contenente l'\textbf{Accepting Item}.
    \item Analizziamo dunque lo stato 2, il cui kernel è composto solamente da:
    \begin{equation*}
        S \to  A \cdot aB, \{\$\}
    \end{equation*}
    Nemmeno in questo caso è necessario calcolare la sua chiusura in quanto il marker si trova davanti ad un non terminale: aggiungiamo dunque la transizione \(\tau(2,a)=5 \) e proseguiamo.
    \item Il kernel dello stato 3 è dato da
    \begin{equation*}
        S \to b \cdot, \{\$\}
    \end{equation*}
    Il che vuol dire che non è possibile calcolare la chiusura e che lo stato 3 contiene un reducing item.
    \item Passiamo allo stato 4, il cui kernel è:
    \begin{equation*}
        A \to B \cdot cBaA, \{a\} 
    \end{equation*}
   Per gli stessi motivi dello stato 2 la chiusura di tale stato non porta nuovi elementi; è comunque possibile definire la transizione \(\tau(4,c)\) allo stato 6
    \item Il kernel dello stato 5 è:
    \begin{equation*}
        S \to  Aa \cdot B, \{\$\}
    \end{equation*}
    La cui chiusura risulta essere pari a 
    \begin{align*}
        B &\to \cdot, \{\$\}
    \end{align*}
    Per questo motivo sappiamo che lo stato contiene un reducing item e possiede una transizione \(\tau(5, B)\) allo stato 7
    \item Procedendo come abbiamo fatto fino ad ora il kernel per lo stato 6 è
    \begin{equation*}
        A \to Bc \cdot BaA, \{a\} 
    \end{equation*}
    la cui chiusura corrisponde a 
    \begin{equation*}
         B \to \cdot, \{a\} 
    \end{equation*}
    Come è intuibile lo stato 6 contiene un reducing item e la sua transizione è \(\tau(6, B)=8\)
    \item Il kernel dello stato 7 è 
    \begin{equation*}
         S \to  AaB \cdot, \{\$\}  
    \end{equation*}
    Essendo che il marker è in fondo alla produzione non è possibile effettuare la chiusura ma solo considerare che lo stato 7 contiene un reducing item
    \item Lo stato 8 ha kernel
    \begin{equation*}
        A \to BcB \cdot aA, \{a\} 
    \end{equation*}
    e possiede una transizione \(\tau(8, a)\) allo stato 9
    \item Lo stato 9 ha kernel
    \begin{equation*}
        A \to BcBa \cdot A, \{a\} 
    \end{equation*}
    di cui possiamo calcolare la chiusura ottenendo
    \begin{align*}
        A &\to \cdot BcBaA, \{a\} \\
        A &\to \cdot, \{a\} \\
        B &\to \cdot, \{c\}
    \end{align*}
    Che contiene due reducing item e possiede due transizioni: la prima è \(\tau(9, A)=10\) e ha come target un nuovo stato mentre la seconda è \(\tau(9, B)=4\) che ha come target uno stato che fa già parte del nostro automa caratteristico.
    \item Lo stato 10 infine ha kernel
    \begin{equation*}
        A \to BcBaA \cdot, \{a\} 
    \end{equation*}
    Visto che il marker è posto alla fine non è possibile calcolare la chiusura di questo stato e possiamo concludere aggiungendo che lo stato 10 ha un reducing item. 
\end{enumerate}

L'automa caratteristico LR(1) risulta dunque così costruito

\begin{figure}[H]
	\centering
	\subimport{assets/figures/}{automa_LR-LALR.tex}
    \caption{Automa LR(1)/LALR(1)}
    \label{fig:lalr-automata}
\end{figure}

Visto che non è possibile unire nessuno degli stati dell'automa LR(1), l'automa disegnato poc'anzi corrisponderà anche a quello LRm(1). Di seguito dunque possiamo costruire la tabella di parsing:

\begin{enumerate}
    \item \(S \to AaB\) 
    \item \(S \to b\) 
    \item \(A \to BcBaA\) 
    \item \(A \to \varepsilon\) 
    \item \(B \to \varepsilon\)
\end{enumerate}

\begin{table}[H]
    \centering
    \subimport{assets/tables/}{lr-lalr-parsing-table.tex}
    \caption{LR(1) \& LALR(1) Parsing Table}
    \label{tab:lr-lalr-parsing-table}
\end{table}


\subsection{L'automa simbolico}
La classe di grammatiche LALR contiene le grammatiche più interessanti per i linguaggi di programmazione.
Abbiamo visto come la modalità più semplice per ottenere le tabelle di parsing LALR è quella di creare in primis un'automa LR(1) e poi tradurlo in LRm(1).

Oggi vediamo un altro metodo che prevede di eliminare il passaggio dall'automa LR(1) e di creare fin da subito un automa con la stessa struttura dell'automa LRm(1).
Quello che andremo a creare è detto automa simbolico, perché utilizza dei simboli (delle variabili) al posto dei lookahead set; una volta terminata la costruzione dell'automa simbolico risolvendo i simboli si va a calcolare quelli che poi saranno i lookahead set dell'automa LRm(1).

Quindi gli item di questo automa simbolico possono essere immaginati cme suddivisi in due componenti:
\begin{itemize}
    \item una componente di tipo LR(0);
    \item una parte composta da un insieme di \emph{lookahead simbolico}.
\end{itemize}
Sostanzialmente questi item sono del tutto assimilabili agli item LR(1), con l'unica differenza che il \(\Delta\) set è simbolico, ovvero è una variabile che viene calcolata solo al termine dell'analisi degli stati. Presto al lettore sarà ben chiaro questo concetto.

Mentre creiamo l'automa simbolico ci memorizziamo in una tabella delle equazioni che ci serviranno alla fine per tradurre i lookahead set simbolici in lookahead effettivi. Come è ormai consuetudine il modo più semplice per spiegare questo procedimento è applicarlo direttamente ad un esempio.

\subsection*{Esempio di costruzione dell'automa simbolico}
Andiamo a costruire l'automa simbolico per la nostra cara grammatica
\begin{align*}
    S &\to L = R \mid R \\
    L &\to *R \mid id \\
    R &\to L
\end{align*}
La costruzione dell'automa simbolico verrà affiancata dalla costruzione dell'automa LR(1) per rendere più chiare le differenze tra queste due costruzioni.

Il primo passo è installare lo stato 0 dell'automa:
\begin{figure}[h!]
    \centering
    \includegraphics[width=.7\textwidth]{ex_automa_simbolico-stato_0.png}
    \caption{Stato 0: sulla sinistra automa simbolico, sulla destra automa LR(1)}
    \label{img:ex_automa_simbolico-stato_0}
\end{figure}
È subito chiaro cosa intendevamo dicendo "invece di inserire il lookahead set inseriamo una \emph{variabile}", di fatto come lookahead inseriamo un simbolo, che risolveremo solo alla fine della costruzione dell'automa.

In questo caso al posto di inserire \(\$\) inseriamo una variabile chiamata \(x_0\).
La corrispondenza tra \(x_0\) e \(\$\) la salviamo (ce la scriviamo) nel sitema di equazioni che ci porteremo dietro per tutta la costruzione.
\begin{align*}
    Sistema:& \\
            & x_0 = \{\$\}
\end{align*}

Quando andiamo a fare la chiusura, essendo gli item simbolici in tutto e per tutto simili agli item LR(1), utilizziamo la \(closure_1\).
Questa volta però abbiamo come lookahead set le variabili, in questo caso il nostro lookahead set è \(x_0\).

Procedendo un passettino alla volta la chiusura dello stato 0 si fa così:
\begin{itemize}
    \item in primis dobbiamo includere la chiusura per le produzioni di S:
    \begin{itemize}
        \item dalle due produzioni di \(S\) della nostra grammatica otteniamo i due item:
        \begin{align*}
            S &\to \cdot L = R,  \; \{x_0\} \\
            S &\to \cdot R,  \; \{x_0\}
        \end{align*}
        in questo caso il lookahead set è facile da calcolare perché \(\beta\) di
        \(S' \to \cdot S\) è \(\varepsilon\) mentre \(\Delta = x_0\);
    \end{itemize}
    \item ora abbiamo introdotto anche un marker davanti alla \(L\), quindi dobbiamo chiudere anche per \(L\):
    \begin{itemize}
        \item dalle produzioni di \(L\) otteniamo i due item:
        \begin{align*}
            L &\to \cdot *R, \; \{=\} \\
            L &\to \cdot id, \; \{=\}
        \end{align*}
        in questo caso, dato che \(\beta \neq \varepsilon\), vediamo come il lookahead set corrisponda ad un carattere e non una variabile, questo è possibile mentre stiamo calcolando la chiusura di item simbolici;
    \end{itemize}
    \item infine, avendo introdotto anche \(R\), dobbiamo calcolare la chiusura indotta dalle produzioni di \(R\):
    \begin{itemize}
        \item abbiamo solo una produzione per \(R\), che ci porta ad inserire l'item 
        \begin{equation*}
            R \to \cdot L, \; \{x_0\}
        \end{equation*}
        è andato tutto liscio come l'olio no?
    \end{itemize}
    \item invece no! perché abbiamo introdotto una nuova chiusura per \(L\), che è diversa dalla precedente perché ora il lookahead set della chiusura è \(x_0\) invece che \(=\), quindi quello che dobbiamo fare è ricalcolare le chiusure di \(L\) con questo nuovo lookahead; dato che siamo skillati e sappiamo già come andrà a finire aggiungiamo semplicemente \(\{x_0\}\) al lookahead set delle produzioni di \(L\) che abbiamo già elencato poco fa, quindi:
    \begin{align*}
        L &\to \cdot *R, \; \{=, x_0\} \\
        L &\to \cdot id, \; \{=, x_0\}
    \end{align*}
\end{itemize}
Il consiglio per chi non ha capito questi passaggi è quello di fare qualche esercizio di costruzione di automi LR(1), fidatevi ne vale la pena!
Per tutti gli altri invece si apre ora una strada tortuosa verso una valle di lacrime, ma non indugiamo oltre, l'esame è vicino e non c'è tempo da perdere.

Orbene, passiamo ad analizzare le transazioni uscenti dallo stato 0.
La prima transizione che troviamo è \(\tau(0,S)\) che ci porta ad un nuovo stato, lo stato 1.
\\
Kernel dello stato 1:
\begin{figure}[h!]
    \centering
    \includegraphics[width=.7\textwidth]{ex_automa_simbolic-kernel_s1.png}
    \caption{Kernel stato 1: sulla sinistra automa simbolico, sulla destra automa LR(1)}
\end{figure}
notiamo come invece che scrivere il lookahead set come insieme creiamo una nuova variabile \(x_1\) per indicarlo: questo stratagemma ci serve per aggiornare il valore del lookahead set nel caso dovessimo tornare in questo stato con un nuovo \(\Delta\), sarà chiaro in seguito. Dobbiamo aggiornare anche il nostro sistema in cui salviamo il valore effettivo della variabile.
In questo caso il lookahed set del kernel, essendo \(\beta = \varepsilon\) è proprio uguale al lookahead set della produzione da cui arriviamo, quindi \(x_0\), aggiorniamo dunque alacremente il nostro sistema di equazioni.
\begin{align*}
    Sistema:& \\
            & x_0 = \{\$\} \\
            & x_1 = x_0
\end{align*}
Lo stato 1 non ha chiusure da calcolare e nemmeno transizioni, quindi passiamo oltre.

La prossima transizione che andiamo ad analizzare è \(\tau(0,L)\) che ci porta in un nuovo stato, diciamo 2.
\\
Kernel dello stato 2:
\begin{figure}[h!]
    \centering
    \includegraphics[width=.7\textwidth]{ex_automa_simbolico-kernel_s2.png}
    \caption{Kernel stato 2: sulla sinistra automa simbolico, sulla destra automa LR(1)}
\end{figure}
in questo caso il kernel contiene due item perché nello stato 0 abbiamo due item che presentano la possibilità di una \(L\)-transizione; ognuno di questi item quindi avrà il suo lookahead set simbolico: andiamo ad aggiornare subito il nostro sistema, sapendo che le produzioni che ci portano allo stato due hanno come lookahead set \(x_0\).
\begin{align*}
    Sistema:& \\
            & x_0 = \{\$\} \\
            & x_1 = x_0 \\
            & x_2 = x_0 \\
            & x_3 = x_0 \\
\end{align*}
grazie al cielo non si presentano chiusure; analizzeremo le transizioni in seguito.

Passiamo quindi all'osservazione della transizione \(\tau(0,R)\), che ci porta nello stato 3.
\\
Kernel dello stato 3:
\begin{figure}[h!]
    \centering
    \includegraphics[width=.7\textwidth]{ex_automa_simbolico-kernel_s3.png}
    \caption{Kernel stato 3: sulla sinistra automa simbolico, sulla destra automa LR(1)}
\end{figure}
anche in questo caso abbiamo che \(\beta = \varepsilon\) quindi il lookahead set del kernel è uguale a \(\Delta\), riportiamo subito questa informazione all'interno del sistema.
\begin{align*}
    Sistema:& \\
            & x_0 = \{\$\} \\
            & x_1 = x_0 \\
            & x_2 = x_0 \\
            & x_3 = x_0 \\
            & x_4 = x_0 \\
\end{align*}
il kernel dello stato 3 non presenta nè chiusure nè transizioni, quindi procediamo oltre senza esitazioni.

La prossima transizione che dobbiamo analizzare è \(\tau(0,*)\), questa deriva dalla produzione \([L \to \cdot *R, \; \{=, x_0\}]\).
\\
Kernel dello stato 4:
\begin{figure}[h!]
    \centering
    \includegraphics[width=.7\textwidth]{ex_automa_simbolico-kernel_s4.png}
    \caption{Kernel stato 4: sulla sinistra automa simbolico, sulla destra automa LR(1)}
\end{figure}
anche in questo caso  \(\beta = \varepsilon\) quindi il lookahead set del kernel è uguale a \(\Delta\), aggiorno quindi il sistema di equazioni.
\begin{align*}
    Sistema:& \\
            & x_0 = \{\$\} \\
            & x_1 = x_0 \\
            & x_2 = x_0 \\
            & x_3 = x_0 \\
            & x_4 = x_0 \\
            & x_5 = \{=, \$\}
\end{align*}
finalmente uno stato che ci dia la \emph{soddisfazione} di calcolare delle chiusure, grazie alle produzioni di \(R\) in primis e di \(L\) poi otteniamo la seguente chiusura per gli item dello stato 4:
\begin{align*}
    R &\to \cdot L, \; \{x_5\} \\
    L &\to \cdot *R, \; \{x_5\} \\
    L &\to \cdot id, \; \{x_5\}
\end{align*}
che spettacolo, eh? passiamo oltre.

Prossima transizione da analizzare: \(\tau(0,id)\) che deriva dall'item \([L \to \cdot id, \; \{=, x_0\}]\) e ci porta nello stato 5.
\\
Kernel dello stato 5:
\begin{figure}[h!]
    \centering
    \includegraphics[width=.7\textwidth]{ex_automa_simbolico-kernel_s5.png}
    \caption{Kernel stato 5: sulla sinistra automa simbolico, sulla destra automa LR(1)}
\end{figure}
anche in questo caso  \(\beta = \varepsilon\) quindi il lookahead set del kernel è uguale a \(\Delta\), aggiorno quindi il sistema di equazioni.
\begin{align*}
    Sistema:& \\
            & x_0 = \{\$\} \\
            & x_1 = x_0 \\
            & x_2 = x_0 \\
            & x_3 = x_0 \\
            & x_4 = x_0 \\
            & x_5 = \{=, \$\} \\
            & x_6 = \{=, \$\} \\
\end{align*}
Anche qui, pace all'anima nostra, inseriamo il lookahead set come variabile aggiuntiva e la valorizziamo nel nostro sistema di equazioni.

Abbiamo finito le transizioni dallo stato 0! Prossima transizione da analizzare \(\tau(2, =)\), che deriva dall'item \([S \to L \cdot = R, \; \{x_2\}]\), ci porta direttamente allo stato 6.
\\
Kernel dello stato 6:
\begin{figure}[h!]
    \centering
    \includegraphics[width=.7\textwidth]{ex_automa_simbolico-kernel_s6.png}
    \caption{Kernel stato 6: sulla sinistra automa simbolico, sulla destra automa LR(1)}
\end{figure}
in questo caso  \(\beta = \varepsilon\) quindi il lookahead set del kernel è uguale a \(\Delta\), aggiorno quindi il sistema di equazioni.
\begin{align*}
    Sistema:& \\
            & x_0 = \{\$\} \\
            & x_1 = x_0 \\
            & x_2 = x_0 \\
            & x_3 = x_0 \\
            & x_4 = x_0 \\
            & x_5 = \{=, \$\} \\
            & x_6 = \{=, \$\} \\
            & x_7 = x_2 \\
\end{align*}
Ora passiamo ad analizzare le chiusure degli item di questo stato.
Grazie alla presenza del marker davanti ad una \(R\) aggiungiamo questi item allo stato:
\begin{align*}
    R &\to \cdot L, \; \{x_7\} \\
    L &\to \cdot *R, \; \{x_7\} \\
    L &\to \cdot id, \; \{x_7\}
\end{align*}

Passiamo ora ad analizzare la transizione \(\tau(4, R)\), che deriva dall'item \([L \to * \cdot R, \; \{x_5\}]\) e ci spara nello stato 7.
\\
Kernel dello stato 7:
\begin{figure}[h!]
    \centering
    \includegraphics[width=.7\textwidth]{ex_automa_simbolico-kernel_s7.png}
    \caption{Kernel stato 7: sulla sinistra automa simbolico, sulla destra automa LR(1)}
\end{figure}
in questo caso  \(\beta = \varepsilon\) quindi il lookahead set del kernel è uguale a \(\Delta\), aggiorno quindi il sistema di equazioni.
\begin{align*}
    Sistema:& \\
            & x_0 = \{\$\} \\
            & x_1 = x_0 \\
            & x_2 = x_0 \\
            & x_3 = x_0 \\
            & x_4 = x_0 \\
            & x_5 = \{=, \$\} \\
            & x_6 = \{=, \$\} \\
            & x_7 = x_2 \\
            & x_8 = x_5 \\
\end{align*}
a questo punto dovremmo fare la chiusura no? ma è già tutto chiuso, come alle 23:30 di sera a Trento...

Passiamo quindi ad analizzare la transizione \(\tau(4, L)\) che prende le mosse dall'item \([R \to \cdot L, \; \{x_5\}]\) e ci porta nello stato 8.
\\
Kernel dello stato 8:
\begin{figure}[h!]
    \centering
    \includegraphics[width=.7\textwidth]{ex_automa_simbolico-kernel_s8.png}
    \caption{Kernel stato 8: sulla sinistra automa simbolico, sulla destra automa LR(1)}
\end{figure}
in questo caso il lookahead set del kernel è uguale a \(\Delta\), aggiorno quindi il sistema di equazioni.
\begin{align*}
    Sistema:& \\
            & x_0 = \{\$\} \\
            & x_1 = x_0 \\
            & x_2 = x_0 \\
            & x_3 = x_0 \\
            & x_4 = x_0 \\
            & x_5 = \{=, \$\} \\
            & x_6 = \{=, \$\} \\
            & x_7 = x_2 \\
            & x_8 = x_5 \\
            & x_9 = x_5 \\
\end{align*}
anche qui niente chiusure.

Passiamo ad analizzare la transizione \(\tau(4,*)\), che deriva dall'item \([L \to \cdot * R, \{x_5\}]\); ora, se osserviamo bene il kernel dello stato in cui arriveremmo tramite questa transizione ci rediamo conto che è un kernel che abbiamo già incontrato, nello specifico proprio nello stato 4 da cui stiamo partendo, cosa significa tutto ciò?

Significa semplicemente che c'è un arco uscente da 4 che ci riporta in 4 (self loop in gergo) tramite una \(*\)-transizione. È qui che vediamo finalmente la grande differenza di costruzione tra automi LR(1) ed automi simbolici: in un automa LR(1) avremmo costruito un nuovo stato (perché il lookahead set in questo caso è diverso dal lookahead set dello stato 4), invece ora andiamo semplicemente ad aggiornare il lookahead set dello stato 4 aggiungendo il lookahead set della transizione  \(\tau(4,*)\) (che fatalità corrisponde proprio a \(x_5\)).
\begin{align*}
    Sistema:& \\
            & x_0 = \{\$\} \\
            & x_1 = x_0 \\
            & x_2 = x_0 \\
            & x_3 = x_0 \\
            & x_4 = x_0 \\
            & x_5 = \{=, \$\} \cup \{x_5\} \\
            & x_6 = \{=, \$\} \\
            & x_7 = x_2 \\
            & x_8 = x_5 \\
            & x_9 = x_5 \\
\end{align*}

Passiamo ora ad analizzare la transizione \(\tau(4, id)\), che deriva dall'item \([L \to \cdot id, \; \{x_5\}]\), colpo di scena! anche questo kernel ci è noto, corrisponde a quello dello stato 5, quindi andiamo ad aggiornare il lookahead set dello stato 5.
Il lookahead set della transizione \(\tau(4, id)\) è ancora una volta \(x_5\).
\begin{align*}
    Sistema:& \\
            & x_0 = \{\$\} \\
            & x_1 = x_0 \\
            & x_2 = x_0 \\
            & x_3 = x_0 \\
            & x_4 = x_0 \\
            & x_5 = \{=, \$\} \cup \{x_5\} \\
            & x_6 = \{=, \$\} \cup \{x_5\} \\
            & x_7 = x_2 \\
            & x_8 = x_5 \\
            & x_9 = x_5 \\
\end{align*}

Passiamo oltre, la prossima transizione su cui ci concentriamo è \(\tau(6, R)\), derivante da \([S \to L = \cdot R, \; \{x_7\}]\), che ci porta a scoprire lo stato 9.
\\
Kernel dello stato 9:
\begin{figure}[h!]
    \centering
    \includegraphics[width=.7\textwidth]{ex_automa_simbolico-kernel_s9.png}
    \caption{Kernel stato 9: sulla sinistra automa simbolico, sulla destra automa LR(1)}
\end{figure}
in questo caso  \(\beta = \varepsilon\) quindi il lookahead set del kernel è uguale a \(\Delta\), ovvero \(x_{10} = \{x_7\}\), aggiorno quindi il sistema di equazioni.
\begin{align*}
    Sistema:& \\
            & x_0 = \{\$\} \\
            & x_1 = x_0 \\
            & x_2 = x_0 \\
            & x_3 = x_0 \\
            & x_4 = x_0 \\
            & x_5 = \{=, \$\} \cup \{x_5\} \\
            & x_6 = \{=, \$\} \cup \{x_5\} \\
            & x_7 = x_2 \\
            & x_8 = x_5 \\
            & x_9 = x_5 \\
            & x_{10} = x_7 \\
\end{align*}
anche questo kernel ci regala poche emozioni (e nessuna chiusura) quindi concludiamo l'analisi dello stato 9.

La prossima transizione è \(\tau(6, L)\) che deriva da \([R \to \cdot L, \{x_7\}]\); tale transizione ci porterebbe in uno stato con kernel LR(0) = \([R \to L \cdot ]\), che quindi corrisponde al nostro stato 8: invece che creare un nuovo stato andiamo ad aggiornare il lookahead set dello stato 8 con il lookahead set che deriva dalla transizione \(\tau(6, L)\).
\begin{align*}
    Sistema:& \\
            & x_0 = \{\$\} \\
            & x_1 = x_0 \\
            & x_2 = x_0 \\
            & x_3 = x_0 \\
            & x_4 = x_0 \\
            & x_5 = \{=, \$\} \cup \{x_5\} \\
            & x_6 = \{=, \$\} \cup \{x_5\} \\
            & x_7 = x_2 \\
            & x_8 = x_5 \\
            & x_9 = x_5 \cup \{x_7\} \\
            & x_{10} = x_7 \\
\end{align*}
niente chiusure, passiamo oltre.

Prossima transizione \(\tau(6, *)\), derivante dall'item \([L \to \cdot * R, \{x_7\}]\), il target di questa transizione è uno stato con kernel LR(0) \([L \to *\cdot R]\), ovvero lo stato 4. Quindi anche questa volta non creiamo un nuovo stato ma andiamo ad aggiornare il kernel dello stato 4.
\begin{align*}
    Sistema:& \\
            & x_0 = \{\$\} \\
            & x_1 = x_0 \\
            & x_2 = x_0 \\
            & x_3 = x_0 \\
            & x_4 = x_0 \\
            & x_5 = \{=, \$\} \cup \{x_5\} \cup \{x_7\} \\
            & x_6 = \{=, \$\} \cup \{x_5\} \\
            & x_7 = x_2 \\
            & x_8 = x_5 \\
            & x_9 = x_5 \cup \{x_7\} \\
            & x_{10} = x_7 \\
\end{align*}

Passiamo alla prossima transizione che è \(\tau(6, id)\), derivante da \([L \to \cdot id, \{x_7\}]\), che ci porta in uno stato con kernel LR(0) = \([L \to id \cdot]\), che corrisponde al già visitato stato 5, quindi aggiorniamo semplicemente il lookahead set di tale stato.
\begin{align*}
    Sistema:& \\
            & x_0 = \{\$\} \\
            & x_1 = x_0 \\
            & x_2 = x_0 \\
            & x_3 = x_0 \\
            & x_4 = x_0 \\
            & x_5 = \{=, \$\} \cup \{x_5\} \cup \{x_7\} \\
            & x_6 = \{=, \$\} \cup \{x_5\} \cup \{x_7\} \\
            & x_7 = x_2 \\
            & x_8 = x_5 \\
            & x_9 = x_5 \cup \{x_7\} \\
            & x_{10} = x_7 \\
\end{align*}

Oh butei, abbiamo finito di analizzare l'automa, quindi andiamo a risolvere lo strascico di equazioni che ci siamo tirati dietro per tutto l'esercizio:
\begin{align*}
    x_0, x_1, x_2, x_3, x_4, x_7, x_{10} &= \{\$\} \\
    x_5, x_6, x_8, x_9 &= \{=, \$\}
\end{align*}
Questa risoluzione è fatta a occhio, ma esiste un metodo algoritmico per risolvere il sistema in modo molto efficiente.
L'algoritmo in questione utilizza le classi di equivalenza, inaspettatamente.

Complessivamente cosa abbiamo ottenuto?
Un automa con le stesse dimensioni di tipo LR(0), poichè abbiamo creato esattamente gli stessi stati che avremmo cerato costruendo un automa LR(0), ma abbiamo utilizzato delle chiusure di tipo LR(1), ricavando quindi dei lookahead set raffinati.

Per concludere questo esempio ricordiamo brevemente quali sarebbero i passi da seguire a questo punto per ottenere l'automa merged:
\begin{enumerate}
    \item inseriamo tutti gli stati;
    \item le mosse di shift (le transizioni) sono esattamente quelle identificate e corrispondono alle transizioni dell'automa caratteristico LR(0) per questa grammatica;
    \item gli stati con riduzioni sono tutti quelli che presentano una produzione con il marker in fondo al body;
    \item le lookahead function (le etichette per le riduizioni) corrispondono ai lookahead set calcolati nella risoluzione del sistema.
\end{enumerate}
Carino questo esempio, eh? per festeggiare il suo completamento ci concediamo la risoluzione di un altro esercizio dello stesso tipo.

\subsection{Esercizio di costruzione dell'automa simbolico}
La grammatica di cui vogliamo ricavare l'automa simbolico questa volta è la nostra cara
\begin{equation}
    S \to aSb \mid ab
\end{equation}
\subsubsection*{Stato 0}
Kernel dello stato 0:
\begin{align*}
    S' \to \cdot S, x_0
\end{align*}
Ci salviamo il valore di \(x_0\).
\begin{align*}
    Sistema:& \\
            & x_0 = \{\$\}
\end{align*} 
Passiamo alla chiusura del kernel:
\begin{align*}
    S &\to \cdot aSb, x_0 \\
    S &\to \cdot ab, x_0
\end{align*}
Lo stato 0 non ci regala altre chiusure, ma ci regala due belle transizioni, una per \(a\) ed una per \(S\).
\\\\
Analizziamo quindi la transizione \(\tau(0,S)\), che ci porta nel nuovo stato 1.
\subsubsection*{Stato 1}
Kernel dello stato 1:
\begin{align*}
    S' \to S \cdot, x_1
\end{align*}
In questo caso \(\beta = \varepsilon\), quindi il lookahead set del kernel sarà uguale al lookahead set della produaizone che ci ha portati in 1, ci salviamo questo risultato.
\begin{align*}
    Sistema:& \\
            & x_0 = \{\$\}\\
            & x_1 = x_0
\end{align*}
Questo stato non ci offre altri divertimenti, passiamo oltre.
\\\\
Analizziamo la transizione \(\tau(0,a)\), che ci porta nel nuovo stato 2.
\subsubsection*{Stato 2}
Da \(\tau(0, a)\) che viene da \([S \to \cdot aSb, x_0], [S \to \cdot ab, x_0]\) abbiamo un kernel non ancora visitato, il nuovo stato presenta quindi questo kernel:
\begin{align*}
    S &\to  a \cdot S b, x_2 \\
    S &\to  a \cdot b, x_3
\end{align*}
In questo caso il lookahead set di entrambe le produzioni del kernel sarà uguale al lookahead della produzione che ci ha portato in 2, poiché \(\beta = \varepsilon\) in entrambi i casi.
\begin{align*}
    Sistema:& \\
            & x_0 = \{\$\}\\
            & x_1 = x_0 \\
            & x_2 = x_0 \\
            & x_3 = x_0
\end{align*}
In questo caso il primo item del kernel richiede che venga calcolata una chiusura:
\begin{align*}
    S &\to  \cdot a S b, \{b\} \\
    S &\to  \cdot a b, \{b\}
\end{align*}
Ma questa volta dobbiamo prestare attenzione perché la chiusura per \(S\) ha come \(\beta\) il carattere \(b\), quindi il lookahead set dei due item appena  ottenuti è \(\{b\}\).
Ora che abbiamo calcolato la chiusura del kernel possiamo passare all'analisi delle transizioni che ci regala, ovvero \(\tau(2, S)\), \(\tau(2, a)\) e \(\tau(2, b)\).
\\\\
Analizziamo quindi \(\tau(2,S)\), che ci porta allo stato 3 tramite l'item \([S \to a \cdot S b, x_2]\).
\subsubsection*{Stato 3}
Da \(\tau(2, S)\) che viene da \([S \to a \cdot Sb, x_2]\) abbiamo un kernel non ancora visitato:
\begin{align*}
    S &\to  a S \cdot b, x_4
\end{align*}
Orbene, generiamo un nuovo stato per questo indomito kernel, questo non ci offre nessuna chiusura, aggiorniamo quindi pacificamente il nostro sistema.
\begin{align*}
    Sistema:& \\
            & x_0 = \{\$\}\\
            & x_1 = x_0 \\
            & x_2 = x_0 \\
            & x_3 = x_0 \\
            & x_4 = x_2
\end{align*}
Lo stato 3 non ci offre ulteriori motivi di svago se non una veloce transizione \(\tau(3,b)\).
\subsubsection*{Aggiornamento lookahead set stato 2}
Da \(\tau(2,a)\) avremmo un kernel con questa forma:
\begin{align*}
    [S &\to a \cdot S b, x_5], x_5=\{b\} \\
    [S &\to a \cdot b, x_6], x_6=\{b\}
\end{align*}
Ma la componente LR(0) (cioè il primo elemento della coppia) del kernel è uguale a quella dello stato 2 che abbiamo già incontrato, quindi la procedura ci dice di non creare un nuovo stato ma aggiornare i \(\Delta\)-set dello stato 2.
\begin{align*}
    Sistema:& \\
            & x_0 = \{\$\}\\
            & x_1 = x_0 \\
            & x_2 = x_0 \cup \{b\} \\
            & x_3 = x_0 \cup \{b\} \\
            & x_4 = x_2 \\
\end{align*}
\subsubsection*{Stato 4}
La prossima produzione che analizziamo è  \(\tau(2,b)\), che ci capitombola nel nuovo stato 4 tramite l'item \([S \to a \cdot b, \{x_3\}]\).
\\
Kernel dello stato 4:
\begin{align*}
    S \to  a b \cdot, x_5
\end{align*}
Questo è un reducing item.
Aggiorniamo il sistema:
\begin{align*}
    Sistema:& \\
            & x_0 = \{\$\}\\
            & x_1 = x_0 \\
            & x_2 = x_0 \\
            & x_3 = x_0 \\
            & x_4 = x_2 \\
            & x_5 = x_3 
\end{align*}
Lo stato è chiuso.
\subsubsection*{Stato 5}
Ci resta da analizzare la transizione \(\tau(3,b)\) che tramite l'item \([S \to a S \cdot b, \{x_4\}]\) ci porta al nuovo stato 5.
\\
Kernel dello stato 5:
\begin{align*}
    S &\to a S b \cdot, x_6
\end{align*}
Anche questo giro incontriamo un reducing item, aggiorniamo il sistema:
\begin{align*}
    Sistema:& \\
            & x_0 = \{\$\}\\
            & x_1 = x_0 \\
            & x_2 = x_0 \cup \{b\} \\
            & x_3 = x_0 \cup \{b\} \\
            & x_4 = x_2 \\
            & x_5 = x_3 \\
            & x_6 = x_4
\end{align*}
Questo stato è chiuso, quindi terminiamo la sua analisi.

Avendo analizzato tutte le transizioni non ci rimane altro da fare se non risolvere il sistema di equazioni per valorizzare le nostre incognite \(x\), ovvero i \(\Delta\)-set effettivi.
\begin{align*}
    x_0 = x_1 &= \{\$\} \\
    x_2 = x_3 = x_4 = x_5 = x_6 &= \{\$, b\}
\end{align*}
A questo punto sarà sufficiente sostituire i \(\Delta\)-set ai simboli \(x_i\) per vedere gli stati che compongono l'automa merged nella loro forma effettiva.

Andiamo quindi, per curiosità, ad analizzare gli stati in cui sono presenti le riduzioni ed a piazzarvi i lookahead set:
\begin{itemize}
    \item in 1 abbiamo la riduzione \(S' \to S \cdot, \; \{\$\}\);
    \item in 4 abbiamo \(S \to a b \cdot, \; \{\$, b\}\);
    \item in 5 abbiamo \(S \to aSb \cdot, \; \{\$, b\}\).
\end{itemize}
Notiamo quindi che non si presentano conflitti, quindi la grammatica è di tipo LALR.

% \section{Un primo esempio di applicazione}
% \subsection{Mosse di shift e reduce}
% Andiamo a introdurre l'algoritmo che utilizzeremo per verificare se una certa parola appartenga o meno al linguaggio denotato da una certa grammatica, rappresentata dal suo automa caratteristico; questo è detto algoritmo di shift/reduce, dal nome delle due mosse che andremo a utilizzare. Come prima cosa, prendiamo familiriatà con le due pile che utilizzeremo nella procedura:
% \begin{itemize}
%     \item nella prima inseriamo gli stati verso cui ci muoviamo;
%     \item nella seconda conserviamo la derivazione parziale a cui siamo arrivati sinora.
% \end{itemize}
% Si tenga presente che in realtà potremmo farci bastare anche una sola pila, ma andrebbe a complicare sensibilmente la gestione della procedura.

% Adesso che conosciamo le strutture necessarie alla procedura, andiamo a vedere le due mosse sopra menzionate:
% \begin{enumerate}
%     \item la mossa di \emph{shift} è quella che compiamo quando passiamo da un nodo (stato) all'altro, inserendo nella pila delle derivazioni parziali il terminale che marca l'arco attraversato e nella pila degli stati il nodo di destinazione;
%     \item la mossa di \emph{reduce} è quella che eseguiamo quando raggiungiamo un nodo etichettato da una formula di riduzione (capiremo nell'esempio quale forma hanno) e che ci porta a eliminare dei terminali dalla pila delle derivazioni parziali e degli stati dalla pila degli stati, coerentemente alla struttura dell'automa caratteristico.
% \end{enumerate}
% Consideriamo come esempio una delle prime grammatiche che abbiamo visto, quella che genera due occorrenze bilanciate:
% \begin{equation}
%     \label{balanced}
%     \G: S \to aSb \mid ab
% \end{equation}
% \subsection{Esempio di automa}
% L'automa caratterisco di tipo LR(1) per questa grammatica è il seguente:
% % \newgeometry{left = 1.7cm, right=1.7cm}
% \begin{figure}[H]
%     \centering
% 	\subimport{assets/figures/}{automa_LR_8-1.tex}
%     \caption{Automa caratteristico LR(1) per Eq. \ref{balanced}}
%     \label{balanced-char_aut-lr1}
% \end{figure}
% % \restoregeometry
% Lo utilizzeremo come guida per determinare, di volta in volta, quali mosse di shift e reduce applicare per verificare se una certa parola appartiene o no al linguaggio generato da \(\G\). 

% \subsection{Procedura}
% Consideriamo ad esempio la parola \(w = aaabbb\). Come prima cosa le applichiamo il carattere terminatore di stringa \(aaabbb\$\). 
% \begin{itemize}
%     \item Partiamo dallo stato \(0\) e inseriamolo nella pila degli stati;
%     \item il primo simbolo che leggiamo in \(w\) è \(a\); vediamo che l'automa presenta una \(a\)-transizione verso lo stato \(2\), per cui la seguiamo, inseriamo lo stato \(2\) nella pila, passiamo oltre al simbolo \(a\) appena "consumato" e passiamo al prossimo simbolo;
%     \item il prossimo simbolo è ancora \(a\); di nuovo, seguiamo la \(a\)-transizione verso lo stato \(5\), lo inseriamo nella pila degli stati, passiamo oltre al simbolo consumato e andiamo avanti;
%     \item abbiamo una terza occorrenza di \(a\) e abbiamo una \(a\)-transizione in forma di self loop in \(5\), che andiamo ad eseguire, reinserendo \(5\) nella pila degli stati passando oltre alla nostra terza \(a\);
%     \item troviamo quindi una \(b\), per cui ci spostiamo allo stato \(8\), il quale ha un'etichetta rossa che riporta un passo di riduzione in forma \(S \to ab, \{b\}\); questo sta a indicare che, se in lettura troviamo \(b\), contenuto nel set \(\{b\}\), possiamo ritornare indietro di due passi, eliminando i due precedenti stati dalla pila e spostarci direttamento dal primo \(5\) a \(7\), dal momento che i due stati sono collegati da una \(S\)-transizione:
%     \begin{align*}
%         \textrm{pila degli stati prima:} &\quad 02558 \\
%         \textrm{pila degli stati dopo:} &\quad 0257 
%     \end{align*}
%     inoltre dobbiamo anche rimuovere gli ultimi due simboli \(ab\) - il body della produzione della riduzione - dalla pila della derivazione e sostituirli con \(S\):
%     \begin{align*}
%         \textrm{pila di derivazione prima:} &\quad \#aaab \\
%         \textrm{pila di derivazione dopo:} &\quad \#aaS 
%     \end{align*}
%     \item leggiamo un'altra \(b\) e avanziamo allo stato \(9\), e anche qui operiamo un passo di riduzione (reduce), nello specifico abbiamo che \(R: S \to aSb, \{b\}\); questo ci dice che dobbiamo tornare indietro di tre passi, eliminando i tre elementi precedenti sia nella pila degli stati e muovendoci verso \(3\), sostituendo nella pila delle derivazioni il body della riduzione con il driver; si osservi attentamente il cambiamento delle pile per capire cosa succede:
%     \begin{align*}
%         \textrm{pila degli stati prima:} &\quad 02579 & \textrm{pila di derivazione prima:} &\quad \#aaSb \\
%         \textrm{pila degli stati dopo:} &\quad 023 & \textrm{pila di derivazione dopo:} &\quad \#aS
%     \end{align*}
%     \item proseguiamo quindi con la lettura e incontriamo una terza \(b\), ci muoviamo verso \(6\) e incontriamo una terza riduzione \(R: S \to aSb, \{\$\}\); di nuovo, torniamo indietro di tre stati e contestualmente sostituiamo gli elementi nelle pile: 
%     \begin{align*}
%         \textrm{pila degli stati prima:} &\quad 0236 & \textrm{pila di derivazione prima:} &\quad \#aSb \\
%         \textrm{pila degli stati dopo:} &\quad 0 & \textrm{pila di derivazione dopo:} &\quad \#S
%     \end{align*}
%     \item abbiamo terminato: ci troviamo nello stato \(0\) e troviamo solamente il nostro start symbol \(S\), che ci permette  muoverci verso lo stato \(1\), e l'endmaker \$; la presenza della keyword \(Accept\) nello stato in cui abbiamo terminato ci indica che la parola è stata riconosciuta dall'automa.
% \end{itemize}

% \subsection{Riassumendo}
% Questo è un esempio del procedimento dell'algoritmo di shift/reduce; vediamo quali regole generali possiamo dedurne:
% \begin{itemize}
%     \item partendo dallo stato iniziale, iniziamo a leggere la parola data attraversando gli archi marchiati dalle \(symbol\)-transizioni che incontriamo di volta in volta;
%     \item quando arriviamo in un nodo in cui si dovrà effettuare un passo di riduzione, questo sarà marcato da un'etichetta che avrà la forma \(A \to B, \{l\}\); a questo punto, dovremo:
%     \begin{itemize}
%         \item eliminare dalla cima della pila della derivazione il body della riduzione;
%         \item mettere al suo posto il driver della riduzione
%         \item eliminare dalla pila degli stati tanti stati quanti i caratteri nel body della derivazione;
%         \item ritornare nello stato che si trova ora in cima alla pila degli stati;
%         \item da qui, effettuare una \(A\) transizione;
%         \item inserire lo stato in cui siamo giunti tramite la \(A\) transizione nella pila degli stati. 
%     \end{itemize}
% \end{itemize}
% Gli automi caratteristici sono una rappresentazione utile, ma si tenga presente che la stessa funzione può essere ottemperata anche da una tabella.

\end{document}
