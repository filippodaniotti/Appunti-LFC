\documentclass[class=book, crop=false, oneside, 12pt]{standalone}
\usepackage{standalone}
\usepackage{../../style}
\usepackage{../../style_automata}
\graphicspath{{./assets/images/}}

% arara: pdflatex: { synctex: yes, shell: yes }
% arara: latexmk: { clean: partial }
\begin{document}
\chapter[Linguaggi regolari e introduzione agli automi]{Linguaggi regolari e introduzione agli automi a stati finiti}

\section{Introduzione}

Ora che abbiamo acquisito dei mezzi più potenti per affrontare questo corso, ribadiamo il concetto di analisi lessicale: dato il programma in ingresso, l'analisi lessicale restituisce una lista di stringhe che corrisponde alle parti del linguaggio identificate; questa lista di stringhe è nota come \emph{flusso dei token}.

Lo studio del capitolo precedente ci consente di sapere come vengono generati linguaggi come il seguente:

\begin{equation}
    \label{a^nb^n}
    \{ a^n b^n \mid n > 0 \}    
\end{equation}

Come detto in passato, un linguaggio con questa forma può essere utilizzato, ad esempio, nel caso in cui vogliamo avere un egual numero di parentesi aperte e chiuse.
Il problema che vogliamo arrivare a risolvere è, più in generale, riconoscere se una stringa faccia parte o meno del linguaggio generato da una certa grammatica.

Ad esempio, per affrontare il problema del linguaggio \ref{a^nb^n} viene naturale l'utilizzo di una struttura di tipo stack:

\begin{enumerate}
    \item leggiamo i simboli uno alla volta;
    \item fino a quando leggiamo \(a\), inseriamoli nella pila;
    \item per ogni successiva \(b\) facciamo un pop dalla pila;
    \item se ad un certo punto troviamo nuovamente il simbolo \(a\) oppure se tentiamo di togliere un elemento da una pila vuota o se, finita l'analisi, rimangono ancora elementi nella pila, allora c'è qualche errore;
    \item nel caso in cui, al termine dell'analisi, sia andato tutto liscio e abbiamo svuotato la pila, allora la parola analizzata appartiene al linguaggio \ref{a^nb^n}.
\end{enumerate} 

La precedente strategia sicuramente è ottima quando il linguaggio ha una forma simile all'esempio che abbiamo considerato. Ma consideriamo invece la grammatica che genera tutte le parole dell'alfabeto:

\begin{equation}
    \label{alfabeto}
    S \to a \mid b \mid … \mid z \mid aS \mid bS \mid … \mid zS
\end{equation}

Per riconoscere parole generate da una tale grammatica il metodo di analisi naturale è una automa a stati finiti come quello in figura \ref{macchina_a_stati_finiti}.

\begin{figure}[H]
	\centering
	\subimport{assets/figures/}{msf.tex}
    \caption{Macchina a stati finiti}
	\label{macchina_a_stati_finiti}
\end{figure}

\noindent Il funzionamento di questi strumenti di analisi ci sarà ben più chiaro in seguito.

La grammatica che produce tutte le lettere dell'alfabeto (il linguaggio \ref{alfabeto}) è una grammatica libera, ma ha anche una caratteristica in più: è regolare. Andiamo a capire meglio di cosa si tratta.


\section{Grammatiche regolari}

Le grammatiche regolari sono un sottoinsieme delle grammatiche libere tale che le loro produzioni sono in una delle seguenti forme:

\begin{itemize}
    \item il body è un solo terminale (\(A \to a\));
    \item il body è composto da un terminale e un non-terminale, nella seguente forma: \(A \to aB\);  
    \item il body è la parola vuota (\(A \to \varepsilon\)).
\end{itemize}

Queste grammatiche possono generare espressioni regolari, che introdurremo a breve; inoltre, i linguaggi generati da queste grammatiche sono riconosciuti dagli automi a stati finiti, sia deterministici che non deterministici.

Queste due osservazioni sono la base fondante dell'utilizzo delle grammatiche regolari per l'analisi lessicale.

\subsection{Espressioni regolari}
Prima di tutto partiamo con il definire un'espressione regolare.\\
Sia fissato un alfabeto \(\mathcal{A}\) da cui estrarre tutte le basi e sia fissato un certo numero di operatori.

Le espressioni regolari sono esprimibili tramite il meccanismo dell'induzione in questo modo.
\begin{labeling}{Step}
    \item[Base] Sono un'espressione regolare tutti i simboli dell'alfabeto che abbiamo scelto; in aggiunta a questi, anche \(\varepsilon\) lo è, indipendentemente dall'alfabeto scelto.
    \item[Step] Se \(r_1\) e \(r_2\) sono espressioni regolari allora:
    \begin{itemize}
        \item \(r_1 \mid r_2\) è un'espressione regolare, detta \emph{alternanza};
        \item \(r_1 \cdot r_2\) è un'espressione regolare, scritta anche come \(r_1 r_2\) e detta \emph{concatenazione};
        \item \(r_1\)\(^\ast\) è un'espressione regolare che significa ripetizione di \(r\) 0 o più volte, detta \emph{Kleene star};
        \item \((r_1)\) è un'espressione regolare; è usata per definire l'ordine di svolgimento delle operazioni ed è detta \emph{parentesi}.
    \end{itemize} 
\end{labeling}

\subsection{I linguaggi delle espressioni regolari}
Se un linguaggio può essere ricavato da un'espressione regolare si dice che l'espressione regolare \emph{denota} quel linguaggio; si presti attenzione a non utilizzare il termine \emph{generare}, poiché quest'ultimo è riservato per le grammatiche generative.

Detto ciò, come capiamo qual è il linguaggio denotato da un'espressione regolare?
Consideriamo un'espressione regolare \(r\) su \(\mathcal{A}\) , il linguaggio denotato da quell'espressione \(\mathcal{L}(r)\) è anch'esso definibile tramite induzione:
\begin{labeling}{Step}
    \item[Base] \begin{itemize}
                    \item \(\mathcal{L}(a) = \{a\} \; \forall a \in \mathcal{A}\)
                    \item \(\mathcal{L}(\varepsilon) = \{\varepsilon\}\)
                \end{itemize}
    \item[Step] \begin{itemize}
                    \item se \(r = r_1 \mid r_2 \) \\
                    allora \(\mathcal{L}(r)= \mathcal{L}(r_1) \cup \mathcal{L}(r_2)\)
                    \item se \(r=r_1 \cdot r_2\) \\
                    allora \(\mathcal{L}(r) = \{w_1 w_2 \mid w_1 \in \mathcal{L}(r_1) \land w_2 \in \mathcal{L}(r_2)\}\)
                    \item se \(r = r_1\)\(^\ast\) \\
                    allora \( \mathcal{L}(r) = \{ \varepsilon \} \cup \{ w_1 w_2 ... w_k \mid k \ge 1 \land \forall i : 1 \le i \le k.w_i \in \mathcal{L}(r_1)\} \)
                    \item se \(r=(r_1)\) \\allora \( \mathcal{L}(r) = \mathcal{L}(r_1)\)
                \end{itemize}
\end{labeling}

\noindent Di seguito l'ordine di precedenza per le operazioni appena descritte.

\begin{equation*}
    \textrm{Kleene Star} \prec \textrm{Concatenazione} \prec \textrm{Alternanza}
\end{equation*}

% \begin{itemize}
%     \item La Kleene star ha la precedenza maggiore;
%     \item La concatenazione è seconda in ordine di precedenza;
%     \item L'alternanza ha la minor precedenza.
% \end{itemize}

\noindent Tutte queste operazioni hanno associatività a sinistra.
Ecco un esempio esplicativo:

\begin{equation*}
    a \mid b c^\ast
\end{equation*}

\noindent Una volta applicate le regole di precedenza, l'espressione si legge in questo modo:

\begin{equation*}
    (a \mid ( b  ( c^\ast ) ) )
\end{equation*}    

\subsubsection{Esercizi sulle operazioni con espressioni regolari}
Ecco presentati in veloce sequenza una serie di semplici esercizi sulle operazioni con grammatiche regolari. 

\begin{itemize}
    \item \(\mathcal{L}(a \mid b) = \{a, b\}\);
    \item \(\mathcal{L}((a \mid b) (a \mid b)) = \{aa, ab, ba, bb\}\);
    \item \(\mathcal{L}(a^*) = \{a^n \mid n \ge 0\}\);
    \item \(\mathcal{L}(a \mid a^\ast b) = \{a\} \cup \{a^n b \mid n \ge 0\}\);
    \item \( (a \mid b \mid \ldots \mid z)(a \mid b \mid \ldots \mid z)^\ast \) denota l'insieme di tutte le parole generabili con l'alfabeto;
    \item \( (0 \mid 1)^\ast 0\) denota l'insieme di tutti i numeri binari pari;
    \item \( b^\ast (abb^\ast )^\ast ( a \mid \varepsilon ) \) denota l'insieme delle parole su \(\{a,b\} \), senza alcuna occorrenza consecutiva di \(a\);
    \item \( (a \mid b)^\ast aa(a \mid b)^\ast \) denota l'insieme delle parole su \( \{a,b\} \) in cui ci sono sicuramente delle occorrenze consecutive di \(a\) (date da \(aa\) in posizione centrale).
\end{itemize}

\section{Automa a stati finiti}
Gli automi a stati finiti sono usati per decidere se le parole appartengono ad un linguaggio denotato da una certa espressione regolare.
Vedremo due tipi diversi di automa a stati finiti: deterministico e non deterministico; di questi tipi poi studieremo i casi di utilizzo ottimali.

Nel caso di automa a stati finiti non deterministico i calcoli sono spesso più pesanti, perché un automa non deterministico deve vagliare più percorsi di derivazione rispetto alla loro controparte deterministica, i quali hanno il vantaggio di dover vagliare solo i percorsi deterministici, che sono un sottoinsieme del totale.

\section{Automa a stati finiti non deterministico}
Un automa a stati finiti non deterministico (Non-Deterministic Finite Automata abbreviato in NFA) è rappresentabile con una tupla
\begin{equation}
    \mathcal{N} := (S, \mathcal{A}, \textrm{move}_n, S_0, F)
    \label{nfa_tupla}
\end{equation}
in cui:
\begin{itemize}
    \item \(S\) è un insieme di stati;
    \item \(\mathcal{A}\) è un alfabeto con \(\varepsilon \notin \mathcal{A}\);
    \item \(s_0 \in S\) è lo stato iniziale;
    \item \(F \subseteq S\) è l'insieme degli stati finali o accettati;
    \item \(\textrm{move}_n : S \times (\mathcal{A} \cup \{\varepsilon\}) \to 2^S\) è la funzione di transizione: da un certo stato e con un certo simbolo (che può essere anche \(\varepsilon\)), mi muovo in un certo sottoinsieme di stati compreso nei sottoinsiemi di \(S\).
\end{itemize}

\subsection{Rappresentazione grafica}
La tupla \((S, \mathcal{A}, \textrm{move}_n, S_0, F)\) può essere rappresentata con un grafo diretto seguendo queste convenzioni:
\begin{itemize}
    \item gli stati rappresentano i nodi;
    \item lo stato iniziale è identificato da una freccia entrante;
    \item gli stati finali sono rappresentati da un doppio cerchio;
    \item le funzioni di transizione rappresentano gli archi, ad esempio \(\textrm{move}_n(S_1, a) = \{S_2, S_3\}\) si rappresenta come in figura \ref{nfa_grafo_esempio}.
\end{itemize}

\begin{figure}[H]
	\centering
	\subimport{assets/figures/}{nfa1.tex}
    \caption{Grafo rappresentante un NFA}
	\label{nfa_grafo_esempio}
\end{figure}

\noindent Si noti anche che, nella rappresentazione grafica, gli archi da \(S_1\) a \(S_2\) ed \(S_3\) vengono denotati da \(a\).

\paragraph{Esempio 1}
Proponiamo ora al lettore il seguente esempio (figura \ref{nfa_grafo_2}):
\begin{figure}[H]
    \centering
    \subimport{assets/figures/}{nfa2.tex}
    \caption{Altro esempio di NFA}
    \label{nfa_grafo_2}
\end{figure}
L'elemento non deterministico è rappresentato dalle due frecce \(a\) uscenti da \(S_0\), che potrebbero portarmi sia in \(S_1\) che in \(S_0\) di nuovo; inoltre, la \(\varepsilon\) introduce sempre indeterminismo, poiché permette di passare a uno stato successivo senza consumare un carattere del terminale (o produrre un carattere nella stringa, che dir si voglia), e quindi inserisce incertezza riguardo al percorso di derivazione compiuto.

Notiamo che dalla rappresentazione grafica si evincono tutti gli elementi che compongono l'NFA; questa è infatti una rappresentazione completa.
Si può anche dare una descrizione tabellare delle funzioni di transizione (\(\textrm{move}_n\)) appartenenti a questo NFA:

\begin{table}[H]
	\centering
	\subimport{assets/tables/}{nfa2-table.tex}
    \caption{Tabella della funzione di transizione per l'automa \ref{nfa_grafo_2}}
    \label{nfa2-table}
\end{table} 

Qual è quindi il linguaggio accettato da un automa di questo tipo?\label{linguaggio_definito_da_un_automa} 
Un NFA \(\mathcal{N}\) accetta (o riconosce) una parola \(w\) se e solo se esiste almeno un cammino che parte dallo stato iniziale di \(\mathcal{N}\) e, seguendo le transizioni dettate da \(w\), termina in uno stato finale per \(\mathcal{N}\).

% Ricorda queste regole di scrittura:
% \begin{itemize}
%     \item \(\varepsilon \varepsilon\) si scrive \(\varepsilon\);
%     \item \(a\varepsilon\) si scrive \(a\);
%     \item \(\varepsilon a\) si scrive \(a\).
% \end{itemize}

Il linguaggio accettato dall'automa è l'insieme di tutte le parole accettate dall'automa.
Ricaviamo, ad esempio, il linguaggio generato dall'automa descritto in Fig.\ref{nfa_grafo_2}:
\begin{itemize}
    \item \(a\) non appartiene al linguaggio, non esiste un percorso che mi porti ad uno stato finale attraversando un solo arco \(a\);
    \item allo stesso modo non posso trovare la parola \(b\) nel linguaggio;
    \item una parola possibile in questo linguaggio è, ad esempio, \(abb\);
    \item tutte le parole nella forma \(\mathcal{L}(a \mid b)^\ast abb\) sono parte del linguaggio.
\end{itemize}

\paragraph{Esempio 2}
Proviamo a determinare il linguaggio dell'NFA descritto in Fig.\ref{nfa_grafo_3}:

\begin{figure}[H]
    \centering
    \subimport{assets/figures/}{nfa3_regex-copy.tex}
    \caption{}
    \label{nfa_grafo_3}
\end{figure}

\noindent Il linguaggio accettato da questo NFA è \(\mathcal{L}((a a^* ) \mid ( b b^\ast ))\).

\section{La costruzione di Thompson}
La costruzione di Thompson è una procedura algoritmica che permette di costruire un automa \(\mathcal{N}\) che genera lo stesso linguaggio denotato da una certa espressione regolare \(r\), ovvero \(\mathcal{L}(\mathcal{N}) = \mathcal{L}(r)\).

\subsection{Definizione}
Anche questa costruzione è espressa in modo induttivo.
\begin{labeling}{Step}
    \item[Base] Ogni simbolo di un qualche alfabeto \(a \in \mathcal{A}\), così come anche la parola vuota \(\varepsilon\), sono un'espressione regolare \(r\); assumiamo di avere sempre un NFA che riconosce \(\mathcal{L}(\varepsilon)\) e uno che riconosce \(\mathcal{L}(a) \; \forall a \in \mathcal{A}\).
    \item[Step] L'espressione regolare \(r\) è una tra le seguenti opzioni: 
    \begin{itemize}[noitemsep]
        \item \(r_1 \mid  r_2 \);
        \item \( r_1 r_2 \); 
        \item \( r_1\) \(^\ast \);
        \item \((r_1)\);
    \end{itemize}
    Dati gli NFA \(\mathcal{N}_1\) e \(\mathcal{N}_2\) tali che \(\mathcal{L}(\mathcal{N}_1)=\mathcal{L}(r_1)\) e \(\mathcal{L}(\mathcal{N}_2) = \mathcal{L}(r_2)\), dobbiamo definire i vari NFA per le quattro operazioni \(r_1 \mid r_2\), \(r_1 r_2\), \(r_1\) \(^\ast\) ed \(r_1\). Le regole per definire questi nuovi NFA sono elencate successivamente.
\end{labeling}
\noindent A questo punto possiamo fare due osservazioni sulla costruzione di Thompson.
\begin{enumerate}
    \item Ogni passo della costruzione introduce al massimo due nuovi stati, vale a dire che l'NFA generato contiene al massimo \(2k\) stati, dove \(k\) è il numero di simboli e operatori nell'espressione regolare.
    \item In ogni stato intermedio dell'NFA c'è esattamente uno stato finale, e inoltre lo stato iniziale non ha nessun arco entrante e lo stato finale non ha alcun arco uscente. 
\end{enumerate}

\subsection{Spiegazione in dettaglio}
Vediamo ora la rappresentazione grafica di questo algoritmo. Partiamo dai passi base.

\begin{figure}[H]
    \centering
    \subimport{assets/figures/}{thompson-base.tex}
    \caption{Thompson nel caso base}
    \label{Thompson_base}
\end{figure}

\noindent La spiegazione dell'algoritmo nel caso base è banale, osserviamo invece con più interesse lo step induttivo.

\begin{figure}[H]
    \centering
    \subimport{assets/figures/}{thompson-step.tex}
    \caption{Thompson nello step induttivo}
    \label{Thompson_step}
\end{figure}
Supponiamo di avere \(\mathcal{N}(r_1)\) e \(\mathcal{N}(r_2)\) descritti come in precedenza, discutiamo le applicazioni di Thompson descritte in figura \ref{Thompson_step}.

\subsubsection{Alternanza}
Guardiamo il primo caso, l'operazione di alternanza \(r = r_1 \mid r_2\).\\
Vogliamo creare un automa che accetta parole di \(\mathcal{L}(r_1)\) o anche parole di \(\mathcal{L}(r_2)\).
Abbiamo i due \(\mathcal{N}\), che rappresentano gli stati intermedi; li posizioniamo uno sopra all'altro. Ora possiamo creare uno stato vuoto e usarlo come stato iniziale per entrambi gli \(\mathcal{N}\), e quindi collegarlo a questi ultimi tramite una \(\varepsilon\)-transizione; allo stesso modo possiamo creare uno stato finale raggiungibile dai due \(\mathcal{N}\) tramite una \(\varepsilon\)-transizione.

\subsubsection{Concatenazione}
Guardiamo ora l'operazione di concatenazione \(r = r_1 r_2\).\\
Per la nostra ipotesi induttiva, sappiamo di poter far affidamento sui due automi \(\mathcal{N}(r_1)\) ed \(\mathcal{N}(r_2)\); in questo momento ci viene in aiuto quella proprietà della costruzione che dice che i passi intermedi (i due \(\mathcal{N}\)) hanno esattamente uno stato finale ed uno stato iniziale, senza alcun arco entrante nello stato iniziale e nessun arco uscente dagli stati finali. 
In questo modo possiamo far coincidere lo stato iniziale di \(\mathcal{N}(r_2)\) con lo stato finale di \(\mathcal{N}(r_1)\). 
Di conseguenza, una parola riesce ad arrivare allo stato terminale blu solo se riesce a passare sia da \(\mathcal{N}(r_1)\) che da \(\mathcal{N}(r_2)\).

\subsubsection{Kleene Star}
Guardiamo il caso della Kleene star \(r = r_1^*\).\\
Vogliamo un automa che riconosce o \(\varepsilon\) o tutte le parole che sono composte dalla ripetizione di altre parole appartenenti al linguaggio \(\mathcal{L}(r_1)\).
Posseggo l'automa \(\mathcal{N}(r_1)\), che ha un solo stato iniziale ed un solo stato finale; aggiungo due stati che serviranno come nuovo stato iniziale e nuovo stato finale.
Siccome voglio poter avere \(\varepsilon\) come parola possibile, introduco un nuovo arco \(\varepsilon\) che collega il nuovo stato iniziale al nuovo stato finale, quindi implemento la ripetizione con degli altri archi \(\varepsilon\) come in figura; la spiegazione è abbastanza banale.

\subsubsection{Parentesi}
Il caso della parentesizzazione \(r = ( r_1 )\) risulta estremamente banale: di fatto l'automa non subisce alcuna modifica.

Questo è il modo di utilizzare il processo di Thompson per costruire automi complessi, ci renderemo conto più avanti di quante \(\varepsilon\) questo ci costringe ad utilizzare e di cosa questo comporti.

\subsection{Applicazione di Thompson}
Presentiamo ora un esempio particolare per assimilare meglio questa costruzione. Analizziamo la seguente espressione regolare

\begin{equation}
    r = (a \mid b)^\ast abb
\end{equation}

\noindent Ci sono quindi due espressioni che compaiono in questo esempio: \(a = r_1\) e \(b = r_2\).
In primis applichiamo la regola di base (\ref{Thompson_base}).

\begin{figure}[H]
    \centering
    \subimport{assets/figures/}{ex-thompson-0.tex}
    \caption{Applicazione del passo base della costruzione di Thompson}
    \label{esempio_Thompson_0}
\end{figure}

Concentriamoci ora sulla prima espressione, \( a \mid b = r_1 \mid r_2 = r_3\); applichiamo la costruzione di Thompson per l'alternanza ai due automi per \(r_1\) e \(r_2\), otteniamo quindi l'automa per \(r_3\).

\begin{figure}[H]
    \centering
    \subimport{assets/figures/}{thompson1_alternanza.tex}
    \caption{Applicazione della costruzione per l'alternanza}
    \label{esempio_Thompson_1}
\end{figure}

La procedura è rappresentata in figura \ref{esempio_Thompson_1} ed è qui brevemente descritta: prendiamo i due automi per \(r_1\) ed \(r_2\), li mettiamo uno sopra l'altro e generiamo due stati (nodo \(1\) e nodo \(6\)); il nodo \(1\) sarà collegato tramite archi \(\varepsilon\) agli stati iniziali di entrambi gli automi, e simmetricamente il nodo \(6\) verrà collegato tramite archi \(\varepsilon\) agli stati finali dei due automi.

Il prossimo passaggio è la traduzione dell'operatore di parentesi, il che è banale, quindi ne riportiamo solo una formulazione nel linguaggio delle espressioni regolari:

\begin{equation*}
    (a \mid b)=(r_3)=r_4
\end{equation*}

\noindent Ora dobbiamo rappresentare la seguente espressione:

\begin{equation*}
    (a\mid b)^\ast = r_4^\ast = r_5
\end{equation*}

\noindent Quindi applichiamo la costruzione di Thompson per la Kleene star all'automa per \(r_4\), ottenendo quindi l'automa per \(r_5\).

\begin{figure}[H]
    \centering
    \subimport{assets/figures/}{thompson2_kleeny.tex}
    \caption{Applicazione della costruzione per la Kleene star}
    \label{esempio_Thompson_2}
\end{figure}

\noindent Continuando con la scansione dell'espressione regolare dobbiamo riscrivere la seguente:

\begin{equation}
    (a \mid b)^\ast a = r_5 r_1 = r_6
\end{equation}

\noindent Applichiamo quindi la costruzione di Thompson per la concatenazione.

\begin{figure}[H]
    \centering
    \subimport{assets/figures/}{thompson3_conc.tex}
    \caption{Applicazione della costruzione per la concatenazione}
    \label{esempio_Thompson_3}
\end{figure}

\noindent Dobbiamo infine riapplicare il passo della concatenazione per le due \(b\) mancanti.

\begin{figure}[H]
    \centering
    \subimport{assets/figures/}{thompson4_conc.tex}
    \caption{Applicazione della costruzione per la concatenazione (II)}
    \label{esempio_Thompson_4}
\end{figure}

Possiamo notare come il risultato cui giungiamo sia pieno di \(\varepsilon\); la costruzione di Thompson ci dà sicuramente un risultato esatto, ma non è detto che sia il risultato migliore, anzi; abbiamo già visto in passato un automa che riconosce lo stesso linguaggio di questo, e quell'automa era più semplice ed elegante (vedi Fig.\ref{nfa_grafo_2}).

Notiamo che comunque l'algoritmo di Thompson garantisce dei limiti superiori sulla complessità dell'automa che genera: ad ogni passo si aggiungono al massimo \(2\) nuovi nodi.
La lunghezza dell'automa finale infatti è limitata da \(2n\), con \(n\) che rappresenta il numero di simboli nell'espressione regolare, dove con simboli si intende sia simboli del vocabolario che operatori.

Anche il numero di archi è limitato, in quanto per ogni passaggio si aggiungono al massimo \(4\) archi.

\section{Il processo di simulazione dell'automa}
Le nostre conoscenze attuali ci permettono di costruire un automa per una certa espressione regolare, ma come si fa a decidere se una certa parola \(w\) fa parte del linguaggio di un dato automa \(\mathcal{N}\)? Si utilizza una procedura detta \emph{simulazione dell'automa}.

Abbiamo già definito in \ref{linguaggio_definito_da_un_automa} come viene deciso se una parola fa parte del linguaggio riconosciuto dall'automa; prendiamo ora ad esempio \(w = bbb\) e l'automa presentato in figura \ref{automa_secondo_esempio}, e presentiamo la procedura di simulazione di un automa.

\begin{figure}[H]
    \centering
    \subimport{assets/figures/}{nfa4_simulazione.tex}
    \caption{Esempio di automa}
    \label{automa_secondo_esempio}
\end{figure}

Se dovessimo fare “a occhio”, questo caso sarebbe molto semplice da risolvere: si parte dallo stato \(S_0\) e si va in un altro stato tramite o archi \(\varepsilon\) o tramite archi \(b\); una volta che raggiungo uno stato con un arco \(b\) elimino il primo carattere della parola e procedo con il secondo, cerco archi con \(\varepsilon\) o con il secondo carattere e così via.
La parola appartiene al linguaggio riconosciuto dall'automa se consumo l'ultimo carattere della parola in uno stato finale.

Noi però stiamo cercando un algoritmo formale che faccia queste operazioni, e che le faccia senza necessità di backtrack per tornare indietro se un percorso si rivela errato, operazione altamente costosa.
Cosa possiamo fare per velocizzare il processo?

Vediamo che leggere \(bbb\) da \(S_0\) è uguale a leggerlo da uno qualunque tra gli stati \(\{S_0, S_1, S_3, S_5\}\), perché arrivo a questi stati solo tramite archi \(\varepsilon\); non potremmo quindi semplificare tutti questi in un solo nodo?

La risposta è sì, possiamo farlo! Per vedere in che modo dobbiamo prima introdurre il concetto di \(\varepsilon\)-chiusura.


\subsection{\(\varepsilon\)-chiusura di un automa}
Sia \((S, \mathcal{A}, \textrm{move}_n, S_0, F)\) un NFA, sia \(t\) uno stato in \(S\) e \(T\) un sottoinsieme di \(S\).

Definiamo \(\varepsilon\)-chiusura\((\{t\})\) l'insieme di stati in \(S\) che sono raggiungibili da \(t\) tramite zero o più \(\varepsilon\)-transizioni (nota che \(t\) è sempre in questo insieme).

Definiamo \(\varepsilon\)-chiusura\((T)\) nel seguente modo:
\begin{equation}
    \varepsilon \textrm{-chiusura}(T) = \bigcup_{t \in T} \;\varepsilon\textrm{-chiusura}(\{t\})
\end{equation} 

\subsubsection{Calcolo della \(\varepsilon\)-chiusura}
Per calcolare la \(\varepsilon\)-chiusura di un nodo di un automa useremo queste strutture:
\begin{enumerate}
    \item uno stack;
    \item un array booleano \texttt{alreadyOn} che ci serve per segnalare se uno stato \(t\) è già sulla pila o meno (la sua dimensione è \(|S|\));
    \item un array bidimensionale per ricordare \(\textrm{move}_n\). Ogni entry \((t,x)\) è una lista concatenata contenente tutti gli stati che sono raggiungibili con un \(x\)-transizione da \(t\).
\end{enumerate}
Ora che sappiamo quali sono le strutture dati che andremo ad utilizzare, studiamo la fase di inizializzazione: 
\begin{itemize}
    \item all'inizio dei tempi non c'è niente sulla pila, poi inseriamo \(t\) e lo segnaliamo su \texttt{alreadyOn};
    \item a questo punto estraiamo dalla cima dello stack, prendiamo il nodo estratto \(e\) e cerchiamo tutti i nodi che da \(e\) si raggiungono tramite un \(\varepsilon\)-arco; inseriamo tutti questi nello stack (se non sono già presenti) e ne segnaliamo l'inserimento settando il loro flag su \texttt{alreadyOn}.
\end{itemize}
Continuiamo così finché lo stack non si svuota. L'algoritmo appena descritto si può trovare in Alg.\ref{alg:ec-computation}.

% \begin{algorithm}[H]
% 	\centering
	\subimport{assets/pseudocode/}{ec-wrapper.tex}
	\subimport{assets/pseudocode/}{ec-computation.tex}
	% \caption{Esempio di parse tree}
% \end{algorithm}

\subsection{Algoritmo della simulazione}
Ora che abbiamo l'arma della \(\varepsilon\)-chiusura possiamo procedere con l'algoritmo per la verifica dell'appartenenza di una parola \(w\) al linguaggio riconosciuto da un automa, ovvero l'algoritmo di simulazione (Alg.\ref{alg:nfa-simulation}).

\subimport{assets/pseudocode/}{nfa-simulation.tex}
Qui \texttt{\$} è l'end-marker per la parola \(w\) (alcuni testi usano la “gratella” invece che \texttt{\$}).
Il funzionamento dell'algoritmo è questo: calcoliamo la \(\varepsilon\)-chiusura di \(S_0\) e la aggiungiamo all'insieme \texttt{states}; prendiamo il primo simbolo nella parola tramite \(nextChar()\) e poi entriamo nel ciclo che compone il corpo della procedura.

In questo corpo si utilizza la \(\varepsilon\)-chiusura per salvare in \texttt{states} la \(\varepsilon\)-chiusura di tutti gli stati che posso raggiungere tramite una \texttt{symbol}-transizione (una transizione marcata con \texttt{symbol}); una volta trovati questi nodi, il \texttt{symbol} prende il valore del prossimo carattere di \(w\).

Una volta che \texttt{symbol} prende il valore di \texttt{\$}, mi fermo e controllo il contenuto di \texttt{states}.

Se \(\textrm{\texttt{states}} \; \cap \; F \neq \emptyset\), allora significa che esiste uno stato in \texttt{states} che è anche uno stato finale dell'automa e quindi abbiamo trovato un percorso che inizia da \(s_0\) riconosce la parola \(w\) e termina in uno stato finale (ammissibile). Ottimo, abbiamo appena dimostrato che \(w\) appartiene al linguaggio riconosciuto dall'automa.

Se invece la condizione precedente non è verificata, allora si è avverato uno dei seguenti casi (o anche entrambi): o non esistono percorsi che riconoscono \(w\), o non esistono percorsi che riconoscono \(w\) e terminano in uno stato finale.

\subsection{Esempi di simulazione}
Applichiamo ora, come esempio, l'algoritmo di simulazione di un NFA dato (Fig.\ref{es-sim-1}), verificando se accetta la parola \(w  = ababb\). Lo svolgimento è rappresentato in Tab.\ref{es-sim-1-tbl}.

\begin{figure}[H]
    \centering
    \subimport{assets/figures/}{thompson4_conc.tex}
    \caption{Esempio di simulazione}
    \label{es-sim-1}
\end{figure}

\begin{table}[H]
	\centering
	\subimport{assets/tables/}{simulazione-nfa-ex1.tex}
    \caption{Tabella risolutiva della simulazione sull'automa \ref{es-sim-1}}
    \label{es-sim-1-tbl}
\end{table} 

\noindent In seguito una breve descrizione della rappresentazione dell'algoritmo.

Inizializziamo la variabile \texttt{states} a \(T0\) inserendo la \(\varepsilon\)-chiusura di \(0\); poi estraiamo il simbolo da aggiungere (il primo simbolo di \(w\)) che è \(a\).

Ora devo comporre l'insieme di stati che posso raggiungere da \texttt{states} con una \(a\)-transizione, e questo insieme è \(\{3, 8\}\). 

Di questo insieme l'algoritmo dice che devo calcolare la \(\varepsilon\)-chiusura, che corrisponde all'insieme nell'ultima colonna della prima riga; questo è il mio nuovo insieme di stati \(T1\) da cui partirò nello step successivo.

Nel secondo step il simbolo che devo aggiungere è la seconda lettera di \(w\), ovvero \(b\), quindi vado a cercare quei nodi che posso raggiungere da \(T1\) con una \(b\)-transizione; questi nodi sono \(\{5,9\}\), e una volta che li ho trovati ne calcolo la \(\varepsilon\)-chiusura, che compone \(T2\) e via così.

Continuo così finché non finisco i simboli; a quel punto devo verificare se esiste nel mio insieme \(states\) uno stato finale. In questo caso è presente (\(10\)), e quindi la parola \(w\) appartiene al linguaggio riconosciuto dall'automa raffigurato.


\subsection{Nota sulla \(\varepsilon\)-chiusura}
\begin{theorem}
    Sia \(\mathcal{N} = (S, \mathcal{A}, \textrm{move}_n, S_0, F)\) un NFA e sia \(M \subseteq S\).
    
    Allora la \(\varepsilon\)-chiusura\((M)\) è il più piccolo insieme \(X \subseteq S\) tale che \(X\) è una soluzione alla seguente equazione:
    \begin{equation}
        X = M \cup \{ N' \mid N \in X \land N' \in \textrm{move}_n (N,\varepsilon)\}
        \label{eps-closure-set-eq}
    \end{equation}
\end{theorem}

\noindent Nota che diciamo il più piccolo insieme per evitare di proseguire in loop infiniti come quello rappresentato in figura \ref{nfa_ciclico}.

\begin{figure}[H]
    \centering
    \subimport{assets/figures/}{nfa5_e-closure.tex}
    \caption{Se non scegliessimo il più piccolo \(X\) potremmo incorrere in cicli infiniti.}
    \label{nfa_ciclico}
\end{figure}

Osserviamo con attenzione la formula appena descritta: \(X\) è \(M\) stesso, unito anche a tutti gli stati \(N'\) raggiungibili da \(N\) tramite una \(\varepsilon\)-transizione, dove \(N\) è uno stato qualsiasi in \(M\).

Balza subito all'occhio come la formula per definire \(X\) dipenda da \(X\) stessa; per questo motivo non dovremmo poterla calcolare, giusto? 
Sbagliato! In informatica possiamo risolvere queste equazioni date certe caratteristiche, ed è qui che giunge in nostro aiuto il teorema del punto fisso. 


\subsection{Teorema del punto fisso}
L'equazione \ref{eps-closure-set-eq} vista poco fa è un'istanza di una più generale equazione su insiemi, della forma \(X = f(X)\), sulla quale possediamo un risultato importante.


Sia \(f: 2^D \to 2^D\) per qualche insieme finito \(D\) e sia inoltre \( f \) monotona, vale a dire
\begin{equation*}
    X \subseteq Y \implies f(X) \subseteq f(Y)
\end{equation*}
allora esiste una precisa tecnica, basata su approssimazioni successive, per risolvere l'equazione \(X = f(X)\). Adesso vediamo meglio il teorema cui fa riferimento.

\begin{theorem}
    Sia \(S\) un insieme finito e sia \(f: 2^S \to 2^S\) una funzione monotona; allora \(\exists m \in \mathbb{N}\) tale che esiste un'unica soluzione minima all'equazione \(X = f(X)\), e questa soluzione è \(f^m(\varnothing)\).
\end{theorem}

\begin{proof}
    I due asserti andranno dimostrati separatamente; come prima cosa, vogliamo dimostrare che \(\exists m \in \mathbb{N}\) tale che \(f^m(\varnothing)\) è soluzione per \(X = f(X)\); prima di procedere è inoltre consigliabile avere ben presente la forma della funzione \(f(X)\) si veda \ref{eps-closure-set-eq}.

    Per definizione stessa di \(f(X)\), possiamo subito dire che \(\varnothing \subseteq f(\varnothing)\) e, poiché \(f(X)\) è monotona, possiamo subito dire anche che \(\varnothing \subseteq f^2(\varnothing)\). Quindi, per il principio di induzione, possiamo affermare senza timore di smentita che \(f^i(\varnothing) \subseteq f^{i + 1}(\varnothing), \forall i \in \mathbb{N} \). A questo punto, possiamo quindi dire di avere una sorta di catena di relazioni tra insiemi:
    \begin{equation*}
        f(\varnothing) \subseteq f^2{\varnothing} \subseteq f^3(\varnothing) \subseteq \ldots
    \end{equation*}
    Questa successione è infinita, ma \(2^S\) è definito come finito, per cui gli insiemi della successione non possono essere tutti diversi l'uno dall'altro! Arriveremo infatti, a un certo punto, a non riuscire più a osservare cambiamenti a successive applicazioni di \(f(X)\); formalmente, troveremo un qualche indice \(m\) tale che:
    \begin{equation*}
        f^m(\varnothing) = f^{m + 1} (\varnothing) = f(f^m(\varnothing))
    \end{equation*}
    Si dice che siamo arrivate al punto di \emph{saturazione}; in questa situazione, possiamo dire che \(f^m(\varnothing)\) è una soluzione per \(X = f(X)\).

    Andiamo adesso a dimostrare che \(f^m(\varnothing)\) è l'unica soluzione minima.\\
    Per assurdo, supponiamo che esista un'altra soluzione \(A\) per \(X = f(X)\); quindi, per ipotesi, deve valere \(A = f(A)\), e quindi dovremo avere che \(A = f(A) = f^2(A) = \ldots = f^m(A)\). Sappiamo anche che \(f^m(\varnothing) \subseteq f^m(A)\), poiché naturalmente l'insieme vuoto è compreso in qualsiasi insieme (\(\varnothing \subseteq A\)) e la funzione \(f\) è monotona. Ma allora possiamo mettere assieme le due precedenti osservazioni e affermare quindi che \(f^m(\varnothing) \subseteq A\), poiché \(f^m(\varnothing) \subseteq f^m(A)\) e \(A = f^m(A)\). Per cui \(f^m(\varnothing)\) è una soluzione unica, ed è anche l'unica minima. 

\end{proof}

Nonostante a prima vista il significato o anche la stessa utilità di questo risultato possa non essere del tutto chiaro, vale la pena di vederlo almeno una volta, poiché tutta la teoria che riguarda la semantica dei linguaggi di programmazione è basata su questo teorema. 

Ad esempio, l'esecuzione di una funzione fattoriale: essa è costituita di una sequenza di costrutti \texttt{while} tali da produrre il risultato per l'input desiderato; ogni iterazione del \texttt{while} calcola un'approssimante del risultato, che viene quindi ricostruito piano piano. È chiaro che in questo esempio non stiamo aggiungendo insiemi ma valori interi, però è utile pensarlo per avere quantomeno un'idea intuitiva di come il teorema del punto fisso sia applicato alla teoria della semantica dei linguaggi di programmazione.

\section{Considerazioni sull'efficienza degli algoritmi sugli NFA}
All'inizio della trattazione sugli automi non deterministici, il nostro obiettivo era riconoscere un linguaggio regolare attraverso degli automi a stati finiti; in particolare, abbiamo detto che l'analisi lessicale utilizzerà le espressioni regolari per identificare i vari elementi del programma scritto. Dalle espressioni regolari vogliamo quindi avere degli strumenti che ci permettono di prendere come input il testo del programma e ottenere in output un flusso di token, che saranno i terminali della grammatica che ha generato il nostro linguaggio.

Ci chiediamo quale sia la complessità degli algoritmi che abbiamo visto per risolvere questa richiesta, anche in vista del momento in cui ci troveremo a scegliere quale tipo di automa scegliere (NFA o DFA) per assolvere diverse tipologie di compiti.

\paragraph{Sunto delle procedure viste}
Data una parola e data un'espressione regolare che denota quello stesso linguaggio, si dica se la parola appartiene a quel particolare linguaggio oppure no. Gli algoritmi che abbiamo visto ci consento di rispondere a questa richiesta applicando le seguenti operazioni:
\begin{itemize}
    \item consideriamo un'espressione regolare \(r\);
    \item applichiamo la costruzione di Thompson e generiamo un NFA che riconosce esattamente il linguaggio denotato da \(r\);
    \item lancia l'algoritmo di simulazione per l'NFA generato.
\end{itemize}
Andiamo quindi ad analizzare la complessità di queste procedure.

\subsection{Complessità della costruzione di Thompson}
Consideriamo la  complessità di generazione di un NFA con \(n\) nodi e \(m\) archi; conoscendo le quattro operazioni, sappiamo che, per ogni passo, aggiungeremo al massimo \(2\) nodi e \(4\) archi (dalla Kleene star, l'operazione più costosa), per cui avremo che \(n \le 2|r|\) e \(m \le 4|r|\), per cui abbiamo che la somma \(n + m\) è dell'ordine della dimensione dell'espressione regolare (\(\mathcal{O}(|r|)\)) e, poiché avremo in totale \(|r|\) passi, ciascuno dei quali eseguibile in tempo costante, possiamo felicemente concludere che \(T(\textrm{Thompson}(r)) = \mathcal{O}(|r|)\).

\subsection{Complessità del calcolo della \(\varepsilon\)-chiusura}
Prima di analizzare il costo della simulazione, dobbiamo analizzare il costo della \(\varepsilon\)-chiusura. Tenendo sempre a mente il codice (Alg.\ref{alg:ec-wrapper} e Alg.\ref{alg:ec-computation}), evidenziamo che le strutture usate sono:
\begin{itemize}
    \item uno stack per contenere gli elementi non ancora incontrati;
    \item un vettore booleano per tenere traccia dei visitati;
    \item \(\textrm{move}_n\), al solito un vettore bidimensionale di puntatori a liste linkate, ciascuna contenente tutti gli stati raggiungibili a partire da uno stato \(t\) tramite un simbolo \(x\).
\end{itemize} 
In sunto, la procedura si compone dei seguenti quattro passi:

\begin{enumerate}
    \item inserimento del nodo \(t\) nello stack;
    \item imposta \texttt{alreadyOn[t] = true};
    \item trova un successivo nodo \(u \in \textrm{move}_n(t, \varepsilon)\);
    \item verifica se è già stato visitato con \texttt{alreadyOn[u]}.
\end{enumerate}

\noindent Ognuna di queste operazioni opera in tempo costante, e ci chiediamo quindi quante volte venga ripetuto ognuna di queste.

Osserviamo che le prime due operazioni vengono ripetute a ogni chiamata della procedura, e mai più di una volta per ogni nodo, dal momento che il vettore \texttt{alreadyOn} non ci permetterà di richiamare la funzione se lanciato su un nodo già settato a \texttt{true}, e non capita mai che qualche bit venga reinizializzato a \texttt{false}. Per cui, il costo delle prime due operazioni è \(\mathcal{O}(n)\).

La terza e la quarta operazione vengono eseguite per ogni nodo raggiungibile con una \(\varepsilon\)-transizione, per cui possiamo immaginare un caso pessimo in cui ogni stato possiede almeno una \(\varepsilon\)-transizione; nel caso peggiore, quindi, andremo a visitare ogni arco del grafo (\(\mathcal{O}(m)\)).

Complessivamente, il costo della procedura è \(\mathcal{O}(n + m)\).

\subsection{Complessità della simulazione di NFA}
Adesso possiamo affrontare il problema dello studio della complessità della simulazione di NFA. Iniziamo con un veloce ripasso delle strutture dati utilizzate: si tenga presente che, per tenere traccia della variabile \texttt{states}, utilizzeremo ben due stacks:

\begin{itemize}
    \item il primo (\texttt{currentStack}) è nel right value della riga \(4\) e viene utilizzato per conservare il contenuto del vettore \texttt{states} durante l'attuale iterazione dell'algoritmo;
    \item il secondo (\texttt{nextStack}) è invece  nel left value e verrà aggiornato durante la attuale iterazione.
\end{itemize}

\noindent Abbiamo anche il nostro vettore \texttt{alreadyOn} di dimensione \(|S|\) e, naturalmente, il vettore bidimensionale per conservare \(\textrm{move}_n\).

Ci rendiamo conto che il ciclo \texttt{while} domina la complessità, soprattutto perché al suo interno avviene una chiamata di \(\varepsilon\)-chiusura, completata dall'estrazione di tutti gli elementi da \texttt{nextStack} e il loro inserimento in \texttt{currentStack}, oltre alla reinizializzazione di \texttt{alreadyOn}.

\subimport{assets/pseudocode/}{nfa-sim-line4.tex}

All'interno di ogni \texttt{while}, quindi, abbiamo uno swap degli stack, che costa \(\mathcal{O}(n)\), poiché è limitato dal numero di nodi dell'NFA; inoltre, ogni stato può essere inserito solo una volta nella pila nel secondo \texttt{foreach}; quindi, ogni ciclo del \texttt{while} costa \(\mathcal{O}(n + m)\). Questo ciclo verrà lanciato per ogni elemento della parola \(w\) analizzate, pertanto il costo complessivo della simulazione è \(\mathcal{O}(|w|(n + m))\); dal momento che il nostro NFA è stato generato da un'esecuzione di costruzione di Thompson, avremo anche che \(n + m = \mathcal{O}(|r|)\), per cui possiamo concludere che il costo  dell'intera procedura di simulazione è dell'ordine di \(\mathcal{O}(|w||r|)\).

Il costo è sicuramente più basso di quello che avrei facendo del backtrack, dal momento che si parla comunque di un modello non deterministico. 

\section{Automa a stati finiti deterministico}
Non c'è modo migliore di introdurre questi automi se non per differenza rispetto agli NFA. Si tenga a mente la definizione di NFA:

\begin{gather*}
    \mathcal{N} := (S, \mathcal{A}, \textrm{move}_n, S_0, F), \textrm{ dove } \\
    \textrm{move}_n : S \times (\mathcal{A} \cup \{\varepsilon\}) \to 2^S
\end{gather*}

Si noti la funzione di transizione, che prende come input uno stato e un simbolo da \(\mathcal{A} \cup \varepsilon\), dove è la presenza stessa di \(\varepsilon\) a causare indeterminatezza; inoltre, un qualunque nodo di un grafo ha tendenzialmente più archi uscenti identificati dal medesimo simbolo.

Un automa a stati finiti deterministici è invece definito come segue:

\begin{gather}
    \mathcal{D} := (S, \mathcal{A}, \textrm{move}_d, S_0, F), \textrm{ dove } \\
    \textrm{move}_d : S \times \mathcal{A} \to S \notag
\end{gather}

Balza subito all'occhio la differente formulazione della funzione di transizione \(move_d\), il cui dominio esclude la possibilità di compiere \(\varepsilon\)-transizioni. Inoltre, quest'ultima ha due modi di presentarsi: totale o parziale, e il possesso dell'una o dell'altra qualificazione inficerà la scelta delle procedure da applicare.

Evidenziamo le caratteristiche dei DFA:

\begin{itemize}
    \item non presentano \(\varepsilon\)-transizioni (ma c'è un cavillo di cui ci occuperemo più avanti);
    \item se \(\textrm{move}_d\) è \emph{totale}, allora per ogni stato c'è \textbf{esattamente} una \(a\)-transizione \(\forall a \in \mathcal{A}\);
    \item se \(\textrm{move}_d\) è \emph{parziale}, allora per ogni stato c'è \textbf{al massimo} una \(a\)-transizione \(\forall a \in \mathcal{A}\); in altre parole, per alcune coppie del dominio (\((s,a) \in S \times A\)) la funzione non è definita.
\end{itemize}

\begin{figure}[H]
    \begin{minipage}[b]{0.4\textwidth}
        \centering
        \subimport{assets/figures/}{dfa_total_transition.tex}
        \subcaption{Esempio di DFA con funzione di transizione totale}
    \end{minipage}
    \hfill
    \begin{minipage}[b]{0.4\textwidth}
        \centering
        \subimport{assets/figures/}{dfa_partial_transition.tex}
        \subcaption{Esempio di DFA con funzione di transizione parziale}
    \end{minipage}
    \caption{Funzioni di transizione dei DFA}
\end{figure}

\subsection{Linguaggi riconosciuti dai DFA}
Prendiamoci un certo DFA \(\mathcal{D} = (S, \mathcal{A}, \textrm{move}_d, S_0, F)\); il linguaggio \(\mathcal{L(D)}\) da lui riconosciuto è l'insieme di parole \(w\) tale che:

\begin{itemize}
    \item o esiste un cammino, che chiamiamo qui \(w = a_1, \ldots, a_k\), con \(k \ge 1\), che vada dallo stato iniziale \(S_0\) a un qualche stato finale in \(F\);
    \item oppure vale che \(S_0 \in F \cap w = \varepsilon\); per l'appunto, quello in cui lo stato di partenza è uno stato finale è l'unico caso in cui un DFA riconosce la parola vuota.
\end{itemize}

\subsection{Simulazione di un DFA con transizione totale}
In questo caso abbiamo la certezza che, per ogni  simbolo che leggiamo, ci sia un qualche arco che lo colleghi a uno stato nel grafo, e questo semplifica di molto la procedura per determinare se una certa parola \(w\) appartenga o meno al linguaggio del nostro DFA.

Per l'appunto, è sufficiente partire dallo stato iniziale \(S_0\) e seguire il cammino dato dagli elementi di \(w\); se terminiamo in un qualche stato finale in \(F\), allora \(w\) appartiene al linguaggio, altrimenti no; la complessità, pertanto, è \(\mathcal{O}(|w|)\).

\subsection{Simulazione di un DFA con transizione parziale}
Poiché qui abbiamo una funzione di transizione parziale, dobbiamo considerare la possibilità di arrivare a uno stato in cui non non possiamo più proseguire. 

Iniziamo comunque dallo stato iniziale e, di nuovo, seguiamo il cammino dato dallo spelling della parola \(w = a_1, a_2, \ldots, a_k\); se stiamo leggendo un qualche simbolo per cui non c'è una transizione "etichettata" da quel simbolo, allora possiamo direttamente ritornare una risposta negativa; se invece riusciamo a leggere tutta la parola e a raggiungere uno stato finale, possiamo dire che \(w\) appartiene al linguaggio.

\subsection{Funzioni di transizioni a confronto}
Dato un DFA \(\mathcal{D}\) con funzione di transizione parziale, ho un modo per definire un DFA \(\mathcal{D}'\) che abbia funzione di transizione totale e che riconosca esattamente lo stesso linguaggio (\(\mathcal{L(D) = \mathcal{L}(\mathcal{D}')}\))?

Sì, posso farlo con l'uso di un \emph{sink} (anche detto dead state, o in italiano stato trappola). In sostanza vado a creare un nuovo stato non finale che aggiungo agli altri stati di \(\mathcal{D}\), e questo stato sarà la destinazione di tutte le transizioni non definite dalla funzione di transizione; infine, per ogni simbolo dell'alfabeto, aggiungo al sink un \emph{self loop} su quel simbolo.

In questo modo, quando vado a trovarmi in quelle situazioni in cui non potevo più proseguire, posso invece proseguire e andare nel sink; in questo modo potrò comunque continuare, esaurire la parola in uno stato non finale e quindi tornare un risultato coerente al linguaggio considerato.

In generale si preferisce lavorare con DFA con funzione di transizione parziale, e ancora più in generale con automi che abbiano il minor numero di stati (nodi e archi) possibile; tuttavia, l'algoritmo di minimizzazione (vedremo più avanti), che usiamo per "ridurre al minimo" il numero di archi e nodi, non funziona con DFA dotati di funzione di transizione parziale. Il motivo sta nel fatto che questa procedura lavora sulla funzione inversa di \(\textrm{move}_d\), che chiaramente non è definita se la funzione non è totale.

\end{document}
