\documentclass{standalone}

\usepackage{booktabs}
\usepackage{amsmath}
\usepackage{tabularx}
\usepackage[table]{xcolor}

\setlength\lightrulewidth{0.1pt}
\providecommand\lightrule{%
	\arrayrulecolor{black!30}%
	\midrule[\lightrulewidth]%
	\arrayrulecolor{black}}

% arara: pdflatex
% arara: latexmk: { clean: partial }
\begin{document}
\begin{tabularx}{\textwidth}{X|X|X}
		Stati (deterministici) & \(\varepsilon\)-chiusura dei punti di arrivo delle \(a\)-transizioni & \(\varepsilon\)-chiusura dei punti di arrivo delle \(b\)-transizioni \\
    \midrule
        Stato iniziale n.d. : 0 \newline
        \(T0 = e-cl(\{0\}) = \{0, 1, 2, 4, 7\}\)
        &
        Stati tramite \(a\)-transizione \{3,8\} \newline
        \(\varepsilon-cl(\{3,8\}) = \{1,2,3,4,6,7,8\} = T1\) \newline
        [T1 unmarked]
        &
        Stati raggiunti tramite \(b\)-transizione \{5\} \newline
        \(\varepsilon-cl(\{5\}) = \{1,2,4,5,6,7\} = T2\) \newline
        [T2 unmarked]
        \\ \lightrule
        \(T1 = \{1,2,3,4,6,7,8\}\)
        &
        \(\{3, 8\}\) \newline
        \(\varepsilon-cl(\{3,8\}) = T1\) \newline
        [T1 già analizzato]
        &
        \(\{5, 9\}\) \newline
        \(\varepsilon-cl(\{5,9\}) = \{1,2,4,5,6,7,9\} = T3\) \newline
        [T3 unmarked]
        \\ \lightrule
        \(T2 = \{1,2,4,5,6,7\}\)
        &
        \(\{3,8\}\) \newline
        [T1 già analizzato]
        &
        \(\{5\}\) \newline
        [T2 già analizzato]
        \\ \lightrule
        \(T3 = \{1,2,4,5,6,7,9\}\)
        &
        \(\{3,8\}\) \newline
        [T1 già analizzato]
        &
        \(\{5, 10\}\) \newline
        \(\varepsilon-cl(\{5, 10\}) = \{1,2,4,5,6,7,10\} = T4\)
        [T4 unmarked]
        \\ \lightrule
        \(T4 = \{1,2,4,5,6,7,10\}\)
        &
        \(\{3,8\}\) \newline
        [T1 già analizzato]
        &
        \(\{5\}\) \newline
        [T2 già analizzato]
        \\
\end{tabularx}
\end{document}
